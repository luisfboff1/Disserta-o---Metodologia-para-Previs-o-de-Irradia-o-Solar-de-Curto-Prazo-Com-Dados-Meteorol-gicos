\section{Estado da Arte}
\label{sec:estado-da-arte}

Para fundamentar este trabalho, foi conduzida uma revisão bibliográfica nas bases \textit{Scopus} e \textit{Google Scholar}, utilizando palavras-chave relacionadas à previsão de irradiação solar: \textit{solar radiation}, \textit{irradiance}, \textit{prediction}, \textit{forecast}, \textit{short-term}, \textit{meteorological data}. A busca retornou 39.208 documentos. Após filtragens (remoção de trabalhos anteriores a 1970 e itens não relacionados diretamente à irradiação/energia solar), obteve-se um conjunto consolidado de 15.317 referências únicas.

\subsection{Produção científica e distribuição geográfica}

A Figura~\ref{fig:pubs_ano} mostra a evolução anual das publicações. Observa-se crescimento moderado até meados de 2010, seguido de uma aceleração a partir de 2015, compatível com a popularização de técnicas de \textit{deep learning} e maior disponibilidade de séries meteorológicas/reanálises. Esse comportamento sugere aumento de interesse tanto em previsões de curto prazo para operação quanto em horizontes maiores para planejamento energético.

\begin{figure}[!h]
    \centering
    \caption{Publicações ao longo do tempo na área de previsão de irradiação solar.}
    \includegraphics[width=0.9\textwidth]{2. Revisão Bibliográfica/Figuras/publicações ao longo do ano.png}
    \par\small{Fonte: Autor (2025)}
    \label{fig:pubs_ano}
\end{figure}


A Figura~\ref{fig:paises} apresenta os quinze países com maior volume de publicações. Estados Unidos e China lideram, seguidos pela Índia e países europeus. O Brasil aparece em 13º lugar, indicando presença relevante da comunidade nacional, mas ainda com espaço para ampliação de esforços em previsões de curto prazo de irradiância solar.

\begin{figure}[!h]
    \centering
    \caption{Top 15 países em número de publicações.}
    \includegraphics[width=0.9\textwidth]{2. Revisão Bibliográfica/Figuras/top paises.png}
    \par\small{Fonte: Autor (2025)}
    \label{fig:paises}
\end{figure}


\subsection{Modelos de previsão utilizados e sua evolução temporal}

O mapeamento por dicionário de sinônimos revelou os modelos com maior recorrência, como mostra a Figura~\ref{fig:modelos_bar}. ANN/MLP e LSTM dominam, seguidos por SVR/SVM, Random Forest e CNN. Em seguida, algoritmos de \textit{boosting} (XGBoost/GBDT), métodos clássicos de séries (ARIMA/SARIMA) e persistência aparecem como referências frequentes.

\begin{figure}[!h]
    \centering
    \caption{Modelos mais utilizados na literatura levantada.}
    \includegraphics[width=0.9\textwidth]{2. Revisão Bibliográfica/Figuras/Modelos mais utilizados.png}
    \par\small{Fonte: Autor (2025)}
    \label{fig:modelos_bar}
\end{figure}


A Figura~\ref{fig:modelos_tempo} detalha a evolução temporal dos cinco modelos mais citados. Observa-se:
\begin{itemize}
    \item \textbf{ANN/MLP}: presença constante desde os anos 2000, com função de \textit{baseline} não linear e uso recorrente em combinações/ensembles.
    \item \textbf{LSTM}: crescimento pronunciado após 2015, associado à capacidade de modelar dependências de longo alcance e não linearidades em séries meteorológicas.
    \item \textbf{SVR/SVM}: pico intermediário e estabilidade posterior; segue competitivo em bases menores e com seleção cuidadosa de atributos.
    \item \textbf{Random Forest}: desempenho sólido em dados tabulares com variáveis meteorológicas, bom como referência robusta.
    \item \textbf{CNN}: curva ascendente recente, especialmente em \textit{nowcasting}/curtíssimo prazo ou quando convoluções 1D capturam padrões locais da série.
\end{itemize}

\begin{figure}[!h]
    \centering
    \caption{Evolução temporal (contagem anual) dos cinco modelos mais utilizados.}
    \includegraphics[width=0.9\textwidth]{2. Revisão Bibliográfica/Figuras/evolução modelos.png}
    \par\small{Fonte: Autor (2025)}
    \label{fig:modelos_tempo}
\end{figure}


\subsection{Horizontes de previsão}

A Figura~\ref{fig:horizontes} mostra os horizontes mais abordados. O horizonte de \textbf{60 minutos} concentra a maior parte dos estudos, seguido pelo \textbf{diário} e, depois, pelo \textbf{15 minutos}. O domínio de 60 min aparece, em parte, pelo uso recorrente do modelo de \textit{persistência} como \textit{baseline} e por atender tanto cenários operacionais quanto testes metodológicos padronizados. O horizonte de 15 min, embora menos frequente, apresenta tendência de crescimento e possui forte potencial de aplicação em \textit{smart grids} e em sistemas de gerenciamento de carga residencial (\textit{HEMS}). Essa granularidade permite capturar variações rápidas da irradiância ao longo do dia, o que possibilita decisões mais precisas no acionamento de cargas, no uso de armazenamento e na resposta da rede a flutuações locais de geração fotovoltaica.


\begin{figure}[!h]
    \centering
    \caption{Horizontes de previsão mais utilizados.}
    \includegraphics[width=0.9\textwidth]{2. Revisão Bibliográfica/Figuras/horizontes mais utilizados.png}
    \par\small{Fonte: Autor (2025)}
    \label{fig:horizontes}
\end{figure}


\subsection{Variáveis de entrada e métricas de avaliação}

A Figura~\ref{fig:variaveis} apresenta as variáveis mais empregadas. Nota-se \textbf{temperatura} com incidência muito superior, seguida de \textbf{umidade}, \textbf{velocidade do vento} e medidas radiométricas como \textbf{GHI}. Também aparecem \textbf{precipitação}, \textbf{pressão}, \textbf{nuvens}, \textbf{aerossóis} e insumos de reanálises/satélite (ERA5, MERRA, CAMS, etc.). Em termos de causalidade física, cobertura de nuvens, aerossóis e modelos de céu claro (\textit{clear\_sky}) têm papel direto na atenuação/variabilidade da irradiação; já variáveis termodinâmicas (temperatura, umidade) frequentemente atuam como proxies de condições de nebulosidade/estabilidade, o que explica sua ampla adoção. Esse cenário reforça a importância de seleção de atributos (\textit{feature selection}) e normalização temporal/estacional para evitar sobreajuste.

\begin{figure}[!h]
    \centering
    \caption{Variáveis mais utilizadas como preditoras.}
    \includegraphics[width=0.9\textwidth]{2. Revisão Bibliográfica/Figuras/variaveis mais utilizadas.png}
    \par\small{Fonte: Autor (2025)}
    \label{fig:variaveis}
\end{figure}


A Figura~\ref{fig:metricas} apresenta as métricas de desempenho mais recorrentes. Observa-se destaque expressivo do \textbf{RMSE}, amplamente empregado como medida de erro quadrático médio e por sua interpretação direta em termos físicos da variável prevista. Em seguida aparecem o \textbf{MSE} e o \textbf{coeficiente de determinação ($R^2$)}, frequentemente utilizados em conjunto para avaliar simultaneamente a magnitude dos erros e a proporção da variabilidade explicada pelo modelo. Outras métricas, como \textbf{MAE}, \textbf{MAPE}, \textbf{MBE} e indicadores normalizados (\textbf{nRMSE}), surgem em menor escala, geralmente para complementar a análise. A ênfase em RMSE, MSE e $R^2$ reflete a busca por \textit{benchmarks} comparáveis na literatura, permitindo avaliação justa entre diferentes abordagens.

\begin{figure}[!h]
    \centering
    \caption{Métricas de avaliação mais utilizadas.}
    \includegraphics[width=0.9\textwidth]{2. Revisão Bibliográfica/Figuras/Métricas mais utilizados.png}
    \par\small{Fonte: Autor (2025)}
    \label{fig:metricas}
\end{figure}


\subsection{Associações entre modelos, horizontes e variáveis}

As matrizes de associação de modelos por horizonte e modelos por variáveis, conforme as Figuras~\ref{fig:assoc_mh} e~\ref{fig:assoc_mv}, respectivamente, indicam padrões consistentes:
\begin{itemize}
    \item \textbf{Persistência} destacada em \textbf{60 min} e \textbf{diário}, caracterizando o uso como \textit{baseline} universal.
    \item \textbf{LSTM} com alta associação em \textbf{60 min} e \textbf{diário}, compatível com séries não lineares e dependências de médio prazo; presença relevante também quando combinações de variáveis meteorológicas são ricas.
    \item \textbf{ARIMA/SARIMA} mais forte em \textbf{60 min}, favorecido por sazonalidade clara e estruturas AR que capturam autocorrelação de curto prazo.
    \item \textbf{ANN/MLP} com bom desempenho em \textbf{60 min} e \textbf{diário}, muitas vezes como parte de \textit{ensembles}/híbridos.
    \item \textbf{CNN} associada de forma mais evidente ao \textbf{15 min}, coerente com extração de padrões locais/rápidos por convoluções 1D e uso em \textit{nowcasting}.
\end{itemize}
Quanto às variáveis, os modelos baseados em aprendizado de máquina (LSTM, ANN, CNN, XGBoost) tendem a aproveitar conjuntos multivariados (temperatura, umidade, vento, nuvens, índices de céu claro e reanálises/satélite), enquanto \textit{time-series} clássicos (ARIMA) operam melhor com transformações da própria série de irradiação (e variações sazonais), por vezes auxiliados por \textit{exogenous regressors} simples.

\begin{figure}[!h]
    \centering
    \caption{Associação (normalizada) entre modelos e horizontes.}
    \includegraphics[width=0.95\textwidth]{2. Revisão Bibliográfica/Figuras/MC modelos x horizonte.png}
    \par\small{Fonte: Autor (2025)}
    \label{fig:assoc_mh}
\end{figure}

\begin{figure}[!h]
    \centering
    \caption{Associação (normalizada) entre modelos e variáveis.}
    \includegraphics[width=0.95\textwidth]{2. Revisão Bibliográfica/Figuras/MC modelos x variaveis.png}
    \par\small{Fonte: Autor (2025)}
    \label{fig:assoc_mv}
\end{figure}

% \newpage
\FloatBarrier

\subsection{Síntese crítica e direcionamento}

Os resultados indicam três pontos práticos para projetos de previsão:
\begin{enumerate}
    \item \textbf{Horizonte de 15 min}: menos explorado que 60 min e diário, porém em crescimento — alinhado a necessidades de operação (\textit{rampas} rápidas de GHI/PV, despacho de armazenamento, \textit{demand response} e qualidade de energia).
    \item \textbf{Modelagem}: \textbf{LSTM} e \textbf{CNN} são adequados a dinâmicas rápidas e não lineares; \textbf{XGBoost} é competitivo em dados tabulares multivariados; \textbf{ARIMA} funciona como referência de série; \textbf{Persistência} permanece como \textit{baseline} mandatória para comparação justa.
    \item \textbf{Variáveis}: combinar medidas radiométricas (GHI/DNI/DHI), proxies de nebulosidade (nuvens, aerossóis, \textit{clear\_sky}) e variáveis meteorológicas (temperatura, umidade, vento) tende a melhorar robustez; o uso de reanálises/satélite (ERA5, CAMS etc.) amplia cobertura temporal e espacial quando redes de piranômetros são esparsas.
\end{enumerate}

Assim, há \textbf{espaço claro} para contribuir com um estudo focado em \textbf{previsão a cada 15 minutos}, comparando \textbf{LSTM}, \textbf{CNN}, \textbf{XGBoost} e \textbf{ARIMA}, com \textbf{persistência} como linha de base. A análise deverá empregar métricas padronizadas (RMSE, MSE, R²) e validação temporal, assegurando comparação justa com a literatura.

Por fim, com base nas tendências observadas, a subseção seguinte apresenta os trabalhos considerados mais relevantes, selecionados por aplicação, horizonte e métodos, que servirão como referências diretas para o desenho experimental desta dissertação.

\section{Trabalhos mais relevantes da bibliografia para esse estudo}

Estudos pioneiros mostraram o potencial de redes neurais na previsão de irradiância solar. \citeonline{mellit2010ann} aplicaram um MLP para previsão diária em Trieste, Itália, obtendo desempenho superior aos métodos convencionais e demonstrando que mesmo arquiteturas simples já capturavam com eficácia a variabilidade da irradiância. A revisão abrangente de \citeonline{voyant2017review} consolidou esse panorama, destacando o avanço de técnicas como SVR, árvores de regressão e ensembles, embora ressaltando a dificuldade de comparação direta entre estudos devido às diferentes bases de dados e métricas. A revisão indica que abordagens híbridas costumam superar modelos isolados, tendência reforçada com a popularização de métodos mais robustos.

Nos últimos anos, modelos LSTM tornaram-se referência por sua capacidade de modelar dependências temporais. \citeonline{qing2018lstm} propuseram uma LSTM para previsão diária usando variáveis meteorológicas, alcançando desempenho substancialmente superior a persistência, regressões e MLPs (ganho de até 42,9\% em RMSE). Trabalhos recentes expandiram esse uso por meio de pré-processamentos e otimização avançada. \citeonline{mohanasundaram2025rstl} combinaram decomposição RSTL com LSTM otimizada por ASOA, obtendo reduções significativas em RMSE/MAE e maior robustez. De forma semelhante, \citeonline{gyeltshen2025rnn} desenvolveram um modelo híbrido ARIMA–LSTM–atenção para regiões montanhosas, resultando em erros muito baixos (RMSE entre 5–8,45 W/m²), beneficiando-se de mecanismos de atenção, regularização e otimização Bayesiana.

Além dos métodos de previsão, a escolha das fontes de dados também tem sido central. \citeonline{urraca2018era5} avaliaram reanálises ERA5 e COSMO-REA6, identificando grande redução de viés em relação a versões anteriores e desempenho comparável a produtos satelitais em várias regiões, embora ainda limitados sob forte nebulosidade e em áreas com gradientes espaciais acentuados. Esses achados reforçam o uso combinado de reanálises, satélite e medições locais como estratégia robusta para alimentar modelos de ML. O desafio da escassez de dados locais motivou abordagens baseadas em similaridade: \citeonline{zambrano2020transfer} demonstraram que selecionar sítios climática e geograficamente semelhantes melhora previsões em locais sem histórico próprio, indicando oportunidades de integração com modelos LSTM via pré-treinamento e adaptação.

Ferramentas físicas também permanecem relevantes. \citeonline{yan2023pvlib} demonstraram o uso da biblioteca \texttt{pvlib} com o modelo de céu claro de Ineichen, evidenciando sua precisão sob céu limpo e utilidade como baseline para modelos híbridos físico–estatísticos, especialmente quando combinados com previsões NWP e ajustes via ML. Essa complementaridade entre modelos físicos e data-driven tem sido um eixo central da evolução recente.

Em síntese, a literatura apresenta um avanço consistente: de modelos estáticos e redes MLP para arquiteturas profundas, híbridas e apoiadas por múltiplas fontes de dados. As LSTM destacam-se como solução robusta para horizontes intra-diários, combinando capacidade temporal, flexibilidade para incorporar variáveis meteorológicas e desempenho superior em diversos estudos. Embora modelos híbridos possam oferecer ganhos adicionais, sua complexidade aumenta substancialmente. Para os objetivos desta dissertação — previsão de irradiância em resolução de 15 minutos — a LSTM isolada constitui base sólida e amplamente validada, com possibilidade de extensão futura mediante integração com dados de reanálise, modelos físicos ou estratégias híbridas de baixa complexidade.
