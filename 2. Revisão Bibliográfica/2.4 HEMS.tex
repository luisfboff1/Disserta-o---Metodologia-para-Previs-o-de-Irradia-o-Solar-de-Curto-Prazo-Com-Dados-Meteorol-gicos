\section{HEMS}

Os \textit{Home Energy Management Systems} (HEMS) consolidaram-se como elementos centrais da transição energética residencial, impulsionados pelo avanço das redes inteligentes e pela expansão massiva da geração fotovoltaica distribuída (\cite{Shafiekhah2019}~). Desde o trabalho seminal de \citeonline{Moen1979}, que introduziu o monitoramento e controle do consumo doméstico, a literatura passou a tratar o HEMS como o mecanismo que transforma o usuário em \textit{prosumer}, coordenando cargas, baterias e geração solar para reduzir custos, suavizar picos de demanda e aumentar o autoconsumo de energia limpa (\cite{Elkazaz2020}~). Estudos recentes reforçam que o HEMS está diretamente condicionado à intermitência da geração fotovoltaica, tornando previsões acuradas de irradiância solar insumo indispensável para decisões ótimas de carregamento, deslocamento de cargas e operação de armazenamento (\cite{Sun2016}~). Nesse sentido, a integração entre previsão de geração e controle preditivo tem se tornado a norma em arquiteturas modernas de HEMS.

No contexto desta dissertação, o papel do HEMS é particularmente relevante, pois sua eficiência depende diretamente da qualidade das previsões de irradiância em horizontes de minutos a horas, responsáveis por antecipar variações rápidas associadas à nebulosidade. Trabalhos como \citeonline{Bot2021} e \citeonline{Antonanzas2016} demonstram que erros de previsão impactam a economia diária, a utilização da bateria e o aproveitamento da energia fotovoltaica disponível, enquanto previsões de curto prazo aumentam significativamente a capacidade do HEMS de operar de modo proativo. Assim, ao aprimorar a previsão de irradiância solar, contribui-se não apenas para o desempenho individual do HEMS, mas também para objetivos sistêmicos da modernização do setor elétrico, como confiabilidade, redução de emissões e maior integração da geração distribuída (\cite{Shafiekhah2019}~).
