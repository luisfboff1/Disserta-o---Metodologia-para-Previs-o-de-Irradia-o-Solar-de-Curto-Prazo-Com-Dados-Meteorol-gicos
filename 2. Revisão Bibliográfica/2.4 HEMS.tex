\section{HEMS}

O conceito de \textit{Home Energy Management System} (HEMS) surgiu no final da década de 1970, motivado pela necessidade de otimizar o uso doméstico de energia elétrica em face do aumento da demanda e das preocupações ambientais. O primeiro sistema do gênero foi proposto por \citeonline{Moen1979}. Esse trabalho seminal introduziu a ideia de monitorar e controlar o consumo residencial de eletricidade, especialmente em casas com geração solar fotovoltaica, de forma a adequar a demanda do usuário ao suprimento disponível. Nos anos seguintes, avanços tecnológicos permitiram a evolução do HEMS: a introdução de microcomputadores na década de 1980 viabilizou sistemas mais eficientes de controle em tempo real, e já em 1982 foram aplicados algoritmos de otimização para gerenciar cargas domésticas com o objetivo de reduzir custos elétricos por meio do deslocamento de consumo para fora dos horários de pico (\cite{Shafiekhah2019}~). A partir dos anos 2000, com a expansão das redes inteligentes (\textit{smart grids}) e a popularização da geração distribuída, o interesse por HEMS cresceu significativamente (\cite{Shafiekhah2019}~). Diferentes estratégias de gestão e controle foram propostas na literatura, incorporando desde técnicas clássicas de controle até algoritmos inteligentes e métodos de otimização avançados, consolidando o HEMS como um elemento fundamental em casas conectadas à rede elétrica inteligente.

De forma geral, um HEMS pode ser definido como um sistema integrado de hardware e software que realiza o monitoramento e gerenciamento eficiente do uso de energia em residências (\cite{Shafiekhah2019}~). Sua arquitetura básica tipicamente envolve: medidores inteligentes e sensores distribuídos pela casa para medir o consumo de eletrodomésticos; um controlador central (local ou na nuvem) capaz de tomar decisões de controle de cargas; interfaces de comunicação que conectam o sistema ao usuário (por exemplo, via aplicativo) e à concessionária de energia; e, cada vez mais comum, a integração de fontes de geração distribuída (especialmente painéis fotovoltaicos) e dispositivos de armazenamento de energia, como baterias estacionárias ou baterias de veículos elétricos. O HEMS monitora em tempo real o consumo das cargas domésticas, podendo classificar os aparelhos em cargas controláveis (que podem ter seu funcionamento deslocado ou modulado) e não controláveis (cargas cuja demanda não pode ser alterada pelo sistema). Com base em medições e em informações da rede elétrica (como preços em tempo real ou sinais de demanda), o HEMS atua sobre dispositivos selecionados (por exemplo, alterando termostatos, ligando/desligando aparelhos ou controlando a potência de carregamento de baterias) para otimizar o perfil de consumo da residência.

Os principais objetivos e vantagens associados à implementação de um HEMS abrangem diversos aspectos de eficiência e economia. Em primeiro lugar, busca-se a otimização do consumo de energia elétrica residencial, evitando desperdícios e melhorando a eficiência energética da casa. Isso inclui o deslocamento de cargas para horários de menor custo ou menor demanda na rede, contribuindo para a redução da conta de luz do consumidor (\cite{Elkazaz2020}~). Essa otimização também suaviza a curva de carga, reduzindo picos de consumo que oneram o sistema elétrico. Em segundo lugar, o HEMS possibilita uma integração mais inteligente com fontes de energia renovável distribuída, como painéis solares fotovoltaicos instalados no telhado da residência. Ao gerenciar o uso de eletrodomésticos e o carregamento/descarregamento de baterias de acordo com a disponibilidade de geração solar, o sistema pode aumentar significativamente o autoconsumo de energia limpa na casa (\cite{Elkazaz2020}~). Isso traz benefícios ambientais (menor dependência de energia proveniente de fontes fósseis) e alivia a carga na rede elétrica local, pois reduz a injeção de excedentes fotovoltaicos na rede pública. Além disso, o HEMS pode participar de programas de resposta à demanda, modulando o consumo residencial em resposta a sinais da concessionária (como preços horários ou requisições de redução de carga), contribuindo para a estabilidade do sistema elétrico maior e potencialmente gerando remuneração ao consumidor participante (\cite{Shafiekhah2019}~). Em suma, as vantagens de um HEMS incluem: maior eficiência energética, redução de custos para o usuário, melhor integração de geração renovável distribuída, diminuição de impactos ambientais e aumento da flexibilidade e inteligência na operação das redes elétricas de distribuição.

Com a difusão dos painéis solares residenciais, tornou-se evidente a forte relação entre o desempenho do HEMS e a previsão de irradiância solar. A geração fotovoltaica é inerentemente intermitente e depende das condições meteorológicas, variando ao longo do dia conforme a incidência solar e fatores como cobertura de nuvens. Nesse contexto, uma previsão precisa da potência gerada pelos painéis solares é crucial para que o HEMS possa planejar o gerenciamento de cargas e armazenamento de forma proativa (\cite{Sun2016}~). Por exemplo, conhecendo antecipadamente a estimativa de geração para as próximas horas, o sistema pode decidir programar o funcionamento de certos eletrodomésticos (como máquinas de lavar ou carregadores de veículos elétricos) para os períodos de alta geração solar, maximizando o uso direto da energia produzida localmente. Da mesma forma, o HEMS pode controlar o carregamento da bateria doméstica: se a previsão indicar excesso de geração solar, o sistema armazena energia na bateria para uso posterior; se indicar uma redução iminente na geração (por exemplo, devido à chegada de nuvens), o HEMS pode optar por poupar a energia armazenada ou até pré-carregar a bateria a partir da rede antes do aumento de tarifa, garantindo atendimento das cargas críticas durante o período de menor geração. Sem uma previsão confiável, o HEMS teria que operar de forma reativa, baseando as decisões apenas nas condições presentes, o que tende a ser subótimo e pode levar tanto a desperdício de energia renovável (por falta de uso adequado do excedente solar) quanto a maior compra de energia da rede em horários caros (\cite{Elkazaz2020}~). Por isso, muitos trabalhos recentes incorporam módulos de previsão dentro da arquitetura do HEMS, fornecendo estimativas de geração fotovoltaica (e muitas vezes também de demanda de cargas) para alimentar algoritmos de controle preditivo e otimização em tempo real (\cite{Bot2021}~).

A importância de se trabalhar com horizontes de previsão curtos, da ordem de minutos, tem sido destacada na literatura de HEMS. Diferentemente do planejamento energético em escala de redes elétricas, onde previsões de carga e geração são frequentemente feitas em bases horárias ou diárias, no ambiente residencial as tomadas de decisão do HEMS ocorrem em tempo quase real. Assim, horizontes de previsão muito longos (por exemplo, vários dias) são menos relevantes para a operação cotidiana de um HEMS, enquanto previsões de curto prazo – como 15 minutos à frente – são extremamente valiosas para a gestão dinâmica dos recursos energéticos da casa. De fato, muitos sistemas HEMS e esquemas de resposta à demanda utilizam intervalos de 15 minutos como passo de tempo para monitoramento e controle, em alinhamento com resoluções de mercados de energia e requisitos técnicos de comunicação com a rede elétrica (\cite{Bot2021}~). Por exemplo, Bot et al. (2021) implementaram em seu HEMS um modelo de previsão multi-passos de potência fotovoltaica com resolução de 15 minutos, visando um horizonte de 12 horas. Essa abordagem permitiu ao controlador preditivo do HEMS reagir a variações rápidas na geração solar, como passagens de nuvens, com antecedência suficiente para realocar cargas ou ajustar o fluxo de energia na bateria. Em geral, previsões com granularidade de minutos (5 a 15 min) inseridas no contexto de controle do HEMS aumentam a capacidade de decisão em tempo real do sistema, evitando tanto déficits inesperados de energia (que ocasionariam consumo súbito da rede em horários desfavoráveis) quanto desperdícios de geração renovável por falta de uso no momento oportuno (\cite{Antonanzas2016}~).

Diversos estudos em revistas de alto impacto têm demonstrado os benefícios práticos da integração entre HEMS e previsões precisas de irradiância solar. \citeonline{Sun2016}, por exemplo, propuseram um esquema de controle preditivo não linear para uma residência equipada com painéis fotovoltaicos e bateria, combinando modelos de previsão de geração solar e de carga elétrica. Os resultados mostraram que o HEMS preditivo conseguiu alcançar 96–98\% da economia de custo obtida no caso ideal com previsão perfeita, ao mesmo tempo em que prolongou em cerca de 25\% a vida útil da bateria ao evitar ciclos de carga/descarga desnecessários. Em outro trabalho, \citeonline{Elkazaz2020} desenvolveram um HEMS baseado em controle preditivo modelo (MPC) e avaliaram seu desempenho em um protótipo de residência inteligente em laboratório, incluindo painéis solares e um sistema de baterias. Nesse estudo, diferentes métodos de previsão foram implementados para antecipar a geração PV e a demanda da casa, e o impacto da acurácia dessas previsões sobre a operação ótima do HEMS foi analisado. Os autores relatam que a combinação de uma ferramenta de previsão acurada com um intervalo de controle adequado (15 minutos) permitiu reduzir significativamente os custos diários de energia da residência, aumentar o uso da energia solar gerada localmente (autoconsumo) e minimizar a energia desperdiçada (injetada de volta na rede) ). Esses trabalhos ilustram que o uso de técnicas de previsão aliadas a algoritmos de otimização e controle avançado (como MPC ou controle por reforço) no HEMS resulta em ganhos concretos: maior economia para o consumidor, melhor aproveitamento da energia solar disponível e manutenção do conforto dos moradores sem interrupções abruptas.

Em síntese, os Home Energy Management Systems evoluíram de um conceito experimental nas décadas passadas para se tornarem componentes centrais dos atuais sistemas energéticos inteligentes em ambiente residencial. Eles possibilitam que o consumidor residencial atue como um “prosumer”, ou seja, simultaneamente produtor e consumidor de energia, otimizando sua interação com a rede elétrica. Estudos recentes reforçam que a gestão doméstica de energia, quando apoiada em previsões de irradiância solar de curto prazo e em estratégias de controle inteligentes, pode atingir níveis elevados de eficiência e confiabilidade (\cite{Shafiekhah2019}~). O HEMS desponta, assim, como uma ferramenta indispensável para viabilizar casas energeticamente autossuficientes e sustentáveis, integrando de forma harmoniosa geração distribuída, armazenamento e consumo eficiente. Ao reduzir picos de demanda e aumentar a utilização local de fontes renováveis, esses sistemas contribuem não apenas para benefícios individuais (redução de custos e maior autonomia energética), mas também para objetivos mais amplos de política energética, como a redução de emissões de gases de efeito estufa e o aumento da resiliência e inteligência das redes de distribuição de energia elétrica.
