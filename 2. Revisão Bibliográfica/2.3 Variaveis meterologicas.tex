\section{Variáveis meteorológicas}

A previsão da irradiância solar tornou-se um tema central na transição energética, porque a geração fotovoltaica depende da radiação incidente e apresenta elevada variabilidade temporal. A literatura destaca que combinações de modelos determinísticos e técnicas estatísticas ou de aprendizado de máquina melhoram a acurácia da previsão quando incorporam variáveis meteorológicas de superfície. Estas variáveis fornecem informações físicas sobre a cobertura de nuvens, o conteúdo de vapor de água e o estado dinâmico da atmosfera, permitindo ajustar modelos empíricos e algoritmos de inteligência artificial. O presente texto sintetiza definições, formas de medição e evidências sobre o papel da temperatura do ar, umidade relativa, pressão atmosférica, precipitação, nebulosidade, índice de céu limpo, aerossóis, velocidade e direção do vento e variáveis prognósticas na previsão da irradiância solar, com ênfase em estudos realizados no Brasil.

\subsection{Temperatura do ar} 
% A temperatura do ar é a medida da energia cinética média das moléculas atmosféricas. Historicamente, utiliza-se termômetros de líquido em vidro, nos quais a dilatação térmica do mercúrio ou do álcool indica a temperatura. Termômetros eletrônicos modernos medem a resistência elétrica de termistores ou termopares e permitem registro contínuo e transmissão de dados (\cite{NIST_temperature}~). Para medições ambientais fidedignas, o instrumento deve ser protegido da radiação direta e da chuva em abrigos tipo \emph{Stevenson screen}, posicionados entre 1.2 e 2 m acima do solo e ventilados. 
Em modelos de previsão da irradiância, a temperatura máxima e mínima diária são correlacionadas com a altura do sol e a extensão dos dias; dias quentes e secos tendem a ocorrer sob anticiclones que favorecem céu claro e alta irradiância. Entretanto, temperaturas elevadas podem reduzir a eficiência das células fotovoltaicas. \citeonline{Viscondi2021} mostraram que a temperatura máxima foi uma das entradas mais relevantes em modelos de regressão para prever a irradiância global em São Paulo.

\subsection{Umidade relativa} 
Umidade é a quantidade de vapor de água presente no ar. A umidade relativa (UR) é definida como a razão entre a pressão parcial de vapor e a pressão de saturação à mesma temperatura. 
% Higrômetros de cabelo e higrômetros digitais determinam a UR por meio da variação de propriedades elétricas ou geométricas de materiais higroscópicos (\cite{RMetS_humidity}~). O psicrômetro, constituído por dois termômetros – um de bulbo seco e outro de bulbo molhado – determina a UR a partir da diferença de temperatura entre os bulbos e de tabelas psicrométricas (\cite{Psychrometer_article}~). Psicrômetros de rotação manual (\emph{sling psychrometer}) usam a ventilação gerada pela rotação para obter equilíbrio rápido. 
Em termos físicos, o vapor de água e as microgotas das nuvens absorvem e espalham radiação de onda curta, por isso, UR elevadas e neblina reduzem a irradiância direta e aumentam a difusa. Modelos estatísticos frequentemente incluem a UR como variável explicativa: estudos de previsão em localidades brasileiras encontraram correlações inversas entre UR e irradiância diurna, pois dias secos geralmente são ensolarados, enquanto UR altas indicam nebulosidade e precipitação iminente.

\subsection{Pressão atmosférica} 
A pressão atmosférica corresponde à força exercida pela coluna de ar sobre uma unidade de área e reflete a massa da atmosfera acima do local. 
% Barômetros de mercúrio ou aneroides medem essa pressão; o barômetro constitui instrumento meteorológico fundamental, pois variações rápidas sinalizam a aproximação de sistemas frontais. \citeonline{NationalGeographic_barometer} explicam que meteorologistas usam o barômetro para prever mudanças de tempo: queda acentuada da pressão associa-se a sistemas de baixa pressão, com vento forte e nuvens, enquanto pressões elevadas indicam céu limpo e tempo estável. 
Como a irradiância é sensível à cobertura de nuvens, a pressão atmosférica atua indiretamente como preditor: altas pressões persistentes são associadas a elevada irradiância global, enquanto baixas pressões implicam redução da radiação incidente devido à nebulosidade.

\subsection{Precipitação} 
A precipitação representa a quantidade de água líquida ou sólida que atinge o solo. 
% O instrumento mais usado é o pluviômetro (udometer), que mede a profundidade de chuva acumulada em milímetros durante determinado período. A Organização Meteorológica Mundial (WMO) e a NOAA padronizam o uso desses instrumentos em estações meteorológicas de superfície (\cite{WMO_rain_gauge}~). Pluviógrafos registram continuamente a altura da lâmina d’água, permitindo estudos de intensidade. 
Em previsões de irradiância, a precipitação pode funcionar como variável binária que sinaliza presença de nuvens densas ou como a quantidade de chuva acumulada; muitas abordagens removem os períodos chuvosos dos conjuntos de dados ou utilizam a chuva do dia anterior como variável exógena. Em regiões tropicais como o Brasil, chuvas convectivas ocorrem preferencialmente à tarde e reduzem drasticamente a irradiância global nesses intervalos, enquanto em dias sem chuva a radiação permanece elevada.

\subsection{Nebulosidade e índice de céu limpo} 
A nebulosidade mede a fração do céu coberta por nuvens, tradicionalmente em oitavos (oktas).
% Instrumentos modernos incluem ceilómetros que emitem feixes de laser ou infravermelho e detectam a altura da base das nuvens a partir da luz retroespalhada (\cite{Britannica_ceilometer}~); câmeras de todo o céu (\emph{total sky imagers}) capturam imagens hemisféricas e classificam a cobertura por técnicas de processamento de imagem. 
O índice de claridade (\emph{clearness index} \(K_t\)) é a razão entre a irradiância global na superfície e a irradiância extraterrestre, enquanto o índice de céu limpo (\(k_c\)) compara a irradiância medida com o valor de um modelo de céu claro. Esses índices adimensionais quantificam a transparência atmosférica; valores próximos de 1 indicam tempo limpo e valores baixos indicam forte atenuação por nuvens (\cite{UL_clearness_index}~). Muitos modelos usam o índice de céu limpo para normalizar séries temporais ou como variável dependente; por exemplo, redes neurais de previsão hora a hora utilizam imagens de câmeras para antecipar a evolução de \(k_c\), permitindo reduzir o erro de previsão em dias parcialmente nublados.

\subsection{Aerossóis} 
Aerossóis são partículas líquidas ou sólidas suspensas no ar, provenientes de poeira, queimadas, poluentes industriais e maresia. Eles reduzem a irradiância direta por absorção e espalhamento e aumentam a irradiância difusa. A profundidade óptica de aerossóis (AOD) quantifica a atenuação da radiação solar; valores muito baixos (<0,1) correspondem a atmosfera limpa, enquanto AOD de 0,4 indicam condições enevoadas.
% O \citeonline{NOAA_AOD} descreve que radiômetros como o \emph{MultiFilter Rotating Shadowband Radiometer} (MFRSR) deduzem a AOD a partir de medições globais e difusas em diferentes comprimentos de onda. 
Eventos extremos, como incêndios florestais no Brasil central, podem elevar a concentração de fumaça e reduzir em 20\% a irradiância fotovoltaica diária, segundo relatos de serviços meteorológicos; tais episódios destacam a importância de incluir variáveis de aerossóis em modelos de previsão.

\subsection{Velocidade e direção do vento} 
O vento é o movimento do ar resultante de gradientes de pressão. 
% É medido por anemômetros de conchas, hélice ou ultrassônicos, que fornecem a velocidade do vento, e por birutas ou sensores de palheta que indicam a direção. 
O vento influencia a irradiância de duas maneiras: condições ventosas são associadas à passagem de frentes e nuvens, reduzindo a radiação direta, e a velocidade do vento modula a temperatura dos módulos fotovoltaicos através da convecção. Estudos de previsão incluem a velocidade do vento para capturar a advecção de nuvens em modelos físicos; algoritmos de aprendizado de máquina observam que rajadas repentinas precedem variações rápidas na irradiância global.

\subsection{Previsão meteorológica operacional} Em escalas de curto e médio prazo, previsões de modelos numéricos de tempo (NWP) alimentam plataformas como o Climatempo e o Instituto Nacional de Meteorologia (INMET). Esses serviços disponibilizam séries horárias de temperatura, UR, pressão, ventos, nebulosidade e probabilidade de chuva em todo o território brasileiro, geradas a partir de modelos como o WRF e o ECMWF. A combinação de previsões de NWP com técnicas estatísticas (\emph{model output statistics}) ajusta vieses regionais e fornece entrada para sistemas de previsão da irradiância. No estado do Rio Grande do Sul, por exemplo, a atuação de frentes frias no outono e primavera é antecipada pelos modelos e permite prever quedas abruptas de irradiância em sistemas fotovoltaicos conectados à rede.

\subsection{Aplicações em modelos de previsão} Pesquisas recentes combinam as variáveis meteorológicas descritas em modelos de previsão de irradiância baseados em regressão, redes neurais e \emph{gradient boosting}. \citeonline{Viscondi2021} apresentaram um estudo de caso brasileiro com 19\,359 observações diárias entre 1962 e 2014 do Instituto de Astronomia, Geofísica e Ciências Atmosféricas da Universidade de São Paulo. O conjunto de dados incluía energia solar diária (MJ\,m\(^{-2}\)) e dez variáveis: temperatura máxima e mínima, velocidade do vento, umidade relativa, precipitação, pressão atmosférica e quantidades de nuvens baixas, médias e altas. Os autores compararam \emph{random forest}, suporte de vetores e redes neurais e verificaram que as temperaturas máxima e mínima e a pressão atmosférica eram os preditores mais importantes. Estudos posteriores incorporaram AOD, imagens de céu e produtos de satélite para melhorar a previsão horária em diferentes regiões do Brasil; ainda assim, variáveis de superfície como temperatura, umidade e vento permanecem essenciais devido à sua disponibilidade operacional.

\subsection{Comportamento sazonal no Brasil} O Brasil apresenta clima tropical e subtropical, com forte sazonalidade da radiação. No verão austral (dezembro–fevereiro), o sol está alto e a irradiância global atinge valores máximos, mas a convecção intensa aumenta a nebulosidade e a umidade, resultando em grande variabilidade diária. No inverno (junho–agosto), o ângulo solar é menor e a irradiância média diminui, porém massas de ar seco associadas a sistemas de alta pressão proporcionam dias claros e estabilidade atmosférica. No Rio Grande do Sul, localizado em latitudes subtropicais (29–33° S), frentes frias provenientes do Pacífico Sul cruzam o estado com frequência, causando quedas bruscas de temperatura, aumento dos ventos e cobertura de nuvens estratiformes. A literatura local mostra que os picos de irradiância horizontal ocorrem na primavera (setembro–novembro), quando o dia se alonga e a atmosfera ainda é relativamente seca, enquanto o outono apresenta boa regularidade de radiação devido à estabilidade sinótica. Esses padrões sugerem que modelos de previsão devem incorporar a estação do ano como variável categórica ou utilizar técnicas de decomposição sazonal.

 A previsão precisa da irradiância solar demanda compreensão das interações entre variáveis meteorológicas e radiação. Termômetros, higrômetros, barômetros, pluviômetros, ceilômetros, fotômetros de aerossóis e anemômetros fornecem dados essenciais sobre o estado atmosférico. Pesquisas recentes demonstram que integrar essas variáveis em modelos híbridos melhora a previsão de irradiância em diferentes horizontes temporais. No Brasil, onde as condições climáticas variam do equatorial ao subtropical, estudos de caso destacam a importância de calibrar modelos regionalmente e de considerar o comportamento sazonal. A ampliação de redes de medição e o acesso a previsões numéricas de alta resolução permitirão avanços na integração da geração solar à matriz energética nacional.
