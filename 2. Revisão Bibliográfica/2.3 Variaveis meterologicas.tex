\section{Variáveis meteorológicas}

A previsão da irradiância solar depende fortemente das condições atmosféricas de superfície, pois variáveis meteorológicas influenciam diretamente a quantidade de radiação que alcança o solo. A literatura indica que a incorporação dessas variáveis em modelos estatísticos e de aprendizado de máquina aumenta a acurácia das previsões, especialmente quando se busca capturar a cobertura de nuvens, o conteúdo de vapor d'água e a estabilidade sinótica. Nesta seção, são sintetizadas as principais variáveis utilizadas: temperatura do ar, umidade relativa, pressão atmosférica, precipitação, índice de céu limpo e previsões meteorológicas operacionais.

A temperatura do ar é amplamente empregada devido à sua associação com padrões sinóticos de céu claro. Dias quentes e secos tendem a ocorrer sob atuação de anticiclones, favorecendo elevada irradiância global. Estudos brasileiros, como o de \citeonline{Viscondi2021}, destacaram que as temperaturas máxima e mínima figuraram entre os preditores mais relevantes em modelos de regressão aplicados à irradiância diária.

A umidade relativa (UR), definida como a razão entre a pressão de vapor e a pressão de saturação à mesma temperatura, é outro indicador importante. UR elevadas, associadas à presença de microgotas e neblina, reduzem a irradiância direta por espalhamento e absorção. Em diversas regiões brasileiras, observou-se correlação inversa entre UR e irradiância diurna, refletindo a maior nebulosidade em ambientes úmidos.

A pressão atmosférica, por sua vez, atua como indicador indireto da cobertura de nuvens. Sistemas de alta pressão tendem a produzir condições estáveis e céu claro, elevando a irradiância disponível. Baixas pressões, em contraste, são associadas à aproximação de frentes e ao aumento da nebulosidade, resultando em queda da radiação incidente.

A precipitação funciona como sinalizador inequívoco de intensa cobertura de nuvens. Em modelagens de irradiância, pode ser utilizada como variável binária (presença ou ausência de chuva) ou como precipitação acumulada. Em regiões tropicais brasileiras, chuvas convectivas vespertinas reduzem abruptamente a radiação global, motivo pelo qual períodos chuvosos são frequentemente removidos ou tratados separadamente nas séries históricas.

O índice de céu limpo, definido como a razão entre a irradiância medida e a irradiância estimada por um modelo de céu claro, quantifica a transparência atmosférica e sintetiza efeitos conjuntos de nuvens, umidade, aerossóis e geometria solar. Valores próximos de 1 indicam atmosfera limpa, enquanto valores reduzidos indicam alta atenuação. Este índice é amplamente empregado para normalização de séries temporais e como variável dependente em modelos de previsão de curto prazo.

Por fim, previsões meteorológicas operacionais provenientes de modelos numéricos de tempo (NWP), como WRF e ECMWF, fornecem estimativas horárias de temperatura, umidade, vento, nebulosidade e probabilidade de chuva. Essas previsões são combinadas a técnicas estatísticas de pós-processamento para correção de vieses regionais, compondo a base de muitos sistemas de previsão de irradiância no Brasil. A antecipação de frentes frias no Sul do país, por exemplo, permite prever reduções abruptas de irradiância decorrentes do aumento de nebulosidade.

Em síntese, a integração dessas variáveis meteorológicas aprimora a capacidade preditiva de modelos de irradiância solar, especialmente em um país de elevada diversidade climática como o Brasil. A adequada representação de padrões de umidade, temperatura, pressão e nebulosidade mostra-se essencial para capturar a variabilidade diária e sazonal da radiação incidente.
