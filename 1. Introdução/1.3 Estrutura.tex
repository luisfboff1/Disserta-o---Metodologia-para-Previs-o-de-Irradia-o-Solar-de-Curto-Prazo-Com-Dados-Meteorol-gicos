\section{Estrutura da Dissertação}


A presente dissertação foi organizada em capítulos que seguem uma ordem lógica, de modo a guiar o leitor desde o contexto geral do problema até os resultados alcançados e as conclusões finais. A seguir, apresenta-se um resumo da estrutura adotada:

\begin{itemize}
    \item \textbf{Capítulo 1 – Introdução}: apresenta a contextualização do problema, destacando a importância da energia solar no cenário energético atual e os desafios relacionados à previsão de irradiância solar de curto prazo. São explicitados os objetivos da pesquisa, bem como as justificativas que fundamentam a realização do estudo.
    
    \item \textbf{Capítulo 2 – Revisão Bibliográfica}: reúne os principais trabalhos científicos relacionados ao tema, abordando o estado da arte em previsão de séries temporais aplicadas à energia solar, bem como as técnicas clássicas e modernas de inteligência artificial empregadas. Também são discutidas aplicações em sistemas de gerenciamento de energia (HEMS) e as lacunas existentes na literatura.
    
    \item \textbf{Capítulo 3 – Fundamentação Teórica}: apresenta os conceitos essenciais que sustentam a pesquisa, incluindo a caracterização da irradiância solar, os índices de claridade atmosférica, variáveis sazonais e climáticas, bem como a descrição dos algoritmos de previsão considerados (como ARIMA, LSTM, XGBoost e arquiteturas baseadas em atenção). O capítulo busca fornecer a base conceitual necessária para a compreensão da metodologia.
    
    \item \textbf{Capítulo 4 – Metodologia}: descreve o percurso metodológico da pesquisa, desde a seleção e preparação da base de dados até a construção da base final de preditores. Inclui o pré-processamento, a engenharia de atributos, a análise exploratória, a escolha e otimização dos modelos de previsão e os critérios de avaliação de desempenho. A etapa é apresentada de forma sistemática e apoiada em fluxograma que sintetiza as fases do processo.
    
    \item \textbf{Capítulo 5 – Estudo de Caso}: aplica a metodologia proposta a um conjunto de dados reais, obtidos a partir de medições meteorológicas e irradiância. São detalhadas as características do local de estudo, a organização dos dados e os cenários de modelagem considerados, de modo a ilustrar a aplicabilidade da abordagem desenvolvida.
    
    \item \textbf{Capítulo 6 – Resultados e Discussão}: apresenta os resultados obtidos a partir da aplicação dos modelos, comparando o desempenho entre diferentes algoritmos e estratégias de engenharia de atributos. São discutidos os ganhos obtidos com a utilização de variáveis derivadas, os efeitos da otimização de hiperparâmetros e a robustez das previsões em diferentes horizontes temporais.
    
    \item \textbf{Capítulo 7 – Conclusão}: sintetiza as principais contribuições do trabalho, discutindo as limitações encontradas e apontando possíveis desdobramentos futuros. Enfatiza-se a relevância dos resultados tanto no campo acadêmico quanto no contexto aplicado, especialmente em sistemas de gerenciamento energético em escala residencial e comercial.
\end{itemize}

Essa estrutura busca assegurar clareza, coerência e progressão lógica, permitindo que o leitor acompanhe a evolução do estudo desde o contexto motivador até a validação experimental e as considerações finais.
