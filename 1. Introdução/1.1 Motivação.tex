\section{Motivação}


A crescente urgência em enfrentar os desafios ambientais globais, especialmente as mudanças climáticas e a necessidade de descarbonização, tem impulsionado a adoção de fontes de energia renovável. No Brasil, a energia solar fotovoltaica vem desempenhando papel central nesse cenário. De acordo com a Associação Brasileira de Energia Solar Fotovoltaica (ABSOLAR), a fonte solar já representa cerca de 22,2\% da capacidade instalada da matriz elétrica, sendo atualmente a segunda maior fonte do país. Além disso, há mais de 3,7 milhões de sistemas de geração distribuída, abrangendo residências, comércios e pequenas indústrias (\cite{absolar2025}~).

Esse crescimento é acompanhado por impactos econômicos e sociais relevantes: desde 2012, o setor já atraiu bilhões de reais em investimentos, gerou importantes volumes de arrecadação tributária, contribuiu para a criação de empregos verdes e evitou emissões expressivas de CO\textsubscript{2} na produção de eletricidade. Projeta-se também que em 2025 a capacidade instalada solar deva crescer cerca de 25--26\%, adicionando mais de 13 GW, com forte expansão da geração distribuída (\cite{absolar2025}~).

Em paralelo a esse cenário macroeconômico, observa-se um interesse crescente em tecnologias e metodologias que viabilizam o uso eficiente da energia, especialmente em nível residencial e comercial. O conceito de \textit{Home Energy Management System} (HEMS) emerge como peça chave: sistemas que monitoram, controlam e otimizam o consumo de energia em domicílios, integrando fontes renováveis, armazenamento, cargas flexíveis e tarifação variável, para reduzir custos, minimizar desperdícios e suavizar picos de demanda. Pesquisas recentes têm investigado tanto arquiteturas de sistemas HEMS quanto algoritmos de otimização, aprendizado de máquina e controles inteligentes aplicados ao contexto doméstico (\cite{hems2019_survey}~).

A relevância acadêmica desse tema é também visível no aumento do número de estudos em previsão de séries temporais, modelagem de variáveis sazonais e climáticas, desenvolvimento de modelos híbridos (por exemplo, LSTM, Attention, GRU, redes neurais profundas) e na engenharia de atributos que capturam padrões físicos (como irradiância, índice de claridade, hora do dia, mês, sazonalidade). Esses modelos têm se mostrado eficazes na previsão de demanda e geração, o que permite que HEMS sejam mais proativos, ajustando cargas, antecipando geração solar e otimizando o uso de baterias ou dispositivos flexíveis.

Portanto, motiva-se a presente pesquisa pela convergência de três vetores: (i) o rápido crescimento do setor solar no Brasil, com sua relevância econômica, ambiental e regulatória; (ii) o potencial de sistemas HEMS para otimizar o consumo residencial e comercial, contribuindo para maior eficiência energética, economia de custo e estabilidade da rede; e (iii) a evolução das técnicas de previsão e inteligência artificial, que possibilitam melhorias substanciais no desempenho desses sistemas, desde que suportados por bases de dados de qualidade, engenharia de atributos adequada e metodologia rigorosa.

Dessa forma, este trabalho busca preencher lacunas importantes, tais como: (a) avaliar e comparar modelos de previsão modernos aplicáveis ao contexto nacional de geração solar e consumo residencial/comercial; (b) desenvolver atributos que representem de forma robusta os componentes físicos e estatísticos da irradiância e demanda; (c) estudar como a previsão pode ser integrada em HEMS para otimizar decisões em tempo real ou quase real; (d) contribuir, assim, para a eficiência do sistema elétrico, redução de custos para consumidores e apoio à transição energética sustentável no Brasil.
