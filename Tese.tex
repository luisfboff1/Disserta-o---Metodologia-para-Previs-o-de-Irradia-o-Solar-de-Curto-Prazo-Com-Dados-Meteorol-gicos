% Meta-monografia de exemplo genérico de uso da classe delaetex.cls
% Copyright (C) 2004..2019 Walter Fetter Lages <fetter@ece.ufrgs.br>
%
% This file was adapted from:
% Meta-monografia de exemplo genérico de uso da classe deletex.cls
% Copyright (C) 2004 Walter Fetter Lages <w.fetter@ieee.org>
%
% This is free software, distributed under the GNU GPL; please take
% a look in `deletex.cls' to see complete information on using, copying
% and redistributing these files
%

\documentclass[diss]{delaetex}

% um tipo específico de monografia pode ser informado como parâmetro opcional:
%\documentclass[tese]{deletex}

% O tipo de monografia pode ser:
% diss 			dissertação de mestrado
% rp 			relatório de pesquisa
% prop-tese 		proposta de tese de doutorado
% plano-doutorado 	plano curso de doutorado
% dipl-ele 		projeto de diplomação em Engenharia Elétrica
% dipl-ecp		projeto de diplomação em Engenharia de Computação
% dipl-cca		projeto de diplomação em Engenharia de Controle e Automaação
% dipl-ene		projeto de diplomação em Engenharia de Energia
% estafio		relatório de estágio supervisionado
% ti			trabalho individual
% pep			plano de estudos e pesquisa
% tese			tese de doutorado
% tc			trabalho de conclusão de mestrado profissional
% espec			monografia de conclusão de curso de especialização

% É importante notar que estes tipos de monografia foram herdados do estilo
% do II/UFRGS e não necessariamente aplicam-se ao DELET/EE/UFRGS. Ou seja,
% embora a classe deletex.cls defina uma opcao para elaborar um PEP, isto nao
% significa que um PEP seja exigido pelo PPGEE.

% monografias em inglês devem receber o parâmetro `english':
%\documentclass[diss,english]{deletex}

% a opção `openright' pode ser usada para forçar inícios de capítulos
% em páginas ímpares
% \documentclass[openright]{deletex}

% para gerar uma versão somente-frente, basta utilizar a opção `oneside':
% \documentclass[oneside]{deletex}

% A opcao numbers pode ser usada para gerar referência numéricas.
% A opcao sort&compress faz com que referencias do tipo [8,5,3,4] sejam
% convertidas para [3-4,8]
%\documentclass[numbers,sort&compress]{deletex}

% O uso dos pacotes abaixo é opcional e depende de preferências pessoais
% \usepackage[latin1]{inputenc}   % Para reconhecer o conjunto de
				% caracteres latin1 (á) na entrada (.tex) e evitar a necessidade
				% de utilizar o formato tradicional: (\'a)
\usepackage{graphicx}           % Para importar figuras
%\usepackage{mathptmx}          % Para usar fonte Adobe Times nas expressoes
\usepackage{float}		% Para posicionar as figuras de forma mais conveniente
\usepackage{url}		% Para imprimir corretamente URLs
\usepackage{enumerate}		% Para poder enumerar com letras ao inves de numeros
\usepackage{float}		% Para posicionar as figuras de forma mais conveniente
\usepackage{url}		% Para imprimir corretamente URLs
\usepackage{systeme}
\usepackage{lscape}
\usepackage{graphicx,color}
\usepackage{graphicx}
\usepackage{amsmath,amsfonts,amssymb}
\usepackage[table,xcdraw]{xcolor}
\usepackage{colortbl}
\usepackage[T1]{fontenc}
\usepackage{ragged2e}
\usepackage[table,xcdraw]{xcolor}
\usepackage[utf8]{inputenc}
\usepackage[T1]{fontenc}
\usepackage{pgf-pie}
\usepackage{placeins}
\usepackage[alf]{abntex2cite} % autor-ano
\usepackage{subfigure}
\bibliographystyle{abntex2-alf}


% Informações gerais
%
\title{Metodologia para Previsão de Irradiação Solar de Curto Prazo Com Dados Meteorológicos}

\author{Boff}{Luis Fernando}
% alguns documentos podem ter varios autores:
%\author{Flaumann}{Frida Gutenberg}
%\author{Flaumann}{Klaus Gutenberg}

% orientador
\advisor[Prof.~Dr.]{Chouhy Leborgne }{Roberto}
\advisorinfo{PPGEE - UFRGS}{PhD. for Chalmers University of Technology -- Gothenburg, Sweden}

% O comando \advisorwidth pode ser usado para ajustar o tamanho do campo
% destinado ao nome do orientador, de forma a evitar que ocupe mais de uma linha 
%\advisorwidth{0.55\textwidth}

% obviamente, o co-orientador é opcional
% \coadvisor[Prof.~Dr.]{do Co-orientador (se houver)}{Nome}
% \coadvisorinfo{UFRGS}{Doutor pela (Instituição onde obteve o título -- Cidade, País)}

% banca examinadora
\examiner[Prof.~Dr.]{do professor)}{(nome}
\examinerinfo{sigla da Instituição onde atual}{Doutor pela (Instituição onde obteve o título -- Cidade, País)}
\examiner[Prof.~Dr.]{do professor)}{(nome}
\examinerinfo{sigla da Instituição onde atual}{Doutor pela (Instituição onde obteve o título -- Cidade, País)}
\examiner[Prof.~Dr.]{do professor)}{(nome}
\examinerinfo{sigla da Instituição onde atual}{Doutor pela (Instituição onde obteve o título -- Cidade, País)}

% a data deve ser a da defesa; se nao especificada, são gerados
% mes e ano correntes
%\date{fevereiro}{2004}

% o nome do curso pode ser redefinido (ex. para Monografias)
%\course{Curso de Qualquer Coisa}

% o local de realização do trabalho pode ser especificado (ex. para Monografias)
% com o comando \location:
%\location{São José dos Campos}{SP}

% itens individuais da nominata podem ser redefinidos com os comandos
% abaixo:
% \renewcommand{\nominataReit}{Prof\textsuperscript{a}.~Dr.~Jos{\'e} Carlos Ferraz Hennemann}
% \renewcommand{\nominataReitname}{Reitor}
% \renewcommand{\nominataPRE}{Prof.~Dr.~Pedro Cezar Dutra Fonseca}
% \renewcommand{\nominataPREname}{Pr{\'o}-Reitor de Ensino}
% \renewcommand{\nominataPRAPG}{Prof\textsuperscript{a}.~Dr\textsuperscript{a}.~Valqu\'{\i}ria Linck Bassani}
% \renewcommand{\nominataPRAPGname}{Pr{\'o}-Reitora de P{\'o}s-Gradua{\c{c}}{\~a}o}
% \renewcommand{\nominataDir}{Prof.~Dr.~Renato Machado de Brito}
% \renewcommand{\nominataDirname}{Diretor da Escola de Engenharia}
% \renewcommand{\nominataCoord}{Prof.~Dr.~Carlos Eduardo Pereira}
% \renewcommand{\nominataCoordname}{Coordenador do PPGEE}
% \renewcommand{\nominataBibchefe}{June Magda Rosa Schamberg}
% \renewcommand{\nominataBibchefename}{Bibliotec{\'a}ria-chefe da Escola de Engenharia}
% \renewcommand{\nominataChefeDELET}{Prof.~Dr.~Roberto Petry Homrich}
% \renewcommand{\nominataChefeDELETname}{Chefe do \delet}

% A seguir são apresentados comandos específicos para alguns
% tipos de documentos.

% Tese de doutorado [tese] e dissertação de mestrado [diss]:
\topic{\se}	% area de concentracao, uma entre:
			% \ca Controle e Automação
			% \tic Engenharia da Computação
			% \se Energia

% Relatório de Pesquisa [rp]:
% \rp{123}             % numero do rp
% \financ{CNPq, CAPES} % orgaos financiadores

% Trabalho Individual [ti]:
% \ti{123}     % numero do TI
% \ti[II]{456} % no caso de ser o segundo TI

% Monografias de Especialização [espec]:
% \topic{Automação Industrial}      % nome do curso
% \coord[Prof.]{Bazanella}{Alexandre Sanfelice} % coordenador do curso
% \department{\delae}                                 % departamento relacionado

% Projeto de diplomação [dipl-ele] ou [dipl-ecp]:
% Pode-se definir explicitamente o nome do curso (\course):
%\course{\cgele}
%\course{\cgecp}
%\course{\cgeca}
%
% palavras-chave
% iniciar todas com letras minúsculas, exceto no caso de abreviaturas
%
\keyword{Previsão de Irradiação Solar}
\keyword{Aprendizado de Máquina}
\keyword{Sustentabilidade}
\keyword{Sistema de gerenciamento de casas inteligentes}
\keyword{Energia Fotovoltaica}
\keyword{Séries Temporais}




%
% inicio do documento
%
\begin{document}

% O comando \maketile gera a capa, a folha de rosto e a folha de aprovacao 
% (se for o caso)
% às vezes é necessário redefinir algum comando logo antes de produzir
% a Capa, folha de rosto e folha de aprovacao:
% \renewcommand{\coordname}{Coordenadora do Curso}
\maketitle

% dedicatoria é opcional

\linespread{1.241} %Este é o equivalente em LATEX de espaçamento 1.5

%\chapter*{Dedicatória}
%
%Dedico este trabalho aos meus pais, em especial pela dedicação e apoio em
%todos os momentos difíceis.
%
%% agradecimentos são opcionais
%\chapter*{Agradecimentos}
%
%Ao Programa de Pós-Graduação em Engenharia Elétrica, PPGEE, pela
%oportunidade de realização de trabalhos em minha área de pesquisa.
%
%Aos colegas do PPGEE pelo seu auxílio nas tarefas desenvolvidas durante o
%curso e apoio na revisão deste trabalho.
%
%À CAPES pela provisão da bolsa de mestrado.
%
%Agradeço ao \LaTeX\ por não ter vírus de macro\ldots

% resumo no idioma do documento
\begin{abstract} 

A previsão precisa da irradiação solar é fundamental para o avanço do gerenciamento energético em residências inteligentes com geração fotovoltaica. Este trabalho propõe uma metodologia para a previsão da irradiação solar global, com resolução de 15 minutos para o dia seguinte, baseada em técnicas de aprendizado de máquina e análise de dados. São explorados diferentes modelos e estratégias de tratamento de dados, buscando identificar a configuração ótima para diferentes contextos e volumes históricos. A proposta visa contribuir para a otimização de despacho energético, operação de baterias e integração de microredes ao sistema de distribuição[MELHORAR, BASE AINDA]. \TODO

 
\end{abstract}


% resumo no outro idioma 
% como parametro devem ser passadas as palavras-chave 
% no outro idioma, separadas por vírgulas
%\begin{englishabstract}{Electrical Engineering, Signal Processing, Automation and Control, Electronic and Instrumentation}
%\end{englishabstract}

% Conforme a NBR 6027, secao 4, o sumário deve ser o último elemento pré-textual. O
% modelo do PPGEE nao atende a esta exigencia. Obviamente, a norma deveria ter a
% precedência. No entanto, neste arquivo optou-se por reproduzir o que está
% no modelo para word.

\linespread{1} % As listas voltam a ter espaçamento 1


% sumario
\tableofcontents

% lista de ilustrações
\listoffigures

%lista de tabelas
\listoftables

% lista de abreviaturas e siglas
% o parametro deve ser a abreviatura mais longa
%\begin{listofabbrv}{PPGEE}
%	\item[ABNT] Associação Brasileira de Normas Técnicas
%	\item[GCAR] Grupo de Controle, Automação e Robótica
%	\item[PPGEE] Programa de Pós-Graduação em Engenharia Elétrica
%\end{listofabbrv}
%
%% lista de símbolos é opcional
%\begin{listofsymbols}{$\alpha\beta\pi\omega$}
%       \item[$\sum$] Somatório
%       \item[$\alpha\beta\pi\omega$] Fator de inconstância do resultado
%\end{listofsymbols}





% AQUI COMEÇA O TEXTO PROPRIAMENTE DITO

\linespread{1.241}% E o restante do texto a ter espaçamento 1.5

% ---------------------------------------
% CAPÍTULO 1 — Introdução
% ---------------------------------------
\chapter{Introdução}\label{chap:introducao}
Neste capítulo é a introdução.
% \input{cap1-introducao/01-contexto}
\section{Motivação}


A crescente urgência em enfrentar os desafios ambientais globais, especialmente as mudanças climáticas e a necessidade de descarbonização, tem impulsionado a adoção de fontes de energia renovável. No Brasil, a energia solar fotovoltaica vem desempenhando papel central nesse cenário. De acordo com a Associação Brasileira de Energia Solar Fotovoltaica (ABSOLAR), a fonte solar já representa cerca de 22,2\% da capacidade instalada da matriz elétrica, sendo atualmente a segunda maior fonte do país. Além disso, há mais de 3,7 milhões de sistemas de geração distribuída, abrangendo residências, comércios e pequenas indústrias (\cite{absolar2025}~).

Esse crescimento é acompanhado por impactos econômicos e sociais relevantes: desde 2012, o setor já atraiu bilhões de reais em investimentos, gerou importantes volumes de arrecadação tributária, contribuiu para a criação de empregos verdes e evitou emissões expressivas de CO\textsubscript{2} na produção de eletricidade. Projeta-se também que em 2025 a capacidade instalada solar deva crescer cerca de 25--26\%, adicionando mais de 13 GW, com forte expansão da geração distribuída (\cite{absolar2025}~).

Em paralelo a esse cenário macroeconômico, observa-se um interesse crescente em tecnologias e metodologias que viabilizam o uso eficiente da energia, especialmente em nível residencial e comercial. O conceito de \textit{Home Energy Management System} (HEMS) emerge como peça chave: sistemas que monitoram, controlam e otimizam o consumo de energia em domicílios, integrando fontes renováveis, armazenamento, cargas flexíveis e tarifação variável, para reduzir custos, minimizar desperdícios e suavizar picos de demanda. Pesquisas recentes têm investigado tanto arquiteturas de sistemas HEMS quanto algoritmos de otimização, aprendizado de máquina e controles inteligentes aplicados ao contexto doméstico (\cite{hems2019_survey}~).

A relevância acadêmica desse tema é também visível no aumento do número de estudos em previsão de séries temporais, modelagem de variáveis sazonais e climáticas, desenvolvimento de modelos híbridos (por exemplo, LSTM, Attention, GRU, redes neurais profundas) e na engenharia de atributos que capturam padrões físicos (como irradiância, índice de claridade, hora do dia, mês, sazonalidade). Esses modelos têm se mostrado eficazes na previsão de demanda e geração, o que permite que HEMS sejam mais proativos, ajustando cargas, antecipando geração solar e otimizando o uso de baterias ou dispositivos flexíveis.

Portanto, motiva-se a presente pesquisa pela convergência de três vetores: (i) o rápido crescimento do setor solar no Brasil, com sua relevância econômica, ambiental e regulatória; (ii) o potencial de sistemas HEMS para otimizar o consumo residencial e comercial, contribuindo para maior eficiência energética, economia de custo e estabilidade da rede; e (iii) a evolução das técnicas de previsão e inteligência artificial, que possibilitam melhorias substanciais no desempenho desses sistemas, desde que suportados por bases de dados de qualidade, engenharia de atributos adequada e metodologia rigorosa.

Dessa forma, este trabalho busca preencher lacunas importantes, tais como: (a) avaliar e comparar modelos de previsão modernos aplicáveis ao contexto nacional de geração solar e consumo residencial/comercial; (b) desenvolver atributos que representem de forma robusta os componentes físicos e estatísticos da irradiância e demanda; (c) estudar como a previsão pode ser integrada em HEMS para otimizar decisões em tempo real ou quase real; (d) contribuir, assim, para a eficiência do sistema elétrico, redução de custos para consumidores e apoio à transição energética sustentável no Brasil.



\section{Objetivos}

O objetivo principal deste trabalho é \textbf{prever a irradiância solar para o dia seguinte com resolução temporal de 15 minutos}, utilizando dados históricos de irradiância e variáveis meteorológicas associadas. A previsão em horizontes intradiários tem relevância direta para o planejamento e a operação de sistemas de energia elétrica, em especial para a integração de fontes renováveis intermitentes.

A fim de viabilizar o objetivo principal, foram estabelecidos os seguintes objetivos específicos:

\begin{itemize}
    \item Avaliar diferentes arquiteturas de modelagem, incluindo redes neurais recorrentes (LSTM e variações), modelos híbridos (ARIMA--LSTM, RSTL--LSTM, Attention) e modelos baseados em árvores (XGBoost).
    \item Analisar o impacto das variáveis de entrada, comparando desempenhos obtidos com variáveis exclusivamente meteorológicas, exclusivamente históricas da irradiância e combinações entre ambas.
    \item Investigar a influência de diferentes tamanhos de janelas de entrada (\textit{input window}) e horizontes de saída (\textit{output window}), verificando sua contribuição para a qualidade preditiva.
    \item Realizar a otimização de hiperparâmetros relevantes de cada arquitetura, tais como número de camadas, número de neurônios, taxa de aprendizado, \textit{dropout} e, no caso de modelos baseados em árvores, profundidade e número de estimadores.
    \item Comparar os modelos em termos de desempenho e generalização, empregando métricas globais e por horizonte, com destaque para RMSE e $R^2$, a fim de identificar quais arquiteturas apresentam melhor compromisso entre acurácia, robustez e interpretabilidade.
    \item Explorar a aplicabilidade prática dos modelos propostos, discutindo suas vantagens e limitações em cenários reais de previsão solar, bem como suas perspectivas de integração em sistemas de gestão e operação de energia.
\end{itemize}
}
\section{Estrutura da Dissertação}


A presente dissertação foi organizada em capítulos que seguem uma ordem lógica, de modo a guiar o leitor desde o contexto geral do problema até os resultados alcançados e as conclusões finais. A seguir, apresenta-se um resumo da estrutura adotada:

\begin{itemize}
    \item \textbf{Capítulo 1 – Introdução}: apresenta a contextualização do problema, destacando a importância da energia solar no cenário energético atual e os desafios relacionados à previsão de irradiância solar de curto prazo. São explicitados os objetivos da pesquisa, bem como as justificativas que fundamentam a realização do estudo.
    
    \item \textbf{Capítulo 2 – Revisão Bibliográfica}: reúne os principais trabalhos científicos relacionados ao tema, abordando o estado da arte em previsão de séries temporais aplicadas à energia solar, bem como as técnicas clássicas e modernas de inteligência artificial empregadas. Também são discutidas aplicações em sistemas de gerenciamento de energia (HEMS) e as lacunas existentes na literatura.
    
    \item \textbf{Capítulo 3 – Fundamentação Teórica}: apresenta os conceitos essenciais que sustentam a pesquisa, incluindo a caracterização da irradiância solar, os índices de claridade atmosférica, variáveis sazonais e climáticas, bem como a descrição dos algoritmos de previsão considerados (como ARIMA, LSTM, XGBoost e arquiteturas baseadas em atenção). O capítulo busca fornecer a base conceitual necessária para a compreensão da metodologia.
    
    \item \textbf{Capítulo 4 – Metodologia}: descreve o percurso metodológico da pesquisa, desde a seleção e preparação da base de dados até a construção da base final de preditores. Inclui o pré-processamento, a engenharia de atributos, a análise exploratória, a escolha e otimização dos modelos de previsão e os critérios de avaliação de desempenho. A etapa é apresentada de forma sistemática e apoiada em fluxograma que sintetiza as fases do processo.
    
    \item \textbf{Capítulo 5 – Estudo de Caso}: aplica a metodologia proposta a um conjunto de dados reais, obtidos a partir de medições meteorológicas e irradiância. São detalhadas as características do local de estudo, a organização dos dados e os cenários de modelagem considerados, de modo a ilustrar a aplicabilidade da abordagem desenvolvida.
    
    \item \textbf{Capítulo 6 – Resultados e Discussão}: apresenta os resultados obtidos a partir da aplicação dos modelos, comparando o desempenho entre diferentes algoritmos e estratégias de engenharia de atributos. São discutidos os ganhos obtidos com a utilização de variáveis derivadas, os efeitos da otimização de hiperparâmetros e a robustez das previsões em diferentes horizontes temporais.
    
    \item \textbf{Capítulo 7 – Conclusão}: sintetiza as principais contribuições do trabalho, discutindo as limitações encontradas e apontando possíveis desdobramentos futuros. Enfatiza-se a relevância dos resultados tanto no campo acadêmico quanto no contexto aplicado, especialmente em sistemas de gerenciamento energético em escala residencial e comercial.
\end{itemize}

Essa estrutura busca assegurar clareza, coerência e progressão lógica, permitindo que o leitor acompanhe a evolução do estudo desde o contexto motivador até a validação experimental e as considerações finais.


% \input{cap1-introducao/03-contribuicoes}
% \input{cap1-introducao/04-organizacao}

% ---------------------------------------
% CAPÍTULO 2 — Revisão Bibliográfica
% ---------------------------------------
\chapter{Revisão Bibliográfica}\label{chap:revisao}

\section{Estado da Arte}
\label{sec:estado-da-arte}

Para fundamentar este trabalho, foi conduzida uma revisão bibliográfica nas bases \textit{Scopus} e \textit{Google Scholar}, utilizando palavras-chave relacionadas à previsão de irradiação solar: \textit{solar radiation}, \textit{irradiance}, \textit{prediction}, \textit{forecast}, \textit{short-term}, \textit{meteorological data}. A busca retornou 39.208 documentos. Após filtragens (remoção de trabalhos anteriores a 1970 e itens não relacionados diretamente à irradiação/energia solar), obteve-se um conjunto consolidado de 15.317 referências únicas.

\subsection{Produção científica e distribuição geográfica}

A Figura~\ref{fig:pubs_ano} mostra a evolução anual das publicações. Observa-se crescimento moderado até meados de 2010, seguido de uma aceleração a partir de 2015, compatível com a popularização de técnicas de \textit{deep learning} e maior disponibilidade de séries meteorológicas/reanálises. Esse comportamento sugere aumento de interesse tanto em previsões de curto prazo para operação quanto em horizontes maiores para planejamento energético.

\begin{figure}[!h]
    \centering
    \caption{Publicações ao longo do tempo na área de previsão de irradiação solar.}
    \includegraphics[width=0.9\textwidth]{2. Revisão Bibliográfica/Figuras/publicações ao longo do ano.png}
    \par\small{Fonte: Autor (2025)}
    \label{fig:pubs_ano}
\end{figure}


A Figura~\ref{fig:paises} apresenta os quinze países com maior volume de publicações. Estados Unidos e China lideram, seguidos pela Índia e países europeus. O Brasil aparece em 13º lugar, indicando presença relevante da comunidade nacional, mas ainda com espaço para ampliação de esforços em previsões de curto prazo de irradiância solar.

\begin{figure}[!h]
    \centering
    \caption{Top 15 países em número de publicações.}
    \includegraphics[width=0.9\textwidth]{2. Revisão Bibliográfica/Figuras/top paises.png}
    \par\small{Fonte: Autor (2025)}
    \label{fig:paises}
\end{figure}


\subsection{Modelos de previsão utilizados e sua evolução temporal}

O mapeamento por dicionário de sinônimos revelou os modelos com maior recorrência, como mostra a Figura~\ref{fig:modelos_bar}. ANN/MLP e LSTM dominam, seguidos por SVR/SVM, Random Forest e CNN. Em seguida, algoritmos de \textit{boosting} (XGBoost/GBDT), métodos clássicos de séries (ARIMA/SARIMA) e persistência aparecem como referências frequentes.

\begin{figure}[!h]
    \centering
    \caption{Modelos mais utilizados na literatura levantada.}
    \includegraphics[width=0.9\textwidth]{2. Revisão Bibliográfica/Figuras/Modelos mais utilizados.png}
    \par\small{Fonte: Autor (2025)}
    \label{fig:modelos_bar}
\end{figure}


A Figura~\ref{fig:modelos_tempo} detalha a evolução temporal dos cinco modelos mais citados. Observa-se:
\begin{itemize}
    \item \textbf{ANN/MLP}: presença constante desde os anos 2000, com função de \textit{baseline} não linear e uso recorrente em combinações/ensembles.
    \item \textbf{LSTM}: crescimento pronunciado após 2015, associado à capacidade de modelar dependências de longo alcance e não linearidades em séries meteorológicas.
    \item \textbf{SVR/SVM}: pico intermediário e estabilidade posterior; segue competitivo em bases menores e com seleção cuidadosa de atributos.
    \item \textbf{Random Forest}: desempenho sólido em dados tabulares com variáveis meteorológicas, bom como referência robusta.
    \item \textbf{CNN}: curva ascendente recente, especialmente em \textit{nowcasting}/curtíssimo prazo ou quando convoluções 1D capturam padrões locais da série.
\end{itemize}

\begin{figure}[!h]
    \centering
    \caption{Evolução temporal (contagem anual) dos cinco modelos mais utilizados.}
    \includegraphics[width=0.9\textwidth]{2. Revisão Bibliográfica/Figuras/evolução modelos.png}
    \par\small{Fonte: Autor (2025)}
    \label{fig:modelos_tempo}
\end{figure}


\subsection{Horizontes de previsão}

A Figura~\ref{fig:horizontes} mostra os horizontes mais abordados. O horizonte de \textbf{60 minutos} concentra a maior parte dos estudos, seguido pelo \textbf{diário} e, depois, pelo \textbf{15 minutos}. O domínio de 60 min aparece, em parte, pelo uso recorrente do modelo de \textit{persistência} como \textit{baseline} e por atender tanto cenários operacionais quanto testes metodológicos padronizados. O horizonte de 15 min, embora menos frequente, apresenta tendência de crescimento e possui forte potencial de aplicação em \textit{smart grids} e em sistemas de gerenciamento de carga residencial (\textit{HEMS}). Essa granularidade permite capturar variações rápidas da irradiância ao longo do dia, o que possibilita decisões mais precisas no acionamento de cargas, no uso de armazenamento e na resposta da rede a flutuações locais de geração fotovoltaica.


\begin{figure}[!h]
    \centering
    \caption{Horizontes de previsão mais utilizados.}
    \includegraphics[width=0.9\textwidth]{2. Revisão Bibliográfica/Figuras/horizontes mais utilizados.png}
    \par\small{Fonte: Autor (2025)}
    \label{fig:horizontes}
\end{figure}


\subsection{Variáveis de entrada e métricas de avaliação}

A Figura~\ref{fig:variaveis} apresenta as variáveis mais empregadas. Nota-se \textbf{temperatura} com incidência muito superior, seguida de \textbf{umidade}, \textbf{velocidade do vento} e medidas radiométricas como \textbf{GHI}. Também aparecem \textbf{precipitação}, \textbf{pressão}, \textbf{nuvens}, \textbf{aerossóis} e insumos de reanálises/satélite (ERA5, MERRA, CAMS, etc.). Em termos de causalidade física, cobertura de nuvens, aerossóis e modelos de céu claro (\textit{clear\_sky}) têm papel direto na atenuação/variabilidade da irradiação; já variáveis termodinâmicas (temperatura, umidade) frequentemente atuam como proxies de condições de nebulosidade/estabilidade, o que explica sua ampla adoção. Esse cenário reforça a importância de seleção de atributos (\textit{feature selection}) e normalização temporal/estacional para evitar sobreajuste.

\begin{figure}[!h]
    \centering
    \caption{Variáveis mais utilizadas como preditoras.}
    \includegraphics[width=0.9\textwidth]{2. Revisão Bibliográfica/Figuras/variaveis mais utilizadas.png}
    \par\small{Fonte: Autor (2025)}
    \label{fig:variaveis}
\end{figure}


A Figura~\ref{fig:metricas} apresenta as métricas de desempenho mais recorrentes. Observa-se destaque expressivo do \textbf{RMSE}, amplamente empregado como medida de erro quadrático médio e por sua interpretação direta em termos físicos da variável prevista. Em seguida aparecem o \textbf{MSE} e o \textbf{coeficiente de determinação ($R^2$)}, frequentemente utilizados em conjunto para avaliar simultaneamente a magnitude dos erros e a proporção da variabilidade explicada pelo modelo. Outras métricas, como \textbf{MAE}, \textbf{MAPE}, \textbf{MBE} e indicadores normalizados (\textbf{nRMSE}), surgem em menor escala, geralmente para complementar a análise. A ênfase em RMSE, MSE e $R^2$ reflete a busca por \textit{benchmarks} comparáveis na literatura, permitindo avaliação justa entre diferentes abordagens.

\begin{figure}[!h]
    \centering
    \caption{Métricas de avaliação mais utilizadas.}
    \includegraphics[width=0.9\textwidth]{2. Revisão Bibliográfica/Figuras/Métricas mais utilizados.png}
    \par\small{Fonte: Autor (2025)}
    \label{fig:metricas}
\end{figure}


\subsection{Associações entre modelos, horizontes e variáveis}

As matrizes de associação de modelos por horizonte e modelos por variáveis, conforme as Figuras~\ref{fig:assoc_mh} e~\ref{fig:assoc_mv}, respectivamente, indicam padrões consistentes:
\begin{itemize}
    \item \textbf{Persistência} destacada em \textbf{60 min} e \textbf{diário}, caracterizando o uso como \textit{baseline} universal.
    \item \textbf{LSTM} com alta associação em \textbf{60 min} e \textbf{diário}, compatível com séries não lineares e dependências de médio prazo; presença relevante também quando combinações de variáveis meteorológicas são ricas.
    \item \textbf{ARIMA/SARIMA} mais forte em \textbf{60 min}, favorecido por sazonalidade clara e estruturas AR que capturam autocorrelação de curto prazo.
    \item \textbf{ANN/MLP} com bom desempenho em \textbf{60 min} e \textbf{diário}, muitas vezes como parte de \textit{ensembles}/híbridos.
    \item \textbf{CNN} associada de forma mais evidente ao \textbf{15 min}, coerente com extração de padrões locais/rápidos por convoluções 1D e uso em \textit{nowcasting}.
\end{itemize}
Quanto às variáveis, os modelos baseados em aprendizado de máquina (LSTM, ANN, CNN, XGBoost) tendem a aproveitar conjuntos multivariados (temperatura, umidade, vento, nuvens, índices de céu claro e reanálises/satélite), enquanto \textit{time-series} clássicos (ARIMA) operam melhor com transformações da própria série de irradiação (e variações sazonais), por vezes auxiliados por \textit{exogenous regressors} simples.

\begin{figure}[!h]
    \centering
    \caption{Associação (normalizada) entre modelos e horizontes.}
    \includegraphics[width=0.95\textwidth]{2. Revisão Bibliográfica/Figuras/MC modelos x horizonte.png}
    \par\small{Fonte: Autor (2025)}
    \label{fig:assoc_mh}
\end{figure}

\begin{figure}[!h]
    \centering
    \caption{Associação (normalizada) entre modelos e variáveis.}
    \includegraphics[width=0.95\textwidth]{2. Revisão Bibliográfica/Figuras/MC modelos x variaveis.png}
    \par\small{Fonte: Autor (2025)}
    \label{fig:assoc_mv}
\end{figure}

% \newpage
\FloatBarrier

\subsection{Síntese crítica e direcionamento}

Os resultados indicam três pontos práticos para projetos de previsão:
\begin{enumerate}
    \item \textbf{Horizonte de 15 min}: menos explorado que 60 min e diário, porém em crescimento — alinhado a necessidades de operação (\textit{rampas} rápidas de GHI/PV, despacho de armazenamento, \textit{demand response} e qualidade de energia).
    \item \textbf{Modelagem}: \textbf{LSTM} e \textbf{CNN} são adequados a dinâmicas rápidas e não lineares; \textbf{XGBoost} é competitivo em dados tabulares multivariados; \textbf{ARIMA} funciona como referência de série; \textbf{Persistência} permanece como \textit{baseline} mandatória para comparação justa.
    \item \textbf{Variáveis}: combinar medidas radiométricas (GHI/DNI/DHI), proxies de nebulosidade (nuvens, aerossóis, \textit{clear\_sky}) e variáveis meteorológicas (temperatura, umidade, vento) tende a melhorar robustez; o uso de reanálises/satélite (ERA5, CAMS etc.) amplia cobertura temporal e espacial quando redes de piranômetros são esparsas.
\end{enumerate}

Assim, há \textbf{espaço claro} para contribuir com um estudo focado em \textbf{previsão a cada 15 minutos}, comparando \textbf{LSTM}, \textbf{CNN}, \textbf{XGBoost} e \textbf{ARIMA}, com \textbf{persistência} como linha de base. A análise deverá empregar métricas padronizadas (RMSE, MSE, R²) e validação temporal, assegurando comparação justa com a literatura.

Por fim, com base nas tendências observadas, a subseção seguinte apresenta os trabalhos considerados mais relevantes, selecionados por aplicação, horizonte e métodos, que servirão como referências diretas para o desenho experimental desta dissertação.

\section{Trabalhos mais relevantes da bibliografia para esse estudo}



Trabalhos iniciais demonstraram o potencial de redes neurais artificiais na previsão de irradiância solar. \citeonline{mellit2010ann}, por exemplo, empregaram uma rede neural MLP (perceptron multicamada) para prever a irradiância global com 24 horas de antecedência em Trieste, Itália, visando estimar o desempenho de uma usina fotovoltaica conectada à rede. O modelo apresentou bom desempenho tanto em dias ensolarados quanto em dias nublados, superando os métodos convencionais da época. Este foi um dos primeiros estudos a aplicar ANN em previsão solar, indicando que mesmo uma arquitetura relativamente simples podia capturar a variabilidade diária da irradiância e fornecer estimativas úteis para o setor fotovoltaico.

Nos anos seguintes, ampliou-se o leque de técnicas de aprendizado de máquina (ML) aplicadas à previsão solar. Uma revisão abrangente de \citeonline{voyant2017review} destaca que diversas abordagens de ML – incluindo redes neurais tradicionais, máquinas de vetores de suporte (SVR), árvores de regressão e ensembles como random forests e boosting – foram utilizadas para prever a irradiância em diferentes contextos. Entretanto, devido à diversidade de conjuntos de dados, horizontes de previsão e métricas empregadas em cada estudo, torna-se difícil comparar diretamente o desempenho dos métodos. Em geral, muitos modelos apresentam erros de previsão equivalentes entre si, e a escolha do melhor depende das condições de cada caso. A revisão conclui que estratégias híbridas ou combinações (ensemble) tendem a melhorar a precisão preditiva, explorando os pontos fortes de cada técnica. Esse panorama abriu caminho para o uso de arquiteturas mais sofisticadas, à medida que dados mais volumosos e computação mais poderosa se tornaram disponíveis.

Nos últimos anos, redes neurais recorrentes e especialmente a arquitetura Long Short-Term Memory (LSTM) ganharam destaque por sua capacidade de modelar dependências temporais de forma eficaz. \citeonline{qing2018lstm} propuseram um modelo LSTM para previsão horária dia-a-dia (day-ahead) que utiliza dados de previsão meteorológica como entradas (temperatura, umidade, velocidade do vento, etc.). Nesse esquema, o problema é tratado como uma predição de múltiplas saídas estruturadas (as 24 horas do dia seguinte em um único pacote) para capturar as correlações entre horas consecutivas. Em testes realizados com dados reais da ilha de Santiago (Cabo Verde), o LSTM apresentou desempenho superior a modelos de referência como persistência, regressão linear e uma rede neural tradicional de múltiplas camadas. Especificamente, o algoritmo proposto mostrou-se 18,3\% mais preciso que um MLP (backpropagation) em termos de RMSE, treinando com ~2 anos de dados para prever 6 meses de valores futuros. Além disso, exibiu menor tendência ao sobreajuste e melhor capacidade de generalização: quando ampliado para usar 10 anos de histórico para prever 1 ano, o erro RMSE do LSTM foi 42,9\% menor que o de uma rede feedforward equivalente. Esses resultados evidenciam as vantagens da LSTM em capturar padrões não lineares e dinâmicos da irradiância diária, sobretudo quando há disponibilidade de variáveis meteorológicas de entrada, algo que aprimora significativamente a qualidade da previsão em comparação a redes neurais estáticas ou modelos puramente estatísticos.

Pesquisas mais recentes têm combinado LSTM com técnicas de pré-processamento de dados e otimização de hiperparâmetros para aprimorar ainda mais a acurácia. \citeonline{mohanasundaram2025rstl} introduziram um modelo que integra uma decomposição de tendência e sazonalidade (Robust Seasonal-Trend Decomposition, RSTL) aos dados de irradiância e variáveis meteorológicas, acoplado a um algoritmo bioinspirado de otimização (Adaptive Seagull Optimization, ASOA) para ajustar automaticamente os pesos e parâmetros de uma LSTM. A ideia é extrair componentes estáveis (tendência, sazonal) do sinal de irradiância antes da previsão e usar o ASOA para encontrar configurações ótimas da rede, inspirando-se no comportamento de busca de alimentos de gaivotas na natureza. Avaliado em conjuntos de dados históricos com fatores meteorológicos essenciais, o método proposto apresentou melhorias significativas nas métricas de erro (reduções em RMSE e MAE) e aumento no coeficiente de determinação ($R^2$) em relação a métodos convencionais. Isso indica que as previsões tornaram-se mais precisas e confiáveis ao empregar essa estratégia híbrida. A decomposição robusta tornou o modelo menos suscetível à variabilidade sazonal, enquanto a otimização adaptativa mitigou o overfitting e refinou o desempenho da LSTM de forma eficiente. Em síntese, o estudo demonstrou que a combinação de técnicas de pré-processing e meta-heurísticas pode potencializar a capacidade preditiva de redes LSTM, reduzindo erros e garantindo maior robustez, especialmente para previsão de energia fotovoltaica sob condições climáticas desafiadoras.

De modo análogo, \citeonline{gyeltshen2025rnn} desenvolveram um modelo híbrido que integra métodos estatísticos tradicionais e aprendizagem profunda para previsão de irradiância em terreno montanhoso (caso de estudo no Butão). Os autores combinaram um modelo ARIMA (para capturar padrões lineares de tendência) com uma rede LSTM (para padrões não lineares), adicionando ainda um mecanismo de atenção (\textit{attention mechanism}) para identificar automaticamente as sequências de entrada mais relevantes na predição. Além disso, diferentes fontes de dados de irradiância foram avaliadas: medições de superfície, estimativas de satélite e reanálises, compondo um repositório diversificado. Entre essas, o dataset NASA POWER destacou-se como o mais confiável para a região, servindo como base para o treinamento. O modelo híbrido ARIMA-LSTM-Atenção foi validado por validação cruzada e comparado isoladamente com o ARIMA e com redes neurais recorrentes puras. Os resultados evidenciaram desempenho superior do modelo híbrido na maioria das estações de medição analisadas, com erros extremamente baixos: os valores de RMSE variaram de 5 a 8,45 W/m² e o MAE de 3,7 a 7,1 W/m², enquanto o MAPE manteve-se em apenas 2–4,5\%. Tais erros são substancialmente menores que os obtidos pelos modelos não-híbridos, indicando maior precisão na previsão da irradiância. O uso do mecanismo de atenção contribuiu para identificar e dar peso às entradas mais informativas, melhorando a capacidade preditiva em meio a muita variabilidade. Adicionalmente, técnicas de regularização (L1/L2) e otimização Bayesiana de hiperparâmetros foram empregadas para calibrar o modelo em cada localidade, evitando sobreajuste e adaptando o modelo às características específicas de cada estação. Esse trabalho demonstra que a sinergia entre componentes lineares e não lineares, aliada ao uso de dados de alta qualidade e ajustes finos, pode elevar significativamente a acurácia em cenários de previsão desafiadores – como regiões de topografia complexa onde a irradiância apresenta alta variabilidade espacial e temporal.


No que tange às fontes de dados disponíveis para alimentar os modelos, a literatura evidencia que a qualidade e disponibilidade desses dados impactam diretamente o desempenho preditivo. Além de medições locais de irradiância e imagens do céu, muitos trabalhos recentes exploram dados de modelos de reanálise climática ou de previsão numérica do tempo (NWP) para aprimorar as entradas dos modelos de ML. Por exemplo, \citeonline{urraca2018era5} avaliaram criticamente as estimativas de irradiância de duas reanálises de última geração – o ERA5 (global, do ECMWF) e o COSMO-REA6 (regional, do DWD para a Europa) – comparando-as com dados observados em solo (estações BSRN) e com produtos baseados em satélite. Os autores constataram que o ERA5 representou um grande avanço em relação às reanálises predecessoras: exibiu viés médio positivo de apenas ~+4 W/m² globalmente, reduzindo em 50–75 \% o viés médio que era observado no ERA-Interim e MERRA-2. Em termos de viés, isso torna o ERA5 comparável aos dados de satélite em muitas localidades do interior (longe de oceanos). Todavia, verificou-se que a representação de nuvens nessas reanálises ainda apresenta limitações: o ERA5 tende a superestimar a irradiância em condições nubladas e a subestimar sob céu claro, indicando dificuldades na modelagem precisa da cobertura de nuvens. Consequentemente, embora seu viés médio seja baixo, o erro absoluto do ERA5 sob céu encoberto é maior do que o de métodos baseados em satélite, que capturam melhor a variabilidade instantânea da nebulosidade. Além disso, a resolução espacial relativamente grosseira do ERA5 (~31 km) mostrou-se insuficiente para regiões com variabilidade de irradiância muito alta em curtas distâncias, como áreas costeiras e montanhosas; nesses casos, o COSMO-REA6 (grade de ~6 km) apresentou desempenho superior, por conseguir resolver melhor os gradientes locais e efeitos orográficos. Em síntese, \citeonline{urraca2018era5} concluíram que ERA5 e COSMO-REA6 reduziram a lacuna de qualidade entre reanálises e dados de satélite, tornando-se alternativas viáveis quando dados satelitais ou medições locais são indisponíveis (por exemplo, em regiões polares ou períodos de falha de satélite). No entanto, ressaltam que a previsão de nuvens ainda requer melhorias nesses modelos, e que a grade espacial do ERA5 pode ser inadequada para certos propósitos, recomendando o uso de dados de satélite sempre que possível como referência principal. Para a comunidade de previsão solar, esses achados sugerem que dados de reanálise (ou previsões NWP derivadas deles) podem servir como entradas úteis para modelos de ML ou mesmo como previsores diretos em horizontes maiores, desde que se tenha em mente seus vieses e incertezas. Combinar previsões LSTM com correções baseadas em reanálise, por exemplo, pode unir a adaptabilidade do ML com a abrangência física dos modelos numéricos.

Um desafio relacionado é a falta de dados históricos locais em sítios onde se deseja prever a irradiância ou a geração fotovoltaica. Nesses casos, pesquisadores têm buscado métodos para transferir conhecimento de locais monitorados para locais não monitorados. \citeonline{zambrano2020transfer} abordaram esse problema formulando uma metodologia de aprendizado por similaridade de localidades. Em vez de assumir que há medições suficientes no local de interesse, eles propõem construir um espaço de características multidimensional usando variáveis exógenas correlacionadas à irradiância (por exemplo, coordenadas geográficas, clima médio, altitude, etc.) e definir uma métrica de distância nesse espaço para comparar sítios. Cada local é representado como um ponto no espaço de características, e para um novo local sem dados, encontram-se os sites mais similares segundo essa métrica aprendida. Em seguida, utiliza-se as séries de irradiância desses sites vizinhos para treinar um modelo que seja aplicado no local-alvo, dispensando medidas locais. Experimentos com dados reais mostraram que selecionar sítios semelhantes como fonte de dados de treinamento produziu previsões mais acuradas do que treinar modelos com dados de todos os sites disponíveis indiscriminadamente. Ou seja, houve ganho em personalização da previsão ao usar apenas dados de contextos parecidos com o do local de interesse, evitando contaminação por padrões muito distintos. Essa abordagem se alinha com técnicas de transfer learning e demonstra que, com critério de seleção adequado, é possível construir modelos preditivos para locais sem histórico, algo especialmente útil para projetar usinas fotovoltaicas em regiões remotas. Em comparação com o uso direto de uma LSTM tradicional, que normalmente exigiria um volume considerável de dados de treino do próprio local para alcançar boa performance, o método de \citeonline{zambrano2020transfer} destaca a importância de incorporar informações de similaridade climatológica/geográfica no pipeline de previsão. Assim, modelos baseados em LSTM também podem se beneficiar dessa estratégia – por exemplo, pré-treinando a LSTM em dados de locais análogos e refinando-a para o novo sítio –, combinando a capacidade de generalização do deep learning com a esperteza na seleção de dados de treino pertinentes.

Também merece menção o uso de ferramentas abertas e modelos físicos como base para previsões de irradiância, complementando os métodos puramente data-driven. \citeonline{yan2023pvlib} demonstram uma abordagem em que se emprega a biblioteca open-source pvlib (Python) para estimar componentes de irradiância e realizar previsões, evidenciando o papel de modelos físicos de céu claro integrados em plataformas de fácil acesso. Em seu estudo, utilizaram o modelo de céu claro de Ineichen implementado no pvlib para calcular irradiância de plano inclinado (POA GHI, DNI e DHI) em três locais da rede BSRN, sob diferentes condições de nebulosidade. Os resultados indicaram que o modelo fornece previsões bastante precisas em condições de céu claro, com erros aumentando conforme cresce a fração de nuvens. Em média, observaram que o erro relativo permaneceu baixo enquanto a cobertura de nuvens estivesse abaixo de ~5\%; já com nebulosidade mais densa, o erro se eleva substancialmente. Essa sensibilidade reflete a limitação esperada de modelos puramente físicos diante de variabilidade de nuvens, já que o modelo de Ineichen não incorpora dados de nuvens em tempo real, usando apenas parâmetros como turbidez. Ainda assim, o trabalho ressalta a importância da disponibilidade de ferramentas open-source: com pvlib, usuários podem escolher modelos e algoritmos adequados às suas necessidades e combinar modelos de cálculo diversos para obter resultados personalizados. Isso viabiliza, por exemplo, a integração de previsões meteorológicas (para turbidez, cobertura de nuvens prevista) com modelos de irradiância física e, em seguida, inclusão desses resultados em modelos de previsão de potência fotovoltaica. Em termos comparativos, essa abordagem oferece transparência e rapidez de implementação, aproveitando conhecimentos físicos consolidados. No entanto, carece da adaptabilidade dos métodos de aprendizado de máquina para ajustar-se a padrões complexos ou eventos inesperados. Assim, uma tendência atual é utilizar tais ferramentas em conjunto com ML – por exemplo, gerando uma previsão física inicial (baseline) e então aplicando correções via modelos estatísticos ou de machine learning, melhorando a precisão geral da previsão.

Em resumo, a literatura revela uma evolução significativa das técnicas de previsão de irradiância solar, indo de modelos estatísticos e redes neurais simples até arquiteturas profundas e sistemas híbridos complexos. Os modelos de LSTM emergem como uma das ferramentas mais eficazes, graças à sua habilidade de capturar dependências temporais de curto e longo prazo nas sequências de irradiância. Estudos comparativos mostraram que a LSTM geralmente supera redes neurais feedforward tradicionais em precisão, reduzindo substancialmente métricas de erro quando treinada com dados históricos suficientes. No contexto de previsões com passo de 15 minutos (horizonte intra diário típico), as LSTMs têm demonstrado excelente capacidade de modelar a variabilidade rápida da irradiância e antecipar flutuações dentro do dia. Em contrapartida, modelos mais simples como regressões ou MLPs, embora mais fáceis de implementar e menos exigentes computacionalmente, costumam apresentar erros maiores por não incorporarem essa dinâmica temporal de forma explícita. As abordagens híbridas recentes – que mesclam LSTM com decomposição de séries, otimizações meta-heurísticas, modelos estatísticos ou inputs adicionais representam o estado da arte em termos de redução de erro, pois atacam o problema de múltiplos ângulos. A vantagem é um ganho notável de precisão e robustez, como evidenciado pelos baixíssimos erros obtidos por modelos híbridos em diversos trabalhos. A limitação, por sua vez, é o aumento da complexidade: tais modelos podem se tornar mais difíceis de reproduzir, demandar maior poder computacional e cuidado na configuração de múltiplos componentes. Em resumo, a escolha do modelo ideal envolve considerar um equilíbrio entre desempenho e viabilidade. No caso da presente dissertação evidencia-se pelas pesquisas levantadas que a LSTM isoladamente já fornece uma base sólida, dada sua capacidade de aprender padrões temporais intra diários complexos. Vantagens como a flexibilidade de incorporar diversas variáveis de entrada (incluindo previsões meteorológicas) e a experiência positiva em diferentes estudos reforçam seu uso. Limitações potenciais, entretanto, devem ser reconhecidas: LSTMs podem exigir bastante dados para treinamento e ajustes finos de hiperparâmetros para atingir o desempenho ótimo, e seu treinamento é mais demorado em comparação a modelos rasos. Assim, soluções complementares encontradas na literatura – como enriquecer o modelo LSTM com informações adicionais ( resultados de modelos físicos ou NWP) ou adotar esquemas híbridos – podem ser incorporadas ou pelo menos consideradas como extensões para elevar ainda mais a qualidade das previsões quando necessário. Em suma, os trabalhos mais relevantes da literatura delineiam um caminho claro de aprimoramento na previsão solar: do uso pioneiro de redes neurais simples até as sofisticadas arquiteturas atuais, observa-se uma melhoria contínua na acurácia e na utilidade prática das previsões, guiada tanto pela evolução dos modelos de aprendizado quanto pela integração de novas fontes de dados e conhecimento ao processo preditivo.


\chapter{Fundamentação teórica}\label{chap:fundamentação}

\section{Irradiância Solar: definições, distinções e modelagens}

\subsection{Conceitos de radiação, irradiância e irradiação}

A energia proveniente do Sol viaja sob a forma de radiação eletromagnética, constituída por fótons com diferentes frequências. Quando se estuda o recurso solar, empregam-se três termos relacionados, mas com significados distintos:

\textbf{Irradiação} (\emph{solar irradiation} ou \emph{insolation}) refere-se à quantidade de energia solar recebida por unidade de área ao longo de um intervalo de tempo. A unidade usual é MJ/m$^{2}$ ou kWh/m$^{2}$. A irradiação é um valor acumulado – por exemplo, a energia solar incidente em uma superfície horizontal durante um dia ou mês. (\cite{solargis_components}~)

\textbf{Irradiância} (\emph{solar irradiance}) é a potência instantânea recebida por unidade de área. Representa a densidade de fluxo de energia (W/m$^{2}$) em um determinado instante (\cite{solargis_components}~). Como a irradiância mede energia por tempo (potência), ela pode ser convertida em irradiação pela integração no intervalo desejado (por exemplo, integrar as leituras de irradiância para cada período 15 minutos).

\textbf{Radiação solar} pode referir-se genericamente à energia eletromagnética emitida pelo Sol. Em contextos meteorológicos, o termo \emph{radiação solar global} (ou radiação global) costuma ser sinônimo de irradiância global horizontal (GHI).

Outra distinção importante é entre a energia incidente \emph{extraterrestre} e a energia terrestre. A irradiância extraterrestre considera a potência recebida fora da atmosfera, sem atenuação por gases ou aerossóis. A irradiância terrestre mede a potência que chega à superfície, sendo atenuada e distribuída em componentes.

\subsection{Componentes da irradiância terrestre}

A irradiância ou irradiação, que atinge uma superfície horizontal ou inclinada, é composta por pela parcela direta, difusa e global, conforme ilustra a Figura~\ref{fig:irradiação}.

\begin{figure}[!h]
    \centering
    \caption{Tipos de irradiações.}
    \includegraphics[width=0.7\textwidth]{2. Revisão Bibliográfica/Figuras/IRRADIAÇÕES.png}
    \par\small{Fonte: Tiepolo et al. (2017)}
    \label{fig:irradiação}
\end{figure}

% \newpage


\begin{itemize}
    \item \textbf{Irradiância direta}: corresponde à porção da luz solar que chega à superfície sem sofrer espalhamento atmosférico. Esse componente é maior em dias claros e é medido em uma superfície normal ao Sol (\cite{solargis_components}~).
    \item \textbf{Irradiância difusa}: resulta do espalhamento de partículas e moléculas na atmosfera. Em dias nublados ou com elevada turbidez, grande parte da energia chega à superfície como difusa (\cite{solargis_components}~).
    \item \textbf{Irradiância global}: soma dos componentes direto e difuso numa superfície. Para uma superfície horizontal, chama-se GHI (\emph{Global Horizontal Irradiance}), enquanto que para uma superfície normal ao Sol chama-se DNI (\emph{Direct Normal Irradiance}) (\cite{pvpmc_insolation}~).
\end{itemize}

Os instrumentos usados para medir esses componentes também diferem: piranômetros convencionais registram o fluxo hemisférico (180° de campo de visão) e, portanto, medem GHI; pirheliógrafos (ou piranômetros com anteparos) têm campo de visão de aproximadamente 5° para medir apenas o feixe direto (\cite{pvpmc_insolation}~).

Além do plano de medição, é necessário especificar a orientação da superfície de coleta. A irradiância pode ser medida em uma superfície horizontal (GHI), normal ao Sol (DNI) ou em um plano inclinado (\emph{plane-of-array}) (\cite{pvpmc_insolation}~). Para converter irradiância em energia (irradiação) em uma superfície inclinada, a posição do Sol (elevação, azimute) e o ângulo de inclinação da superfície precisam ser considerados.

\FloatBarrier

\subsection{Irradiância extraterrestre e modelagem matemática}

A energia recebida na parte superior da atmosfera varia com a distância Terra–Sol e com a posição da Terra na órbita. A FAO apresenta a fórmula para a irradiação extraterrestre diária sobre uma superfície horizontal:

\begin{equation}
    R_{a} = \frac{24 \times 60}{\pi} \, G_{sc} \, d_{r} \Big[ \omega_{s} \sin\phi \sin\delta + \cos\phi\cos\delta \sin\omega_{s} \Big]
    \label{eq:ra}
\end{equation}
\noindent
Em que $G_{sc}=0{,}0820$ MJ/m$^{2}\,\text{min}$ é a constante solar; $d_{r}$ representa a distância relativa Terra–Sol em função do dia Juliano; $\delta$ é a declinação solar (rad); $\omega_{s}$ corresponde ao ângulo horário ao nascer e ao pôr do Sol (rad); e $\phi$ é a latitude (rad).


A fórmula demonstra que o valor de $R_{a}$ depende fortemente da latitude e da época do ano. Em latitudes baixas (trópicos) a amplitude anual é menor, resultando em valores de irradiância extraterrestre mais elevados e pouco variáveis. Para latitudes mais altas, as variações sazonais são maiores. Em outras palavras, a latitude determina o ângulo solar máximo e o comprimento do dia; no hemisfério Sul (latitudes negativas) as estações são defasadas em relação ao hemisfério Norte. A longitude, por sua vez, influencia o horário local (fuso horário), mas não altera o valor da irradiância instantânea (apenas determina em que momento do dia ocorre o pico).


\subsection{Brasil: recurso solar e variação regional}

Devido à sua localização majoritariamente entre as latitudes 5°N e 33°S, o Brasil possui elevado potencial de energia solar. Regiões próximas ao equador, como Norte e Nordeste, apresentam altos valores de irradiância global média anual (superior a 5,0 kWh/m$^{2}.dia$). Já na região Sul (latitudes >25°S), a sazonalidade é mais pronunciada e a média anual é menor ($\sim$4,0–4,5 kWh/m$^{2}.dia$), porém ainda competitiva em relação a países europeus.

Por exemplo, Fortaleza (3°S) apresenta pouca variação sazonal (diferença entre verão e inverno $<20\%$), enquanto Caxias do Sul (29°S) observa variações bem mais acentuadas devido à maior inclinação do eixo terrestre.


A distinção entre irradiância (potência instantânea) e irradiação (energia acumulada) é essencial para comparar estudos e dimensionar sistemas solares. A irradiância extraterrestre fornece um limite teórico de potência, cuja variação diária e sazonal depende da latitude. Para estimar a irradiância real na superfície, consideram-se as componentes direto e difuso e as perdas por absorção/espalhamento na atmosfera. No Brasil, as elevadas irradiâncias, especialmente nas regiões tropicais, combinadas com uma variabilidade sazonal relativamente baixa, conferem ótimo potencial para geração fotovoltaica. Modelos e métricas corretas (RMSE, MAE, nRMSE, etc.) são fundamentais para avaliar previsões de irradiância de curto prazo e integrar a geração solar à operação de redes elétricas.



\section{Variáveis meteorológicas}

A previsão da irradiância solar tornou-se um tema central na transição energética, porque a geração fotovoltaica depende da radiação incidente e apresenta elevada variabilidade temporal. A literatura destaca que combinações de modelos determinísticos e técnicas estatísticas ou de aprendizado de máquina melhoram a acurácia da previsão quando incorporam variáveis meteorológicas de superfície. Estas variáveis fornecem informações físicas sobre a cobertura de nuvens, o conteúdo de vapor de água e o estado dinâmico da atmosfera, permitindo ajustar modelos empíricos e algoritmos de inteligência artificial. O presente texto sintetiza definições, formas de medição e evidências sobre o papel da temperatura do ar, umidade relativa, pressão atmosférica, precipitação, nebulosidade, índice de céu limpo, aerossóis, velocidade e direção do vento e variáveis prognósticas na previsão da irradiância solar, com ênfase em estudos realizados no Brasil.

\subsection{Temperatura do ar} 
% A temperatura do ar é a medida da energia cinética média das moléculas atmosféricas. Historicamente, utiliza-se termômetros de líquido em vidro, nos quais a dilatação térmica do mercúrio ou do álcool indica a temperatura. Termômetros eletrônicos modernos medem a resistência elétrica de termistores ou termopares e permitem registro contínuo e transmissão de dados (\cite{NIST_temperature}~). Para medições ambientais fidedignas, o instrumento deve ser protegido da radiação direta e da chuva em abrigos tipo \emph{Stevenson screen}, posicionados entre 1.2 e 2 m acima do solo e ventilados. 
Em modelos de previsão da irradiância, a temperatura máxima e mínima diária são correlacionadas com a altura do sol e a extensão dos dias; dias quentes e secos tendem a ocorrer sob anticiclones que favorecem céu claro e alta irradiância. Entretanto, temperaturas elevadas podem reduzir a eficiência das células fotovoltaicas. \citeonline{Viscondi2021} mostraram que a temperatura máxima foi uma das entradas mais relevantes em modelos de regressão para prever a irradiância global em São Paulo.

\subsection{Umidade relativa} 
Umidade é a quantidade de vapor de água presente no ar. A umidade relativa (UR) é definida como a razão entre a pressão parcial de vapor e a pressão de saturação à mesma temperatura. 
% Higrômetros de cabelo e higrômetros digitais determinam a UR por meio da variação de propriedades elétricas ou geométricas de materiais higroscópicos (\cite{RMetS_humidity}~). O psicrômetro, constituído por dois termômetros – um de bulbo seco e outro de bulbo molhado – determina a UR a partir da diferença de temperatura entre os bulbos e de tabelas psicrométricas (\cite{Psychrometer_article}~). Psicrômetros de rotação manual (\emph{sling psychrometer}) usam a ventilação gerada pela rotação para obter equilíbrio rápido. 
Em termos físicos, o vapor de água e as microgotas das nuvens absorvem e espalham radiação de onda curta, por isso, UR elevadas e neblina reduzem a irradiância direta e aumentam a difusa. Modelos estatísticos frequentemente incluem a UR como variável explicativa: estudos de previsão em localidades brasileiras encontraram correlações inversas entre UR e irradiância diurna, pois dias secos geralmente são ensolarados, enquanto UR altas indicam nebulosidade e precipitação iminente.

\subsection{Pressão atmosférica} 
A pressão atmosférica corresponde à força exercida pela coluna de ar sobre uma unidade de área e reflete a massa da atmosfera acima do local. 
% Barômetros de mercúrio ou aneroides medem essa pressão; o barômetro constitui instrumento meteorológico fundamental, pois variações rápidas sinalizam a aproximação de sistemas frontais. \citeonline{NationalGeographic_barometer} explicam que meteorologistas usam o barômetro para prever mudanças de tempo: queda acentuada da pressão associa-se a sistemas de baixa pressão, com vento forte e nuvens, enquanto pressões elevadas indicam céu limpo e tempo estável. 
Como a irradiância é sensível à cobertura de nuvens, a pressão atmosférica atua indiretamente como preditor: altas pressões persistentes são associadas a elevada irradiância global, enquanto baixas pressões implicam redução da radiação incidente devido à nebulosidade.

\subsection{Precipitação} 
A precipitação representa a quantidade de água líquida ou sólida que atinge o solo. 
% O instrumento mais usado é o pluviômetro (udometer), que mede a profundidade de chuva acumulada em milímetros durante determinado período. A Organização Meteorológica Mundial (WMO) e a NOAA padronizam o uso desses instrumentos em estações meteorológicas de superfície (\cite{WMO_rain_gauge}~). Pluviógrafos registram continuamente a altura da lâmina d’água, permitindo estudos de intensidade. 
Em previsões de irradiância, a precipitação pode funcionar como variável binária que sinaliza presença de nuvens densas ou como a quantidade de chuva acumulada; muitas abordagens removem os períodos chuvosos dos conjuntos de dados ou utilizam a chuva do dia anterior como variável exógena. Em regiões tropicais como o Brasil, chuvas convectivas ocorrem preferencialmente à tarde e reduzem drasticamente a irradiância global nesses intervalos, enquanto em dias sem chuva a radiação permanece elevada.

\subsection{Nebulosidade e índice de céu limpo} 
A nebulosidade mede a fração do céu coberta por nuvens, tradicionalmente em oitavos (oktas).
% Instrumentos modernos incluem ceilómetros que emitem feixes de laser ou infravermelho e detectam a altura da base das nuvens a partir da luz retroespalhada (\cite{Britannica_ceilometer}~); câmeras de todo o céu (\emph{total sky imagers}) capturam imagens hemisféricas e classificam a cobertura por técnicas de processamento de imagem. 
O índice de claridade (\emph{clearness index} \(K_t\)) é a razão entre a irradiância global na superfície e a irradiância extraterrestre, enquanto o índice de céu limpo (\(k_c\)) compara a irradiância medida com o valor de um modelo de céu claro. Esses índices adimensionais quantificam a transparência atmosférica; valores próximos de 1 indicam tempo limpo e valores baixos indicam forte atenuação por nuvens (\cite{UL_clearness_index}~). Muitos modelos usam o índice de céu limpo para normalizar séries temporais ou como variável dependente; por exemplo, redes neurais de previsão hora a hora utilizam imagens de câmeras para antecipar a evolução de \(k_c\), permitindo reduzir o erro de previsão em dias parcialmente nublados.

\subsection{Aerossóis} 
Aerossóis são partículas líquidas ou sólidas suspensas no ar, provenientes de poeira, queimadas, poluentes industriais e maresia. Eles reduzem a irradiância direta por absorção e espalhamento e aumentam a irradiância difusa. A profundidade óptica de aerossóis (AOD) quantifica a atenuação da radiação solar; valores muito baixos (<0,1) correspondem a atmosfera limpa, enquanto AOD de 0,4 indicam condições enevoadas.
% O \citeonline{NOAA_AOD} descreve que radiômetros como o \emph{MultiFilter Rotating Shadowband Radiometer} (MFRSR) deduzem a AOD a partir de medições globais e difusas em diferentes comprimentos de onda. 
Eventos extremos, como incêndios florestais no Brasil central, podem elevar a concentração de fumaça e reduzir em 20\% a irradiância fotovoltaica diária, segundo relatos de serviços meteorológicos; tais episódios destacam a importância de incluir variáveis de aerossóis em modelos de previsão.

\subsection{Velocidade e direção do vento} 
O vento é o movimento do ar resultante de gradientes de pressão. 
% É medido por anemômetros de conchas, hélice ou ultrassônicos, que fornecem a velocidade do vento, e por birutas ou sensores de palheta que indicam a direção. 
O vento influencia a irradiância de duas maneiras: condições ventosas são associadas à passagem de frentes e nuvens, reduzindo a radiação direta, e a velocidade do vento modula a temperatura dos módulos fotovoltaicos através da convecção. Estudos de previsão incluem a velocidade do vento para capturar a advecção de nuvens em modelos físicos; algoritmos de aprendizado de máquina observam que rajadas repentinas precedem variações rápidas na irradiância global.

\subsection{Previsão meteorológica operacional} Em escalas de curto e médio prazo, previsões de modelos numéricos de tempo (NWP) alimentam plataformas como o Climatempo e o Instituto Nacional de Meteorologia (INMET). Esses serviços disponibilizam séries horárias de temperatura, UR, pressão, ventos, nebulosidade e probabilidade de chuva em todo o território brasileiro, geradas a partir de modelos como o WRF e o ECMWF. A combinação de previsões de NWP com técnicas estatísticas (\emph{model output statistics}) ajusta vieses regionais e fornece entrada para sistemas de previsão da irradiância. No estado do Rio Grande do Sul, por exemplo, a atuação de frentes frias no outono e primavera é antecipada pelos modelos e permite prever quedas abruptas de irradiância em sistemas fotovoltaicos conectados à rede.

\subsection{Aplicações em modelos de previsão} Pesquisas recentes combinam as variáveis meteorológicas descritas em modelos de previsão de irradiância baseados em regressão, redes neurais e \emph{gradient boosting}. \citeonline{Viscondi2021} apresentaram um estudo de caso brasileiro com 19\,359 observações diárias entre 1962 e 2014 do Instituto de Astronomia, Geofísica e Ciências Atmosféricas da Universidade de São Paulo. O conjunto de dados incluía energia solar diária (MJ\,m\(^{-2}\)) e dez variáveis: temperatura máxima e mínima, velocidade do vento, umidade relativa, precipitação, pressão atmosférica e quantidades de nuvens baixas, médias e altas. Os autores compararam \emph{random forest}, suporte de vetores e redes neurais e verificaram que as temperaturas máxima e mínima e a pressão atmosférica eram os preditores mais importantes. Estudos posteriores incorporaram AOD, imagens de céu e produtos de satélite para melhorar a previsão horária em diferentes regiões do Brasil; ainda assim, variáveis de superfície como temperatura, umidade e vento permanecem essenciais devido à sua disponibilidade operacional.

\subsection{Comportamento sazonal no Brasil} O Brasil apresenta clima tropical e subtropical, com forte sazonalidade da radiação. No verão austral (dezembro–fevereiro), o sol está alto e a irradiância global atinge valores máximos, mas a convecção intensa aumenta a nebulosidade e a umidade, resultando em grande variabilidade diária. No inverno (junho–agosto), o ângulo solar é menor e a irradiância média diminui, porém massas de ar seco associadas a sistemas de alta pressão proporcionam dias claros e estabilidade atmosférica. No Rio Grande do Sul, localizado em latitudes subtropicais (29–33° S), frentes frias provenientes do Pacífico Sul cruzam o estado com frequência, causando quedas bruscas de temperatura, aumento dos ventos e cobertura de nuvens estratiformes. A literatura local mostra que os picos de irradiância horizontal ocorrem na primavera (setembro–novembro), quando o dia se alonga e a atmosfera ainda é relativamente seca, enquanto o outono apresenta boa regularidade de radiação devido à estabilidade sinótica. Esses padrões sugerem que modelos de previsão devem incorporar a estação do ano como variável categórica ou utilizar técnicas de decomposição sazonal.

 A previsão precisa da irradiância solar demanda compreensão das interações entre variáveis meteorológicas e radiação. Termômetros, higrômetros, barômetros, pluviômetros, ceilômetros, fotômetros de aerossóis e anemômetros fornecem dados essenciais sobre o estado atmosférico. Pesquisas recentes demonstram que integrar essas variáveis em modelos híbridos melhora a previsão de irradiância em diferentes horizontes temporais. No Brasil, onde as condições climáticas variam do equatorial ao subtropical, estudos de caso destacam a importância de calibrar modelos regionalmente e de considerar o comportamento sazonal. A ampliação de redes de medição e o acesso a previsões numéricas de alta resolução permitirão avanços na integração da geração solar à matriz energética nacional.

\section{HEMS}

Os \textit{Home Energy Management Systems} (HEMS) consolidaram-se como elementos centrais da transição energética residencial, impulsionados pelo avanço das redes inteligentes e pela expansão massiva da geração fotovoltaica distribuída (\cite{Shafiekhah2019}~). Desde o trabalho seminal de \citeonline{Moen1979}, que introduziu o monitoramento e controle do consumo doméstico, a literatura passou a tratar o HEMS como o mecanismo que transforma o usuário em \textit{prosumer}, coordenando cargas, baterias e geração solar para reduzir custos, suavizar picos de demanda e aumentar o autoconsumo de energia limpa (\cite{Elkazaz2020}~). Estudos recentes reforçam que o HEMS está diretamente condicionado à intermitência da geração fotovoltaica, tornando previsões acuradas de irradiância solar insumo indispensável para decisões ótimas de carregamento, deslocamento de cargas e operação de armazenamento (\cite{Sun2016}~). Nesse sentido, a integração entre previsão de geração e controle preditivo tem se tornado a norma em arquiteturas modernas de HEMS.

No contexto desta dissertação, o papel do HEMS é particularmente relevante, pois sua eficiência depende diretamente da qualidade das previsões de irradiância em horizontes de minutos a horas, responsáveis por antecipar variações rápidas associadas à nebulosidade. Trabalhos como \citeonline{Bot2021} e \citeonline{Antonanzas2016} demonstram que erros de previsão impactam a economia diária, a utilização da bateria e o aproveitamento da energia fotovoltaica disponível, enquanto previsões de curto prazo aumentam significativamente a capacidade do HEMS de operar de modo proativo. Assim, ao aprimorar a previsão de irradiância solar, contribui-se não apenas para o desempenho individual do HEMS, mas também para objetivos sistêmicos da modernização do setor elétrico, como confiabilidade, redução de emissões e maior integração da geração distribuída (\cite{Shafiekhah2019}~).

\section{Modelos de previsão temporal}

A previsão de séries temporais de irradiância solar é uma tarefa desafiadora e de grande relevância para a integração de sistemas fotovoltaicos na rede elétrica.  Modelos de previsão temporal variam desde abordagens estatísticas clássicas até técnicas modernas de aprendizado de máquina baseadas em redes neurais e ensembles de árvores de decisão.  Neste texto são revistos os principais modelos utilizados para previsão de séries temporais, com ênfase em Long Short–Term Memory (LSTM), redes neurais convolucionais (CNN), modelos autorregressivos integrados de média móvel (ARIMA), decomposição sazonal e tendência por LOESS (STL/RSTL), XGBoost e o modelo de persistência.  Cada subseção descreve o contexto histórico, o funcionamento matemático, a categoria do modelo, a adaptação para séries temporais e aplicações na previsão de irradiância solar.


\subsection{LSTM – Long Short–Term Memory}

As redes LSTM foram introduzidas por \citeonline{hochreiter1997} para mitigar o problema do gradiente que se dissipa em RNNs convencionais. A Figura~\ref{fig:lstm-cell} ilustra a célula LSTM adotada neste trabalho e  \eqref{eq:lstm} correspondem exatamente às operações mostradas na figura (portas de entrada, esquecimento e saída, atualização do estado de célula e estado oculto). As portas usam, por padrão, ativação sigmoide e o candidato de memória utiliza a função $\tanh$ (ambas passíveis de substituição). 

\begin{equation}
\begin{aligned}
\mathbf{i}_t &= \sigma\!\left(\mathbf{W}_{xi}\mathbf{x}_t + \mathbf{W}_{hi}\mathbf{H}_{t-1} + \mathbf{b}_i\right),\\
\mathbf{f}_t &= \sigma\!\left(\mathbf{W}_{xf}\mathbf{x}_t + \mathbf{W}_{hf}\mathbf{H}_{t-1} + \mathbf{b}_f\right),\\
\mathbf{o}_t &= \sigma\!\left(\mathbf{W}_{xo}\mathbf{x}_t + \mathbf{W}_{ho}\mathbf{H}_{t-1} + \mathbf{b}_o\right),\\
\mathbf{g}_t &= \tanh\!\left(\mathbf{W}_{xg}\mathbf{x}_t + \mathbf{W}_{hg}\mathbf{H}_{t-1} + \mathbf{b}_g\right),\\[6pt]
\mathbf{C}_t &= \mathbf{f}_t \odot \mathbf{C}_{t-1} + \mathbf{i}_t \odot \mathbf{g}_t,\\
\mathbf{H}_t &= \mathbf{o}_t \odot \tanh(\mathbf{C}_t).
\end{aligned}
\label{eq:lstm}
\end{equation}
\noindent
Em que $\mathbf{x}_t \in \mathbb{R}^d$ representa o vetor de entrada no instante $t$; $\mathbf{H}_{t-1}$ é o estado oculto no instante anterior; $\mathbf{C}_{t-1}$ corresponde ao estado da célula no instante anterior; $\mathbf{i}_t$ é a porta de entrada, que controla quanto da nova informação será incorporada; $\mathbf{f}_t$ é a porta de esquecimento, responsável por decidir o quanto da memória anterior será descartado; $\mathbf{o}_t$ é a porta de saída, que regula a contribuição do estado da célula para o estado oculto; $\mathbf{g}_t$ é o candidato de memória, vetor de novos conteúdos candidatos a serem adicionados; $\mathbf{C}_t$ é o estado de célula atualizado (memória de longo prazo); $\mathbf{H}_t$ é o estado oculto atualizado (saída da célula); $\sigma(\cdot)$ representa a função sigmoide logística, que restringe valores ao intervalo $(0,1)$; $\tanh(\cdot)$ representa a função tangente hiperbólica, que restringe valores ao intervalo $(-1,1)$; e $\odot$ denota o produto de Hadamard (produto elemento a elemento).


\begin{figure}[!h]
    \centering
    \caption{Arquitetura de uma célula LSTM}
    
    \includegraphics[width=0.8\textwidth]{2. Revisão Bibliográfica/Figuras/Arqueitetura LSTM.png}
    % \captionsetup{justification=centering}
    \par\small{Fonte: Adaptado de \citeonline{hochreiter1997}.}
    \label{fig:lstm-cell}
\end{figure}

\FloatBarrier


Como a LSTM opera sobre sequências, não é estritamente necessário segmentar em janelas; contudo, em séries longas (ou de alta frequência) utiliza-se janelas deslizantes para reduzir a complexidade. As ativações internas (sigmoide e $\tanh$) e a ativação de \emph{saída do modelo} (camada densa final) podem ser alteradas: para regressão de irradiância, usa-se tipicamente ativação \emph{linear} na saída; ReLU/LeakyReLU podem ser usados no candidato \( \mathbf{g}_t \) em cenários específicos, embora $\tanh$ seja o padrão. Evidências empíricas mostram a efetividade da LSTM em previsão de irradiância horária e diária, superando modelos lineares como ARIMA (\cite{d2l_lstm,solarReview2021}~); variantes recentes acoplam atenção e CNN para explorar componentes espaciais e temporais~(\cite{heliyon2023}~).

% --- Tabela de hiperparâmetros
\begin{table}[!h]
    \centering
    % \captionsetup{justification=centering}
    \caption{Hiperparâmetros da LSTM.}
    \resizebox{0.9\textwidth}{!}{%
    \begin{tabular}{|l|c|p{8.2cm}|c|}
        \hline
        \textbf{Hiperparâmetro} & \textbf{Símbolo} & \textbf{Efeito principal} & \textbf{Default} \\
        \hline
        Camadas ocultas & $n_{\text{layers}}$ & Profundidade da pilha LSTM; maior capacidade, maior risco de sobreajuste. & 2 \\
        Unidades por camada & $h$ & Dimensão de $\mathbf{H}_t$ e $\mathbf{C}_t$; controla a complexidade do modelo. & 64 \\
        Janela de entrada & $L_{\text{in}}$ & Nº de passos passados usados como entrada (janelas deslizantes). & 96 \,(=24h) \\
        Horizonte de previsão & $L_{\text{out}}$ & Nº de passos à frente previstos (multi-\emph{step}). & 4 \,(=1h) \\
        Dropout & $p$ & Regularização entre camadas/saídas recorrentes; reduz sobreajuste. & 0.2 \\
        Taxa de aprendizado & $\eta$ & Passo de atualização do otimizador. & $10^{-3}$ \\
        Peso L2 (decay) & $\lambda$ & Penaliza pesos grandes (regularização de Tikhonov). & $10^{-5}$ \\
        Tamanho do lote & $B$ & Amostras por atualização; afeta estabilidade e tempo de treino. & 64 \\
        Épocas & $E$ & Varreduras completas no treino; controla a convergência. & 100 \\
        Otimizador & -- & Regra de atualização dos parâmetros. & Adam \\
        Ativação das portas & $\sigma(\cdot)$ & Controla passagem (0–1) em $i_t,f_t,o_t$; pode ser \textit{hard-sigmoid}. & Sigmoide \\
        Ativação do candidato & $\phi(\cdot)$ & Gera $g_t$; afeta faixa/estabilidade do conteúdo adicionado. & $\tanh$ \\
        Ativação de saída (rede) & $\psi(\cdot)$ & Camada densa final; adequada ao alvo (contínuo/categórico). & Linear \\
        Bidirecionalidade & $\beta$ & Usa LSTM bidi (não causal; somente \emph{off-line}). & \texttt{False} \\
        Clipping de gradiente & $c_{\max}$ & Limite da norma do gradiente para estabilidade. & 1.0 \\
        Semente aleatória & $s$ & Reprodutibilidade do treino/embaralhamento. & 42 \\
        Parada antecipada & $P_{\text{ES}}$ & Paciência (épocas sem melhora no val.); evita sobreajuste. & 10 \\
        \hline
    \end{tabular}
    }
    \par\small{Fonte: Autor (2025)}
    
    \label{tab:lstm-hparams}
\end{table}

\FloatBarrier

\subsection{Redes neurais convolucionais (CNN).}  As CNN foram introduzidas por \citeonline{lecun1989} com o objetivo de reconhecer códigos postais manuscritos usando redes treinadas com retropropagação.  A primeira versão, denominada LeNet, utilizava camadas de convolução e subsampling (pooling) e foi aperfeiçoada ao longo dos anos; a LeNet‑5, lançada em 1998, atingiu taxa de erro de 0,95\% no conjunto MNIST (\cite{lecun1998}~).  Em redes convolucionais, a operação de convolução desliza um filtro (ou kernel) sobre a entrada para gerar \emph{mapas de características}.  Cada mapa utiliza pesos compartilhados para detectar um padrão local específico; os campos receptivos definem a vizinhança percebida por cada unidade.  A saída de uma convolução é seguida por uma operação de pooling (média ou máximo) que reduz a resolução espacial, conferindo invariância a pequenas translações.  A CNN é uma rede neural feedforward com pesos compartilhados e pertence à categoria de modelos de \emph{deep learning} determinísticos.

As CNNs foram projetadas para dados com estrutura espacial (imagens), mas podem ser adaptadas para séries temporais usando convoluções unidimensionais (1D).  Nessa adaptação, a série é segmentada em janelas de comprimento fixo e cada filtro convolucional detecta padrões locais ao longo do tempo; a saída pode ser combinada com modelos recorrentes ou camadas densas para gerar previsões.  As CNN oferecem vantagens como extração eficiente de características locais e capacidade de processamento paralelo; por outro lado, capturam apenas dependências de curto alcance e podem exigir camadas adicionais para incorporar contexto temporal.  Para previsão de irradiância solar, a combinação CNN+LSTM tem mostrado resultados promissores: um estudo recente propôs um modelo CNN com mecanismo de atenção e LSTM para prever a irradiância do dia seguinte, utilizando análise de dias semelhantes; os resultados indicaram maior precisão em comparação com abordagens convencionais (\cite{heliyon2023}~).

\subsection{Modelos ARIMA e a metodologia Box–Jenkins.}  O método Box–Jenkins, desenvolvido por \citeonline{box1970}, combina componentes autorregressivos (AR), de média móvel (MA) e de diferenciação (I) para modelar séries estacionárias.  A suposição fundamental é de que a série temporal seja estacionária; portanto, séries não estacionárias são diferenciadas uma ou mais vezes para atingir estacionaridade (\cite{nist_arima}~).  Um modelo ARIMA$(p,d,q)$ obedece à (\ref{eq:arima}).
\begin{equation}
\Phi(B)\,\Delta^d y_t = \Theta(B)\,\varepsilon_t,
\label{eq:arima}
\end{equation}
\noindent
Em que $\Phi(B)$ é o polinômio no operador de retardo $B$ de ordem $p$; $\Theta(B)$ é o polinômio no operador de retardo $B$ de ordem $q$; $\Delta^d$ representa o operador de diferenciação de ordem $d$; e $\varepsilon_t$ corresponde ao termo de ruído branco.

O método inclui etapas de identificação, estimação e validação do modelo (\cite{nist_arima}~). ARIMA pertence à categoria de modelos estatísticos lineares e é determinístico na fase de previsão.  As vantagens incluem simplicidade, interpretabilidade e capacidade de capturar padrões lineares e sazonais (quando estendido para SARIMA).  No entanto, exige estacionaridade e apresenta desempenho limitado em séries altamente não lineares ou caóticas.  Para séries com tendências sazonais marcantes, técnicas de decomposição como STL podem ser aplicadas antes do ajuste do ARIMA.  Na previsão de irradiância solar, modelos ARIMA foram utilizados desde a década de 1970: estudos pioneiros empregaram ARMA para prever irradiância e calor solar (\cite{solarReview2021}~). Como ARIMA é um modelo uni variado, janelas deslizantes podem ser usadas para treinar o modelo em subseções móveis da série, permitindo capturar mudanças locais de comportamento.

\subsection{RSTL}  A decomposição sazonal e de tendência por LOESS (STL) foi proposta por \citeonline{cleveland1990} e é uma técnica que separa uma série temporal em componentes de tendência, sazonalidade e resíduo usando regressão local (LOESS).  O método requer a escolha de comprimentos de suavização: o parâmetro \emph{seasonal} controla a suavização do componente sazonal, \emph{trend} define o tamanho da janela de tendência (normalmente 1{,}5 vezes a janela sazonal) e \emph{low\_pass} controla um filtro de baixa frequência (\cite{statsmodels_stl}~). STL não é um modelo de previsão por si só, mas sua saída pode ser combinada com modelos como ARIMA ou redes neurais para prever os componentes separadamente e recombiná‑los.  Uma extensão robusta (RSTL) utiliza ponderações dependentes dos dados para tolerar observações aberrantes. A decomposição STL pertence à categoria de métodos de pré‑processamento e é determinística; a robustez aumenta a resiliência a valores atípicos.  Para séries de irradiância solar, STL pode remover tendências e sazonalidades diurnas, permitindo que modelos regressivos ou de aprendizado de máquina se concentrem no componente residual.  Em termos de janelas temporais, a decomposição é aplicada à série completa ou a janelas rolantes, dependendo da estacionaridade local.

\subsection{XGBoost – Boosting de árvores com regularização.}  O algoritmo XGBoost, apresentado por \citeonline{chen2016}, é uma implementação eficiente de \emph{gradient boosting} que utiliza árvores de decisão como base.  O modelo aprende uma coleção de árvores de regressão ou classificação $\{f_k\}_{k=1}^K$ para prever a saída $\hat{y}_i = \sum_{k=1}^K f_k(\mathbf{x}_i)$ e otimiza a função objetivo \eqref{eq:xgboost-obj} composta por perda de treinamento e regularização.
\begin{equation}
\text{obj}(\theta) = \sum_{i=1}^n l(y_i, \hat{y}_i) + \sum_{k=1}^K \Omega(f_k),
\label{eq:xgboost-obj}
\end{equation}
\noindent
Em que $l$ representa a função de perda utilizada no treinamento, tipicamente o erro quadrático médio (MSE) em problemas de regressão; e $\Omega(f_k)$ é o termo de regularização que mede a complexidade da árvore $f_k$ (por exemplo, número de folhas e magnitude dos pesos).


Durante o treinamento, as árvores são adicionadas de forma aditiva e a expansão de Taylor de segunda ordem da perda é usada para obter gradientes $g_i$ e Hessianas $h_i$ para cada observação (\cite{chen2016}~).  A regularização controla o tamanho e os pesos das árvores, sendo definida por \eqref{eq:xgb-regularizacao}~.

\begin{equation}
\Omega(f) \;=\; \gamma T \;+\; \tfrac{1}{2}\lambda \sum_{j=1}^T w_j^2,
\label{eq:xgb-regularizacao}
\end{equation}
\noindent
Em que $T$ é o número de folhas da árvore; $w_j$ representa o peso associado à $j$-ésima folha; $\gamma$ é o parâmetro de penalização pelo número de folhas (complexidade estrutural); e $\lambda$ é o parâmetro de regularização L2 aplicado aos pesos das folhas.


XGBoost é um modelo de \emph{ensemble} baseado em árvores, pertencente à categoria de métodos de aprendizado de máquina supervisionado e não probabilístico. Suas vantagens incluem alta precisão, capacidade de lidar com dados esparsos e regularização incorporada. As limitações são a complexidade computacional crescente com o número de árvores e a menor interpretabilidade em comparação com modelos lineares. 

Embora não seja intrinsecamente temporal, XGBoost pode ser aplicado a séries temporais mediante a criação de variáveis de atraso (\emph{lag features}) e janelas deslizantes; cada janela fornece um vetor de entrada e a previsão corresponde ao próximo valor. Outra abordagem consiste em treinar um modelo separado para cada horizonte de previsão (\emph{direct multi-step}), em que cada estimador $f_k$ prevê um passo à frente específico, e as saídas de todos os modelos são então combinadas para formar a previsão em múltiplos passos. 

O desempenho do XGBoost depende fortemente da escolha dos seus hiperparâmetros de treino, como o número de árvores $M$, a profundidade máxima $d_{\max}$, a taxa de aprendizado $\eta$, as taxas de amostragem de linhas e colunas ($s_{\text{row}}$ e $s_{\text{col}}$), e os termos de regularização ($\alpha$, $\lambda$, $\gamma$). A Tabela~\ref{tab:xgb-hparams} resume os principais hiperparâmetros, seus efeitos e valores usuais.



\begin{table}[!h]
    \centering
    \caption{Principais Hiperparâmetros do XGBoost.}
    \resizebox{\textwidth}{!}{%
    \begin{tabular}{|l|c|p{8.2cm}|c|}
        \hline
        \textbf{Hiperparâmetro} & \textbf{Símbolo} & \textbf{Efeito principal} & \textbf{Default} \\
        \hline
        Número de árvores & $M$ & Define o número de iterações de boosting (árvores no ensemble). & 100 \\
        Profundidade máxima & $d_{\max}$ & Controla a complexidade de cada árvore; maior profundidade captura mais relações, mas aumenta risco de sobreajuste. & 6 \\
        Taxa de aprendizado & $\eta$ & Reduz a contribuição de cada árvore; valores menores melhoram generalização, exigindo mais árvores. & 0.3 \\
        Subamostragem de instâncias & $s_{\text{row}}$ & Proporção de exemplos usados em cada árvore; reduz variância e acelera treino. & 1.0 \\
        Subamostragem de colunas & $s_{\text{col}}$ & Proporção de atributos usados por árvore ou nó; promove diversidade. & 1.0 \\
        Peso L1 & $\alpha$ & Regularização L1 nos pesos das folhas; promove esparsidade. & 0 \\
        Peso L2 & $\lambda$ & Regularização L2 nos pesos das folhas; estabiliza os coeficientes. & 1 \\
        Mínimo de perda para divisão & $\gamma$ & Ganho mínimo requerido para que um nó seja dividido; controla a complexidade estrutural. & 0 \\
        Número mínimo de amostras por folha & $n_{\min}$ & Restringe folhas com poucos exemplos, evitando sobreajuste em outliers. & 1 \\
        Objetivo & $l(\cdot)$ & Função de perda; em regressão pode ser \texttt{reg:squarederror}. & \texttt{reg:squarederror} \\
        Métrica de avaliação & $m(\cdot)$ & Usada no monitoramento do treino; ex.: RMSE, MAE. & RMSE \\
        Parada antecipada & $P_{\text{ES}}$ & Número de iterações sem melhora antes de encerrar o treino. & -- \\
        Semente aleatória & $s$ & Controla a reprodutibilidade do resultado. & 42 \\
        \hline
    \end{tabular}
    }
    \par\small{Fonte: Autor (2025)}
    \label{tab:xgb-hparams}
\end{table}

\FloatBarrier
\subsection{Modelo de persistência (Naïve).}  O modelo de persistência, também chamado de previsão ingênua ou \emph{no‑change}, utiliza o último valor observado como previsão de todos os horizontes futuros: $\hat{y}_{t+h} = y_t$ para $h\geq 1$ .  Esse modelo serve como referência mínima, pois é teoricamente ótimo para séries que seguem um passeio aleatório (\emph{random walk}).  Em séries com elevado nível de ruído ou comportamento integrado, modelos mais sofisticados dificilmente superam o desempenho do naïve.  A persistência pertence à categoria de métodos determinísticos e não requer parametrização; sua implementação trivial faz dele um benchmark indispensável em comparação de modelos de previsão.  Para séries de irradiância solar, o modelo de persistência é frequentemente usado para avaliar o ganho obtido por modelos complexos.  Ele pode ser interpretado como um caso extremo de janela deslizante de largura 1 (\citeonline{petropoulos2022}~).

\subsection{Discussão comparativa.}  Os modelos revisados diferem quanto à estrutura, capacidade de capturar relações temporais e requisitos de dados.  ARIMA e STL são métodos estatísticos clássicos que assumem linearidade e estacionaridade; ARIMA modela a dependência temporal de forma explícita, enquanto STL decompõe a série antes da modelagem.  LSTM e CNN representam redes de aprendizado profundo que aprendem representações não lineares; a LSTM é inerentemente temporal, ao passo que a CNN precisa de adaptação por convoluções 1D e janelas.  XGBoost pertence à classe de ensembles de árvores e oferece equilíbrio entre performance e controle de \emph{overfitting}; é versátil, mas exige transformação de características para dados temporais.  O modelo de persistência, apesar de simples, fornece uma base imprescindível para avaliar se abordagens mais complexas realmente agregam valor.  Métodos baseados em decomposição, como STL/RSTL, são eficazes para isolar componentes sazonais e melhorar o desempenho de preditores subsequentes.  Em todos os casos, a escolha do modelo deve considerar a natureza da série (tendências, sazonalidades, ruído), a disponibilidade de dados e o objetivo operacional.

% \input{2. Revisão Bibliográfica/2.6 Base de dados}

% % ---------------------------------------
% % CAPÍTULO 3 — Dados e Pré-processamento
% % ---------------------------------------
% \chapter{Dados e Pré-processamento}\label{chap:dados}
% \input{cap3-dados-preproc/01-fontes-brsn-inmet}
% \input{cap3-dados-preproc/02-qualidade-dados-fusohorario-outliers}
% \input{cap3-dados-preproc/03-features-lags-rolling-derivadas}
% \input{cap3-dados-preproc/04-normalizacao-splits-janelas}
% \input{cap3-dados-preproc/05-metricas-definicoes}

% ---------------------------------------
% CAPÍTULO 4 — Metodologia
% ---------------------------------------
\chapter{Metodologia}\label{chap:metodologia}

Este capítulo apresenta de forma panorâmica o percurso metodológico adotado no estudo, sintetizando as principais etapas que estruturam o fluxo de trabalho, conforme ilustrado na Figura \ref{fig:fluxo_metodologia}. O processo parte da seleção e preparação da base de dados, passando pela engenharia de atributos e análise exploratória, até chegar à modelagem, otimização de hiperparâmetros e geração das previsões. Cada etapa será detalhada em subseções específicas, mas aqui é fornecida uma visão geral que contextualiza o encadeamento das atividades.

\begin{figure}[!h]
    \centering
    \caption{Fluxo metodológico para previsão de irradiância solar.}
    \includegraphics[width=1\textwidth]{Figuras/Flowchart.jpg}
    \par\small{Fonte: Autor (2025)}
    \label{fig:fluxo_metodologia}
\end{figure}

Na etapa de \textit{seleção da base de dados}, são discutidos os critérios de escolha do conjunto utilizado, com atenção à sua cobertura temporal, qualidade de medição e representatividade climática. Essa escolha fundamenta todo o processo, uma vez que garante a disponibilidade de dados consistentes para o treinamento e avaliação dos modelos.

Em seguida, ocorre a \textit{limpeza e padronização}, fase em que se trata de valores ausentes, inconsistências e discrepâncias de escala. O objetivo é assegurar a coerência dos dados e prepará-los para a etapa seguinte.

A \textit{engenharia de atributos e análise exploratória} constitui um dos núcleos da metodologia. São exploradas as características estatísticas da base, bem como construídas variáveis derivadas com potencial de enriquecer os modelos preditivos. Entre os principais atributos considerados estão o índice de claridade atmosférica ($k_t$), variáveis sazonais (como hora do dia e mês), resíduos de modelos autorregressivos (como ARIMA) e codificações que refletem padrões cíclicos. Essa etapa é fundamental para capturar tanto a variabilidade física da irradiância quanto os efeitos estatísticos observados na série temporal.

A partir desse conjunto enriquecido, é consolidada a \textit{base final}, que serve de entrada para a etapa de modelagem. Na \textit{escolha do modelo de previsão}, são explorados diferentes algoritmos com naturezas complementares: redes neurais recorrentes do tipo LSTM, capazes de capturar dependências temporais de longo prazo; modelos baseados em árvores de decisão, como o XGBoost, reconhecidos pela robustez e capacidade de lidar com relações não lineares; e arquiteturas baseadas em mecanismos de atenção, que possibilitam maior flexibilidade na identificação de padrões relevantes ao longo do tempo. Essa diversidade de abordagens permite comparar desempenhos sob diferentes perspectivas.

A etapa seguinte é a de \textit{otimização de hiperparâmetros}, na qual os modelos são ajustados por meio de técnicas de busca sistemática. São exploradas abordagens como o \textit{grid search}, que avalia combinações exaustivas de parâmetros, e a \textit{Bayesian Optimization}, que busca soluções mais eficientes por meio da exploração guiada do espaço de busca. O objetivo é identificar configurações que maximizem a capacidade preditiva dos modelos.

Por fim, com os modelos ajustados, é realizada a etapa de \textit{previsão}, em que se avalia o desempenho sobre um conjunto de teste independente. Essa fase contempla a comparação entre diferentes modelos e estratégias de pré-processamento, permitindo identificar as soluções mais adequadas ao problema de previsão de irradiância solar.

Dessa forma, a Figura \ref{fig:fluxo_metodologia} resume o encadeamento metodológico, enquanto este capítulo introduz cada etapa em linhas gerais, preparando o terreno para o detalhamento subsequente em seções específicas.


\FloatBarrier


\section{Seleção da Base de Dados}

A escolha da base de dados constitui etapa fundamental em estudos de previsão de irradiância solar, uma vez que a qualidade e a completude das observações impactam diretamente a robustez dos modelos preditivos. Em linhas gerais, será buscada uma base que contemple as seguintes características desejáveis:

\begin{itemize}
    \item \textbf{Alta resolução temporal:} dados com amostragem em escala de minutos, possibilitando posterior agregação em janelas de 15 minutos e a captura de variações rápidas na irradiância;
    \item \textbf{Confiabilidade das medições:} séries provenientes de estações meteorológicas de referência, com instrumentação calibrada e documentação do processo de aquisição;
    \item \textbf{Amplitude de variáveis meteorológicas:} além da irradiância solar, a inclusão de variáveis exógenas como temperatura, umidade relativa e pressão atmosférica, que auxiliarão na modelagem das condições atmosféricas;
    \item \textbf{Extensão temporal suficiente:} histórico de múltiplos anos, permitindo contemplar diferentes estações, ciclos sazonais e padrões climáticos;
    \item \textbf{Disponibilidade pública ou institucional:} acesso transparente e reprodutível, garantindo a possibilidade de replicação do estudo por outros pesquisadores.
\end{itemize}

Embora a descrição detalhada da base utilizada seja apresentada no estudo de caso, nesta seção metodológica estabelecem-se as diretrizes que nortearão a sua seleção. Dessa forma, assegurar-se-á que o processo de escolha não se limite a uma decisão circunstancial, mas seja fundamentado em critérios técnicos alinhados às boas práticas da literatura.

\section{Pré-processamento e Engenharia de Dados}

Após a seleção da base de dados, será realizado o pré-processamento das séries temporais de irradiância e variáveis meteorológicas. Esta etapa é fundamental para garantir a consistência e a confiabilidade do conjunto de dados a ser utilizado na modelagem, além de possibilitar a extração de informações adicionais relevantes por meio da engenharia de atributos. O processo será conduzido em múltiplas fases, descritas a seguir.

\subsection{Análise de completude e qualidade dos dados}

Inicialmente, será realizada a verificação da \textit{completude temporal} da base, avaliando-se o número de minutos observados em cada ano em comparação ao esperado para uma série contínua. Essa análise permitirá identificar anos com lacunas significativas e quantificar a proporção de registros efetivamente disponíveis. A inspeção da completude temporal é relevante, pois séries com baixa cobertura histórica tendem a prejudicar a capacidade de generalização dos modelos preditivos.

\subsection{Energia diária e definição do intervalo de análise}

Para definir a faixa horária mais adequada ao estudo, será calculada a energia diária total de irradiância incidente. A energia em um dia $d$ será obtida pela integração da irradiância ao longo do tempo, conforme a Equação (\ref{eq:energia_diaria}).

\begin{equation}
E_d = \int_{t \in d} SWD(t)\, dt \approx \sum_{i=1}^{N_d} SWD(t_i) \cdot \Delta t
\label{eq:energia_diaria}
\end{equation}
\noindent
Em que $E_d$ representa a energia diária total incidente em um dia $d$ (Wh/m$^2$); $SWD(t_i)$ é a irradiância global no instante $t_i$ (W/m$^2$); $\Delta t$ corresponderá ao passo temporal de integração (1 minuto); e $N_d$ será o número de amostras disponíveis no dia.

Antes do cálculo da energia, será verificado o fuso horário dos dados. Em diversas bases meteorológicas e radiométricas, os registros são fornecidos em tempo universal coordenado (UTC). Para que os valores façam sentido físico em relação ao nascer e ao pôr do sol locais, será necessário realizar a conversão para o fuso horário da estação de medição, no caso UTC$-3$. Esse ajuste assegurará que a distribuição horária da irradiância corresponda à realidade, permitindo que o modelo represente adequadamente os padrões locais.

A análise dos perfis de energia ao longo das 24 horas indicará o período em que existe energia efetivamente relevante para alimentar os modelos. Assim, será adotado esse intervalo como janela de observação, reduzindo-se a quantidade de valores nulos (noturnos) e concentrando a modelagem nos períodos mais significativos para aplicações fotovoltaicas.

\subsection{Tratamento de valores faltantes e lacunas temporais}

Após o recorte temporal, serão verificadas lacunas de diferentes magnitudes. O tratamento será realizado de forma seletiva:

\begin{itemize}
    \item \textbf{Lacunas curtas} (até 30 minutos): serão preenchidas por interpolação utilizando a função \texttt{interpolate} da biblioteca \texttt{pandas}. No caso do método linear com índice temporal, os valores intermediários serão calculados pela Equação (\ref{eq:interp_linear}), assumindo variação linear entre os pontos vizinhos.

    \begin{equation}
        y(t) = y(t_a) + \frac{y(t_b) - y(t_a)}{t_b - t_a} \cdot (t - t_a)
        \label{eq:interp_linear}
    \end{equation}

    Em que $y(t)$ será o valor interpolado para um instante intermediário $t$; $t_a$ representará o instante anterior válido à lacuna; e $t_b$ representará o instante posterior válido à lacuna.

    \item \textbf{Lacunas longas} (superiores a 30 minutos): dias contendo tais falhas serão descartados, de forma a não comprometer a consistência da série.
\end{itemize}

\subsection{Correção de valores espúrios}

Serão identificados valores fisicamente inválidos ou inconsistentes nas medições, tais como: irradiância solar negativa ($SWD < 0$), removida por não ter significado físico; umidade relativa fora do intervalo $[0, 100]\,\%$, corrigida por truncamento nos limites; e pressão atmosférica ou temperatura fora de faixas climatológicas plausíveis para a região (850–1100 hPa e $[-10, 50]^{\circ}C$, respectivamente), que serão descartadas ou substituídas por interpolação. Adicionalmente, valores iguais a zero durante o período diurno (07:00–20:00) serão tratados como falhas instrumentais e corrigidos por interpolação quando isolados.

\subsection{Reamostragem temporal}

Visando reduzir a variabilidade de alta frequência e alinhar a resolução da previsão com aplicações energéticas, as séries serão reamostradas para intervalos de 15 minutos. O valor médio da irradiância em cada janela será descrito pela Equação (\ref{eq:reamostragem}).

\begin{equation}
SWD^{15}(t) = \frac{1}{N}\sum_{i=1}^{N} SWD(t_i)
\label{eq:reamostragem}
\end{equation}

Em que $SWD^{15}(t)$ será a irradiância média reamostrada a cada 15 minutos; $SWD(t_i)$ corresponderão aos valores originais de irradiância na janela; e $N$ será o número de observações originais dentro da janela de 15 minutos.

\subsection{Variáveis derivadas a partir de modelos físicos}

Será utilizado o modelo de céu claro de \textit{Ineichen}, implementado na biblioteca \texttt{pvlib}, para estimar a irradiância teórica sem atenuação atmosférica ($SWD_{cs}$). A partir disso, será construído o índice de claridade $k_t^{*}$, conforme a Equação (\ref{eq:ktstar}).

\begin{equation}
k_t^{*}(t) = \frac{SWD(t)}{SWD_{cs}(t)}
\label{eq:ktstar}
\end{equation}

Em que $k_t^{*}(t)$ será o índice de claridade adimensional; $SWD(t)$ corresponderá à irradiância global medida no instante $t$ (W/m$^2$); e $SWD_{cs}(t)$ será a irradiância estimada para céu claro (W/m$^2$).

Os valores serão truncados no intervalo $[0,1]$, de forma a evitar distorções por inconsistências de medição. Este índice fornecerá uma medida da transparência atmosférica, capturando a influência de nuvens e aerossóis.


\subsection{Codificação de variáveis sazonais}

A variabilidade intradiária e anual da irradiância será incorporada por meio de uma mesma família de funções sazonais contínuas, evitando descontinuidades típicas de variáveis categóricas e garantindo padronização entre diferentes escalas temporais. A estratégia consistirá em normalizar a variável cíclica para o intervalo $[0,1]$ e, em seguida, aplicar uma função periódica suave com pico controlável por um parâmetro de fase. A forma geral será apresentada na Equação (\ref{eq:sazonal_unificada}).

\begin{equation}
\text{saz}(t) \;=\; \cos^{2}\!\left(\pi \cdot \Big(\tilde{x}(t) - \varphi\Big)\right),
\qquad
\tilde{x}(t) \;=\; \frac{x(t)-x_{\min}}{x_{\max}-x_{\min}}
\label{eq:sazonal_unificada}
\end{equation}

\noindent
Em que $\text{saz}(t)$ representará a codificação sazonal contínua e adimensional no instante $t$; $x(t)$ será a variável cíclica de interesse (hora do dia ou mês do ano); $\tilde{x}(t)$ será a versão normalizada de $x(t)$ para o intervalo $[0,1]$; $x_{\min}$ e $x_{\max}$ delimitarão o ciclo considerado; e $\varphi \in [0,1]$ será o parâmetro de \textit{fase}, que posicionará o máximo da função (pico) ao longo do ciclo.

A escolha de $\cos^{2}(\cdot)$ apresenta três vantagens práticas: (i) domínio naturalmente limitado em $[0,1]$, evitando escalas negativas; (ii) suavidade e derivabilidade, úteis a modelos baseados em gradiente; e (iii) controle explícito da posição do pico por meio de $\varphi$. Note-se que $\sin^{2}(\cdot)$ é equivalente a um deslocamento de fase de $\cos^{2}(\cdot)$; portanto, adotar-se-á uma única forma para padronizar a construção.

Do ponto de vista metodológico, a construção acima evitará descontinuidades entre valores adjacentes (por exemplo, entre mês=12 e mês=1), padronizará a amplitude em $[0,1]$ e permitirá alinhar os picos aos comportamentos físicos de interesse por meio de $\varphi$. Essa padronização também facilitará a interpretação e a comparação entre estudos, uma vez que a mesma forma funcional será aplicada às diferentes escalas temporais consideradas.

\subsection{Engenharia de variáveis adicionais}

Serão introduzidos atributos derivados da própria série de irradiância com o objetivo de fornecer ao modelo informações explícitas sobre a dinâmica temporal, conforme as Equações (\ref{eq:deltas})--(\ref{eq:direcao}).

\begin{align}
\Delta SWD_{1}(t) &= SWD(t) - SWD(t-1), \nonumber \\
\Delta SWD_{5}(t) &= SWD(t) - SWD(t-5), \nonumber \\
\Delta SWD_{10}(t) &= SWD(t) - SWD(t-10)
\label{eq:deltas}
\end{align}

\begin{align}
SWD_{m3}(t) &= \frac{1}{3}\sum_{i=0}^{2} SWD(t-i), \nonumber \\
SWD_{m9}(t) &= \frac{1}{9}\sum_{i=0}^{8} SWD(t-i), \nonumber \\
\Delta SWD_{mmed}(t) &= SWD_{m3}(t) - SWD_{m9}(t)
\label{eq:medias}
\end{align}

\begin{equation}
dir_{SWD}(t) = \text{sign}\left(\Delta SWD_{1}(t)\right)
\label{eq:direcao}
\end{equation}

\noindent
Em que $\Delta SWD_{1}(t)$, $\Delta SWD_{5}(t)$ e $\Delta SWD_{10}(t)$ representarão variações de curto prazo em diferentes escalas temporais; $SWD_{m3}(t)$ e $SWD_{m9}(t)$ corresponderão às médias móveis de 3 e 9 passos, respectivamente; $\Delta SWD_{mmed}(t)$ será definido como a diferença entre as duas médias móveis, funcionando como um indicador explícito de tendência; e $dir_{SWD}(t)$ será a variável categórica derivada do sinal de $\Delta SWD_{1}(t)$, assumindo valores $+1$ para tendência de subida, $-1$ para tendência de descida e $0$ para estabilidade.

Em síntese, essas variáveis adicionais serão construídas com o intuito de ampliar a capacidade dos modelos de aprendizado de máquina em capturar comportamentos de curtíssimo prazo (rampas), padrões de oscilação (diferenças em múltiplos horizontes), tendências locais (diferença entre médias) e direção da variação (sinal).

\subsection{Incorporação de resíduo ARIMA}

Adicionalmente, será ajustado um modelo autorregressivo integrado de médias móveis (ARIMA) simples sobre a série de irradiância. O ARIMA é um modelo estatístico clássico para séries temporais, capaz de capturar dependências lineares por meio de três componentes principais: (i) a parte autorregressiva (AR), que modela a série em função de valores passados; (ii) a parte de diferenciação (I), que garantirá a estacionariedade por meio de diferenças sucessivas; e (iii) a parte de médias móveis (MA), que considerará o efeito de choques ou erros de previsão passados. A previsão fornecida pelo ARIMA, denotada por $SWD_{ARIMA}(t)$, corresponderá a uma estimativa linear da irradiância baseada na dinâmica histórica da série.

A partir dessa previsão, será calculado o resíduo, definido na Equação (\ref{eq:arima_residuo}), que representará a parcela da variabilidade não explicada pelo modelo linear.

\begin{equation}
Res_{ARIMA}(t) = SWD(t) - SWD_{ARIMA}(t)
\label{eq:arima_residuo}
\end{equation}

\noindent
Em que $SWD_{ARIMA}(t)$ será a previsão de irradiância obtida pelo modelo ARIMA; e $Res_{ARIMA}(t)$ será o resíduo, correspondente à diferença entre a série observada e a previsão linear.

A motivação para incorporar tanto a previsão $SWD_{ARIMA}(t)$ quanto o resíduo $Res_{ARIMA}(t)$ como variáveis explicativas está no fato de que o ARIMA tende a capturar bem o comportamento médio e suave da série, mas não reproduz de forma satisfatória variações repentinas, como aquelas provocadas por nuvens de passagem rápida. Nesse contexto, espera-se que o resíduo contenha informação adicional sobre essas flutuações abruptas, fornecendo ao modelo de aprendizado de máquina um sinal explícito de discrepâncias entre o comportamento esperado (linear) e o observado (não linear).

\subsection{Integração de previsores do dia seguinte}

Por fim, será incorporado ao modelo um conjunto de informações meteorológicas que estariam disponíveis em um cenário real de previsão. Para simular essa situação, serão utilizadas as próprias séries de temperatura, umidade relativa e pressão atmosférica, deslocadas em um dia para trás, gerando as variáveis descritas pela Equação (\ref{eq:prev_meteo}).

\begin{equation}
X_{prev}(t) = X(t+1\,dia), \quad X \in \{\text{Temp}, RH, Pressure\}
\label{eq:prev_meteo}
\end{equation}

\noindent
Em que $X_{prev}(t)$ representará a série deslocada de um dia, utilizada como proxy da previsão meteorológica operacional para o instante $t$; e $X(t)$ corresponderá à variável meteorológica original (temperatura, umidade relativa ou pressão).

A motivação para a inclusão dessas variáveis será fornecer ao modelo uma visão antecipada de possíveis mudanças meteorológicas no dia seguinte. Por exemplo, no meio da tarde poderá ocorrer queda de temperatura, aumento de umidade ou precipitação, fatores que impactarão diretamente a irradiância solar. Atualmente, tais previsões são amplamente disponíveis por modelos numéricos de previsão do tempo (NWP), sendo natural sua utilização em sistemas de previsão solar.

Entretanto, muitas bases históricas de dados radiométricos não fornecem diretamente a previsão meteorológica associada. Nesses casos, poderá ser assumido que a melhor aproximação para a previsão de uma variável no dia seguinte é o próprio valor observado nesse dia, deslocado no tempo. Essa construção garantirá que o modelo seja capaz de acompanhar tendências gerais do dia seguinte, ainda que na prática, em um cenário operacional, exista um erro adicional proveniente da incerteza da previsão.

Com o intuito de simular essa incerteza, também serão construídas versões perturbadas por ruído gaussiano $N(0,\sigma^2)$, resultando nas variáveis $X_{prev\_ruido}$. Essa abordagem representará situações em que a previsão meteorológica não é exata, mas possui margens de erro, como no caso de previsões probabilísticas de chuva (por exemplo, 80\% de chance de precipitação). Assim, será avaliado não apenas o impacto da previsão meteorológica perfeita, mas também a resiliência do modelo diante de previsões imperfeitas.

O resultado desse processo de pré-processamento e engenharia de dados será um conjunto de variáveis originais e derivadas que sintetizarão tanto informações físicas quanto estatísticas da série temporal, ampliando a capacidade potencial dos modelos preditivos em capturar padrões complexos de variação da irradiância solar. Ressalta-se, contudo, que a utilidade efetiva de cada variável auxiliar não será assumida previamente. Todas as variáveis criadas serão analisadas e testadas na etapa de experimentos, de forma a verificar sua relevância prática para a tarefa de previsão. Assim, a seleção final das variáveis a serem utilizadas será fundamentada nos resultados obtidos, considerando tanto o impacto sobre as métricas de desempenho quanto a análise de importância de atributos.


\section{Análise de Dados}

Após a etapa de seleção e pré-processamento da base, será realizada a análise exploratória de dados (AED), também denominada ciência de dados aplicada. O objetivo dessa etapa será compreender a estrutura estatística e temporal da série de irradiância e das variáveis meteorológicas associadas, de modo a embasar a engenharia de variáveis e a posterior modelagem preditiva. Cada análise descrita a seguir será conduzida com a biblioteca \texttt{pandas} para manipulação de dados, enquanto as visualizações serão construídas com \texttt{matplotlib} e \texttt{seaborn}.

\subsection{Distribuição temporal dos dados}
Inicialmente, será verificada a granularidade da série temporal, identificando-se as horas únicas presentes no índice. Essa verificação permitirá confirmar a consistência da base e assegurar que os registros possuam periodicidade regular de 15 minutos. Espera-se, com isso, confirmar a adequação do conjunto para análises sazonais e aplicação de modelos de séries temporais.

\subsection{Irradiância por hora do dia}
A irradiância global será agregada por hora do dia conforme a Equação (\ref{eq:soma_hora}).
\begin{equation}
SWD_{h} = \sum_{t \in h} SWD_{t}
\label{eq:soma_hora}
\end{equation}

\noindent
Em que $SWD_{t}$ representará a irradiância no instante $t$; e $SWD_{h}$ será a soma horária da irradiância.

Essa análise permitirá avaliar a contribuição relativa de cada hora para a irradiância total diária, caracterizando o perfil médio da curva solar. Espera-se identificar os horários de maior intensidade, fundamentais para a construção de variáveis sazonais.

\subsection{Completude da base de dados}
Será construída uma tabela cruzada de contagem de registros por hora e por mês, representada pela Equação (\ref{eq:contagem_dados}).
\begin{equation}
N_{m,h} = \sum \mathbb{1}_{\{SWD_{t} \neq \emptyset\}}, \quad t \in (m,h)
\label{eq:contagem_dados}
\end{equation}

\noindent
Em que $N_{m,h}$ será o número de observações no mês $m$ e hora $h$; e $\mathbb{1}_{\{SWD_{t} \neq \emptyset\}}$ será a função indicadora que assumirá valor 1 quando existir registro em $t$ e 0 caso contrário.

Essa análise será fundamental para verificar a qualidade e uniformidade da base, evitando que anos incompletos comprometam a modelagem.

\subsection{Médias mensais e anuais}
A irradiância e as variáveis meteorológicas serão agregadas por mês e ano, conforme a Equação (\ref{eq:media_mensal}).
\begin{equation}
\overline{X}_{a,m} = \frac{1}{n_{a,m}} \sum_{t \in (a,m)} X_{t}
\label{eq:media_mensal}
\end{equation}

\noindent
Em que $X_{t}$ representará a variável analisada (irradiância, temperatura, umidade relativa ou pressão atmosférica); $n_{a,m}$ será o número de observações no mês $m$ do ano $a$; e $\overline{X}_{a,m}$ será o valor médio mensal da variável.

Com isso, buscar-se-á caracterizar padrões sazonais e variabilidade interanual, aspectos relevantes para avaliar a estabilidade dos modelos.

\subsection{Distribuições univariadas}
Serão elaborados histogramas e boxplots de todas as variáveis. Os histogramas permitirão visualizar a densidade de probabilidade empírica, enquanto os boxplots destacarão mediana, quartis e outliers. Espera-se, com essa análise, identificar assimetrias, dispersões e valores extremos que influenciam a modelagem.

\subsection{Relações bivariadas}
Para formalizar dependências entre variáveis meteorológicas, serão calculados coeficientes de correlação de Pearson e Spearman, apresentados nas Equações (\ref{eq:pearson}) e (\ref{eq:spearman}), respectivamente.

\begin{equation}
\rho_{xy}^{Pearson} = \frac{\sum (x_i - \overline{x})(y_i - \overline{y})}{\sqrt{\sum (x_i - \overline{x})^2 \sum (y_i - \overline{y})^2}}
\label{eq:pearson}
\end{equation}

\begin{equation}
\rho_{xy}^{Spearman} = 1 - \frac{6 \sum d_i^2}{n(n^2-1)}
\label{eq:spearman}
\end{equation}

\noindent
Em que $x_i, y_i$ serão pares de observações; $\overline{x}, \overline{y}$ serão as médias de $x$ e $y$; $d_i$ será a diferença entre os postos de $x_i$ e $y_i$; e $n$ será o número de observações.

A correlação de Pearson medirá associações lineares, enquanto a de Spearman capturará monotonicidade. Ambas fornecerão subsídios para a seleção de variáveis meteorológicas relevantes.

\subsection{Agrupamentos por faixas}
A irradiância será analisada por categorias de temperatura e umidade relativa. Para tanto, as variáveis serão discretizadas em intervalos, conforme a Equação (\ref{eq:faixas}).
\begin{equation}
C_{j} = \{ x \in \mathbb{R} \;|\; b_{j-1} \leq x < b_{j} \}
\label{eq:faixas}
\end{equation}

\noindent
Em que $C_{j}$ será a $j$-ésima classe de discretização; e $b_{j}$ serão os limites dos intervalos definidos de forma equi espaçada.

Essa abordagem permitirá avaliar a média de irradiância em condições atmosféricas específicas, o que poderá auxiliar na criação de variáveis indicadoras.

\section{Modelagem}

Essa seção descreverá a preparação dos dados e a infraestrutura de rastreamento experimental que serão adotadas para os modelos de previsão de irradiância. Em cada etapa, explicitar-se-á \textit{por que} será realizada, \textit{como} será implementada (bibliotecas, estruturas e equações) e \textit{o que se espera} como impacto metodológico. 

\subsection{Preparação dos Dados para Modelagem}

A preparação será implementada com \texttt{pandas} e \texttt{numpy} (manipulação temporal/tabular), \texttt{scikit-learn} (escalonamento), e utilidades de janelamento como \texttt{tricks.sliding}. O objetivo será transformar a série multivariada em um conjunto supervisionado de pares entrada--saída, controlando sazonalidade, escala e vazamento de informação.

\subsection{Definição de variáveis e alvo}
A cada instante $t$ será considerado um vetor de preditores $\mathbf{x}_t \in \mathbb{R}^{p}$ e uma variável-alvo $y_t \in \mathbb{R}$ (irradiância). Em notação matricial, o painel multivariado será dado por $\mathbf{X} = [\mathbf{x}_1,\dots,\mathbf{x}_T]^\top \in \mathbb{R}^{T \times p}$ e o alvo por $\mathbf{y} = [y_1,\dots,y_T]^\top \in \mathbb{R}^{T \times 1}$. A separação explícita formalizará a construção posterior de janelas.

\subsection{Seleção temporal e consistência}
Para reduzir vieses por anos incompletos e lacunas sazonais, a série será filtrada conforme a Equação (\ref{eq:filtro_anos}).
\begin{equation}
\mathcal{T}^{\ast} = \{\, t \in \mathcal{T} \;|\; \mathrm{year}(t) \notin \mathcal{A}_{\mathrm{rem}} \,\}
\label{eq:filtro_anos}
\end{equation}

\noindent
Em que $\mathcal{T}$ será o conjunto temporal original; $\mathcal{A}_{\mathrm{rem}}$ será o conjunto de anos removidos por incompletude; e $\mathcal{T}^{\ast}$ será o conjunto temporal resultante.

Espera-se melhorar a robustez dos estimadores, preservando a sazonalidade intrínseca.

\subsection{Particionamento temporal }
Será adotado corte cronológico puro (treino/validação/teste) para evitar vazamento do futuro para o passado, conforme a Equação (\ref{eq:split_temporal}).
\begin{equation}
\mathcal{D}_{\mathrm{train}} = \{\, t \leq \tau_1 \,\}, \quad 
\mathcal{D}_{\mathrm{val}}   = \{\, \tau_1 < t \leq \tau_2 \,\}, \quad
\mathcal{D}_{\mathrm{test}}  = \{\, t > \tau_2 \,\}
\label{eq:split_temporal}
\end{equation}

\noindent
Em que $\tau_1$ e $\tau_2$ serão marcos temporais (datas) do particionamento; e $\mathcal{D}_{\mathrm{train}}, \mathcal{D}_{\mathrm{val}}, \mathcal{D}_{\mathrm{test}}$ serão subconjuntos temporais disjuntos.

Espera-se avaliação realista de generalização fora da amostra.

\subsection{Escalonamento e inversão de escala}

Para redes neurais e modelos baseados em distância, será aplicado exclusivamente o escalonamento do tipo \textit{min--max}, ajustado no conjunto de treino $\mathcal{D}_{\mathrm{train}}$ e posteriormente aplicado aos conjuntos de validação e teste. Esse procedimento evitará o vazamento de informação do futuro para o passado e garantirá consistência estatística. A transformação e a inversão para o alvo serão implementadas conforme as Equações (\ref{eq:minmax}) e (\ref{eq:invert_scale}).

\begin{equation}
x' = \frac{x - \min\limits_{\mathcal{D}_{\mathrm{train}}}(x)}{\max\limits_{\mathcal{D}_{\mathrm{train}}}(x) - \min\limits_{\mathcal{D}_{\mathrm{train}}}(x)}
\label{eq:minmax}
\end{equation}

\begin{equation}
\hat{y} = \hat{y}' \cdot \left(\max\limits_{\mathcal{D}_{\mathrm{train}}}(y) - \min\limits_{\mathcal{D}_{\mathrm{train}}}(y)\right) + \min\limits_{\mathcal{D}_{\mathrm{train}}}(y)
\label{eq:invert_scale}
\end{equation}

\noindent
Em que $x$ será o valor original da variável e $x'$ sua versão escalonada; $\min\limits_{\mathcal{D}_{\mathrm{train}}}(x)$ e $\max\limits_{\mathcal{D}_{\mathrm{train}}}(x)$ serão, respectivamente, o valor mínimo e máximo da variável no conjunto de treino; $\hat{y}'$ será a previsão no domínio escalonado; e $\hat{y}$ será a previsão revertida para o domínio físico da irradiância.

Esse procedimento será aplicado a todas as variáveis presentes nos conjuntos de treino, validação e teste, assegurando comparabilidade entre escalas distintas e estabilidade numérica no processo de treinamento.

Para a organização dos dados em lotes durante o treinamento dos modelos sequenciais (como a LSTM), será empregado o \texttt{DataLoader} da biblioteca \texttt{PyTorch}, o qual receberá como entrada tensores normalizados e produzirá lotes no formato $\{ \text{features}, \text{input}, \text{output} \}$, conforme definido no processo de janelação. Isso garantirá eficiência computacional e consistência na passagem dos dados durante o ciclo de treinamento, validação e teste.

\subsection{Janelação supervisionada}
A série multivariada será transformada em amostras supervisionadas definindo-se \textbf{janela de entrada} (comprimento $L$), \textbf{janela de saída} (horizonte $H$) e \textbf{passo} (\textit{stride} $s$). Cada amostra $n$ será construída conforme a Equação (\ref{eq:janela_superv}):
\begin{equation}
\mathbf{X}^{(n)} = \big[\, \mathbf{x}_{t-L+1}, \dots, \mathbf{x}_{t} \,\big] \in \mathbb{R}^{L \times p}, 
\quad 
\mathbf{y}^{(n)} = \big[\, y_{t+1}, \dots, y_{t+H} \,\big] \in \mathbb{R}^{H}
\label{eq:janela_superv}
\end{equation}

\noindent
Em que $\mathbf{x}_t \in \mathbb{R}^{p}$ conterá variáveis meteorológicas, sazonais (seno/cosseno) e resíduos/decomposições quando aplicável; $\mathbf{X}^{(n)} \in \mathbb{R}^{L \times p}$ agregará $L$ instantes passados; e $\mathbf{y}^{(n)} \in \mathbb{R}^{H}$ conterá $H$ passos futuros da irradiância.

O índice $t$ avançará pelo tempo com passo $s$ (típico $s=1$), gerando amostras sobrepostas (maior cobertura) ou não (menor correlação entre amostras), de acordo com o desenho experimental. Espera-se que $L$ capture memória suficiente e que $H$ reflita os horizontes de interesse operacional.

\begin{itemize}
    \item \textbf{Redes recorrentes/convolucionais}: lotes com dimensão $B$ resultarão em tensores $\mathbf{X}\_{\mathrm{batch}} \in \mathbb{R}^{B \times L \times p}$ (LSTM/Attention) ou $\mathbb{R}^{B \times p \times L}$ (CNN-1D). Saídas multi-etapas produzirão $\hat{\mathbf{y}} \in \mathbb{R}^{B \times H}$.
    \item \textbf{XGBoost/árvores}: a janela será \textit{vetorizada} em $\mathbf{u}^{(n)} \in \mathbb{R}^{L \cdot p}$ ou expandida em estatísticas (médias, mínimos, máximos, últimas $k$ observações). O treinamento poderá ser \textit{multi-output} direto (um modelo para $H$ saídas) ou \textit{direct per horizon} (um modelo por $h\in\{1,\dots,H\}$).
\end{itemize}

\subsection{Estratégia multi-etapas}
Será adotado o \textbf{regime multi-saída direta}, no qual um único mapeamento aprenderá simultaneamente todos os horizontes $\{1,\dots,H\}$, reduzindo a acumulação de erro recursivo. A função de perda agregará os erros por horizonte, conforme a Equação (\ref{eq:loss_multioutput}):
\begin{equation}
\mathcal{L} = \frac{1}{B\,H} \sum_{b=1}^{B} \sum_{h=1}^{H} \big(\hat{y}_{b,h} - y_{b,h}\big)^2
\label{eq:loss_multioutput}
\end{equation}

\noindent
Em que $B$ será o tamanho do lote; $H$ será o horizonte de previsão; e $\hat{y}_{b,h}$ e $y_{b,h}$ serão, respectivamente, previsão e valor verdadeiro no passo $h$ do lote $b$.

Espera-se melhor consistência entre horizontes e treinamento mais estável para $H$ moderado.

\subsection{Reconstrução temporal das previsões}
As saídas $\hat{\mathbf{y}}^{(n)}$ serão reindexadas nas datas futuras correspondentes $\{t+1,\dots,t+H\}$ e reinseridas no \textit{dataframe} original, preservando granularidade e calendário. Esse espalhamento permitirá inspeção visual contínua e cálculo de métricas por horizonte.

\subsection{Infraestrutura de Rastreamento Experimental (MLflow)}

Será adotado o \texttt{MLflow} para rastrear parâmetros, métricas, artefatos e versões de modelos (e.g., LSTM, ARIMA--LSTM, Attention, RSTL, XGBoost). A motivação será garantir reprodutibilidade, auditoria e comparabilidade sistemática.

A instrumentação incluirá:
\begin{itemize}
    \item definição de experimento e de execuções com \texttt{mlflow.set\_experiment} e \texttt{mlflow.start\_run};
    \item \textit{autolog} quando aplicável;
    \item \textbf{parâmetros logados}: $L$ (input window), $H$ (output window), $s$ (stride), conjunto de \textit{features}, escalonadores, arquiteturas, camadas, neurônios, \textit{dropout}, função de ativação, taxa de aprendizado, épocas, lote $B$, hiperparâmetros de árvores (estimadores, profundidade, taxa de aprendizado do \textit{boosting});
    \item \textbf{artefatos}: curvas treino/validação, diagnósticos por horizonte, gráficos real vs. previsto, amostras de janelas, \textit{checkpoints} de modelo e serialização do pré-processamento;
    \item \textbf{assinatura} do modelo (\textit{input schema}/\textit{output schema}) e exemplo canônico de entrada para servir/inferência.
\end{itemize}
\subsection{Avaliação de erros nas previsões}

Serão registradas as métricas \textbf{RMSE} e \textbf{$R^2$}, tanto em valores globais quanto desagregadas por horizonte. As definições constarão nas Equações (\ref{eq:rmse}) e (\ref{eq:r2}); as variantes por horizonte considerarão o subconjunto $\{(b,h)\,|\,h=\bar{h}\}$.

\begin{equation}
\mathrm{RMSE} = \sqrt{\frac{1}{N}\sum_{i=1}^{N}\big(\hat{y}_i - y_i\big)^2}
\label{eq:rmse}
\end{equation}

\begin{equation}
R^2 = 1 - \frac{\sum_{i=1}^{N} \big(y_i - \hat{y}_i\big)^2}{\sum_{i=1}^{N} \big(y_i - \overline{y}\big)^2}
\label{eq:r2}
\end{equation}

\noindent
Em que $N$ será o número de previsões agregadas (global ou por horizonte); $y_i$ e $\hat{y}_i$ serão, respectivamente, observação e previsão no instante $i$; e $\overline{y}$ será a média das observações no período considerado.

Essas métricas permitirão quantificar a precisão das previsões (RMSE) e a proporção da variabilidade explicada pelo modelo ($R^2$), garantindo rastreabilidade de desempenho por horizonte e comparações fiéis entre diferentes arquiteturas e configurações.

\subsection{Boas práticas de reprodutibilidade}
Serão adotados \textit{seeds} determinísticos (\texttt{numpy}, e \texttt{PyTorch} quando aplicável), registro de versões de bibliotecas e armazenamento do caminho/versão do conjunto de dados como artefato. Serão utilizados \textbf{tags} descritivos (dataset, $L$, $H$, \textit{features}, escalonador, arquitetura) para facilitar a consulta. Espera-se possibilitar a repetição exata das execuções e auditoria metodológica.

\section{LSTM}

A implementação dos modelos em \texttt{PyTorch} será organizada em três blocos principais: (i) estrutura de dados e tensores, (ii) modelo LSTM básico e (iii) função orquestradora \texttt{train\_lstm\_forecast}, a qual permitirá treinar não apenas a LSTM, mas também variantes como arquiteturas com atenção.

\subsection{Estrutura de dados e tensores}

Será criada a classe \texttt{TimeSeriesDataset}, herdando de \texttt{torch.utils.data.Dataset}, a qual organizará os dados supervisionados em pares de entrada e saída. A classe converterá as matrizes de janelas $(X,y)$ para tensores \texttt{float32}, garantindo consistência no treinamento. 

Cada lote do \texttt{DataLoader} será estruturado conforme a Equação (\ref{eq:lote_dataloader}).
\begin{equation}
\mathbf{X}_{\mathrm{batch}} \in \mathbb{R}^{B \times L \times p}, \quad \mathbf{y}_{\mathrm{batch}} \in \mathbb{R}^{B \times H}
\label{eq:lote_dataloader}
\end{equation}

\noindent
Em que $B$ será o tamanho do \textit{batch}; $L$ será o tamanho da janela de entrada (\textit{input window}); $p$ será o número de variáveis preditoras (\textit{features}); e $H$ será o horizonte de previsão (\textit{output window}). Essa organização materializará explicitamente o triplo \{features, input window, output window\}, fundamental para a formulação supervisionada.

\subsection{Modelo LSTM}

O modelo \texttt{LSTMForecast} será implementado como uma subclasse de \texttt{torch.nn.Module}, recebendo explicitamente os hiperparâmetros de arquitetura e função de ativação. Sua construção compreenderá duas camadas principais: (i) a rede recorrente LSTM e (ii) uma camada totalmente conectada para mapear o estado oculto para as previsões.

Na inicialização, a classe receberá os argumentos: \texttt{input\_size} (número de \textit{features} em cada passo temporal), \texttt{hidden\_size} (dimensão do estado oculto), \texttt{num\_layers} (número de camadas empilhadas), \texttt{output\_size} (número de passos futuros a prever), \texttt{dropout} (taxa de desativação) e \texttt{activation} (função de ativação final). 

A rede recorrente será instanciada conforme Equação (\ref{eq:lstm_instancia}).
\begin{equation}
\mathrm{LSTM}(p, h, L_r, \texttt{batch\_first=True}, d)
\label{eq:lstm_instancia}
\end{equation}

\noindent
Em que $p$ será o número de \textit{features} (\texttt{input\_size}); $h$ será o tamanho do estado oculto (\texttt{hidden\_size}); $L_r$ será o número de camadas empilhadas (\texttt{num\_layers}); e $d$ será a taxa de dropout aplicada entre camadas.

Após a camada recorrente, será adicionada uma camada linear $\mathbf{W}\in\mathbb{R}^{H\times h}$, $\mathbf{b}\in\mathbb{R}^{H}$, responsável por projetar o estado oculto final em $H$ passos de previsão. Opcionalmente, aplicar-se-á uma função de ativação definida pelo usuário:
\[
\phi \in \{\mathrm{ReLU}, \tanh, \sigma, \mathrm{Identity}\}.
\]

A passagem direta do modelo será descrita pela Equação (\ref{eq:lstm_forward_full}):
\begin{equation}
\begin{aligned}
\mathbf{H}, (\mathbf{h}_t,\mathbf{c}_t) &= \mathrm{LSTM}(\mathbf{X}) \\
\mathbf{z} &= \mathbf{H}_{[:, -1, :]} \\
\widehat{\mathbf{y}} &= \phi(\mathbf{W}\mathbf{z} + \mathbf{b}) \in \mathbb{R}^{H}
\end{aligned}
\label{eq:lstm_forward_full}
\end{equation}

\noindent
Em que $\mathbf{X} \in \mathbb{R}^{B \times L \times p}$ será o lote de entradas; $\mathbf{H} \in \mathbb{R}^{B \times L \times h}$ será a sequência de estados ocultos retornados pela LSTM; $\mathbf{H}_{[:, -1, :]}$ selecionará o último estado oculto da sequência; $\mathbf{z} \in \mathbb{R}^{B \times h}$ será o vetor oculto condensado; e $\widehat{\mathbf{y}} \in \mathbb{R}^{B \times H}$ será o vetor final de previsões, já no domínio escalonado.

A implementação seguirá a estratégia \textit{multi-output direto}, em que todo o horizonte de previsão será produzido de uma única vez a partir do último estado oculto. Isso evitará a acumulação de erro típica de abordagens recursivas e garantirá consistência no alinhamento temporal das saídas.

O uso da função de ativação final $\phi(\cdot)$ será configurável: \texttt{ReLU}, $\tanh$ e $\sigma$ poderão impor restrições de faixa aos valores previstos; enquanto \texttt{Identity} manterá a saída linear, adequada quando as previsões forem revertidas da escala \textit{min--max} para o domínio físico (irradiância em W/m²).

\subsection{Função orquestradora \texttt{train\_lstm\_forecast}}

Será desenvolvida a função \texttt{train\_lstm\_forecast}, responsável por encapsular todo o pipeline de treinamento e avaliação. Essa função realizará:

\begin{enumerate}
    \item Fixação da semente aleatória (\texttt{NumPy}, \texttt{PyTorch}) para reprodutibilidade.
    \item Criação das janelas supervisionadas $(X,y)$ e normalização \textit{min--max}, com ajuste feito apenas sobre o conjunto de treino.
    \item Separação temporal em treino, validação e teste com base em anos específicos.
    \item Criação de \texttt{DataLoader}s otimizados para GPU (\texttt{pin\_memory}, \texttt{non\_blocking}).
    \item Definição da função de perda (\texttt{MSELoss}), otimizador configurável (Adam, SGD, RMSProp, AdamW, etc.) e \textit{weight decay} para regularização.
    \item Treinamento com \textit{mixed precision} (AMP) e critério de \textit{early stopping} baseado na perda de validação.
    \item Avaliação no conjunto de teste, inversão da escala para domínio físico (W/m²) e cálculo de métricas de desempenho.
    \item Espalhamento das previsões $\widehat{\mathbf{y}}$ no índice temporal original, assegurando consistência no horizonte de 15 minutos.
    \item Registro completo do experimento no \texttt{MLflow} (parâmetros, métricas, modelo e artefatos).
\end{enumerate}


\subsection{Parâmetros controlados}

Os principais parâmetros aceitos pela função serão apresentados na Tabela \ref{tab:parametros_treino}.

\begin{table}[H]
    \centering
    \caption{Principais parâmetros da função \texttt{train\_lstm\_forecast}.}
    \resizebox{0.95\textwidth}{!}{%
    \begin{tabular}{|l|p{10cm}|}
        \hline
        \textbf{Parâmetro} & \textbf{Descrição} \\
        \hline
        \texttt{input\_window} ($L$) & Comprimento da janela de entrada (nº de passos históricos). \\
        \hline
        \texttt{output\_window} ($H$) & Horizonte de previsão (nº de passos futuros, aqui 15 min $\times$ 64 para 1 dia). \\
        \hline
        \texttt{features} & Lista de variáveis preditoras utilizadas. \\
        \hline
        \texttt{target} & Variável alvo (irradiância global). \\
        \hline
        \texttt{n\_layers} & Número de camadas LSTM empilhadas. \\
        \hline
        \texttt{hidden\_size} & Dimensão do estado oculto em cada célula LSTM. \\
        \hline
        \texttt{dropout} & Taxa de desativação de neurônios para regularização. \\
        \hline
        \texttt{batch\_size} & Tamanho do mini-batch no treinamento. \\
        \hline
        \texttt{learning\_rate} & Taxa de aprendizado do otimizador. \\
        \hline
        \texttt{optimizer\_name} & Algoritmo de otimização escolhido (Adam, SGD, RMSProp, etc.). \\
        \hline
        \texttt{early\_stopping\_patience} & Nº de épocas sem melhoria para ativar parada antecipada. \\
        \hline
        \texttt{activation\_function} & Função de ativação na camada final (ReLU, Tanh, Sigmoid, Linear). \\
        \hline
        \texttt{weight\_decay} & Penalização L2 nos pesos para regularização. \\
        \hline
        \texttt{mlflow\_experiment\_name} & Nome do experimento no \texttt{MLflow}. \\
        \hline
    \end{tabular}
    }
    \par\small{Fonte: Autor (2025)}
    \label{tab:parametros_treino}
\end{table}



\subsection{Métricas de avaliação}

Após o treinamento, as previsões serão avaliadas em domínio físico (W/m²) por meio de RMSE e $R^2$, definidos nas Equações (\ref{eq:rmse}) e (\ref{eq:r2}). Além do cálculo global, as métricas também serão computadas por horizonte e em janelas diurnas (7h--19h), de modo a focar no período de maior relevância energética.

\section{XGBoost}
\subsection{Janelamento da série}

Para a modelagem via XGBoost será adotado o esquema de janelas deslizantes, sem qualquer normalização de variáveis. Considere-se a série alvo $y_t$ (irradiância global) e o vetor de preditores $\mathbf{x}_t \in \mathbb{R}^{p}$ em frequência de 15 minutos. Serão definidas uma janela de entrada de comprimento $L$ e um horizonte de previsão de comprimento $H$, ambos em passos de 15 minutos. A construção dos pares entrada--saída para cada instante $t$ será dada pelas Equações (\ref{eq:x_window})--(\ref{eq:dataset_direct}).

\begin{equation}
\mathbf{X}_t = \big[\, \mathbf{x}_{t-L+1}, \mathbf{x}_{t-L+2}, \ldots, \mathbf{x}_{t} \,\big] \in \mathbb{R}^{L \times p}
\label{eq:x_window}
\end{equation}

\noindent
Em que $\mathbf{x}_{t}$ será o vetor de $p$ preditores no instante $t$, $L$ será o número de passos históricos considerados na entrada e $\mathbf{X}_t$ empilhará, por tempo, os $L$ vetores de preditores mais recentes.

\begin{equation}
\mathbf{y}_t = \big[\, y_{t+1}, y_{t+2}, \ldots, y_{t+H} \,\big] \in \mathbb{R}^{H}
\label{eq:y_window}
\end{equation}

\noindent
Em que $y_{t+h}$ será a irradiância alvo no horizonte $h \in \{1,\ldots,H\}$ e $H$ será o número de passos futuros a serem previstos (aqui, $H=64$, equivalente a 16 horas).

Como o XGBoost opera sobre vetores em $\mathbb{R}^d$, será aplicado o operador de \textit{flatten} (vetorização) à janela de entrada, conforme a Equação (\ref{eq:flatten}). Ressalta-se que não será realizada normalização nem padronização; as variáveis permanecerão em suas unidades originais.
\begin{equation}
\mathbf{z}_t = \operatorname{vec}(\mathbf{X}_t) \in \mathbb{R}^{L \cdot p}
\label{eq:flatten}
\end{equation}

\noindent
Em que $\operatorname{vec}(\cdot)$ concatenará linha a linha (ou por tempo) os elementos de $\mathbf{X}_t$, resultando em um vetor de dimensão $L \cdot p$, e $\mathbf{z}_t$ será a representação vetorial final fornecida ao XGBoost.

O conjunto de dados para aprendizado direto multi-passo será então dado por:
\begin{equation}
\mathcal{D} \;=\; \big\{\, (\mathbf{z}_t, \mathbf{y}_t) \;\big|\; t = L, \ldots, T-H \,\big\}.
\label{eq:dataset_direct}
\end{equation}

\noindent
Em que $T$ será o último índice temporal disponível na base e $\mathcal{D}$ conterá todos os pares entrada--saída válidos respeitando a causalidade temporal.

\subsection{Modelagem XGBoost direta com $H$ modelos (um por passo)}

Será adotada a estratégia direta (\textit{direct multi-step forecasting}), na qual serão treinados $H$ modelos independentes $\{f_1,\ldots,f_H\}$, cada qual mapeando $\mathbf{z}_t \mapsto y_{t+h}$. Para cada horizonte $i \in \{1,\ldots,H\}$, o problema empírico otimizado pelo XGBoost será descrito pela Equação (\ref{eq:xgb_obj}), com regularização estrutural sobre o conjunto de árvores $\mathcal{F}_{\mathrm{XGB}}$.
\begin{equation}
\widehat{f}_i \;=\; \arg\min_{f \in \mathcal{F}_{\mathrm{XGB}}} 
\;\sum_{(\mathbf{z}, y_i) \in \mathcal{D}} \big(y_i - f(\mathbf{z})\big)^2 \;+\; \Omega(f)
\label{eq:xgb_obj}
\end{equation}

\noindent
Em que $y_i$ denotará o elemento $i$ de $\mathbf{y}_t$ (isto é, $y_{t+i}$), $\mathcal{F}_{\mathrm{XGB}}$ será a classe de funções implementada por \textit{gradient boosting} sobre árvores de decisão e $\Omega(f)$ penalizará a complexidade do conjunto de árvores (número de folhas, profundidade, pesos), controlando viés--variância.

O vetor de previsões para uma entrada $\mathbf{z}_t$ será composto pela concatenação das saídas individuais conforme a Equação (\ref{eq:aggregate_preds}), com imposição de não negatividade para refletir a natureza física da irradiância conforme a Equação (\ref{eq:nonneg_clip}).
\begin{equation}
\widehat{\mathbf{y}}_t \;=\; 
\big[\, \widehat{f}_1(\mathbf{z}_t), \widehat{f}_2(\mathbf{z}_t), \ldots, \widehat{f}_H(\mathbf{z}_t) \,\big]
\label{eq:aggregate_preds}
\end{equation}

\noindent
Em que $\widehat{\mathbf{y}}_t$ conterá as $H$ previsões futuras em 15 minutos cada.

\begin{equation}
\widehat{f}_i^{+}(\mathbf{z}_t) \;=\; \max\big\{\,0,\, \widehat{f}_i(\mathbf{z}_t)\,\big\}
\label{eq:nonneg_clip}
\end{equation}

\noindent
Em que $\widehat{f}_i^{+}$ será a versão \textit{clipped} da predição, restringindo valores negativos.

\subsection{Espalhamento temporal e avaliação por horizonte}

Após a predição em janelas de teste, será realizado o espalhamento das previsões no eixo temporal, de modo que cada $\widehat{f}_i^{+}(\mathbf{z}_t)$ seja alocada no timestamp $t+i$. Em instantes $\tau$ com sobreposição de janelas, será considerada a média das contribuições conforme a Equação \ref{eq:spread_mean} e o desvio-padrão associado conforme a Equação \ref{eq:spread_std}, permitindo análise por horizonte e agregada.
\begin{equation}
\widehat{y}_{\tau} \;=\; \frac{1}{\lvert \mathcal{T}(\tau) \rvert}
\sum_{(t,i)\,:\,t+i=\tau} \widehat{f}_i^{+}(\mathbf{z}_t)
\label{eq:spread_mean}
\end{equation}

\noindent
Em que $\mathcal{T}(\tau)$ será o conjunto de pares $(t,i)$ cujas previsões incidirem em $\tau$ e $\widehat{y}_{\tau}$ será a predição agregada no instante $\tau$.

\begin{equation}
\widehat{\sigma}_{\tau} \;=\; 
\sqrt{\,\frac{1}{\lvert \mathcal{T}(\tau) \rvert - 1}\sum_{(t,i)\,:\,t+i=\tau}
\Big(\widehat{f}_i^{+}(\mathbf{z}_t) - \widehat{y}_{\tau}\Big)^2\,}
\label{eq:spread_std}
\end{equation}

\noindent
Em que $\widehat{\sigma}_{\tau}$ quantificará a incerteza empírica por sobreposição de janelas no instante $\tau$.

Para avaliação global multi-saída, será utilizada a raiz do erro quadrático médio (RMSE) agregando instantes e horizontes, conforme Equação (\ref{eq:rmse_multi}).
\begin{equation}
\mathrm{RMSE} \;=\; \sqrt{\frac{1}{N \cdot H}\sum_{t}\sum_{i=1}^{H}
\Big(\widehat{f}_i^{+}(\mathbf{z}_t) - y_{t+i}\Big)^2}
\label{eq:rmse_multi}
\end{equation}

\noindent
Em que $N$ será o número de janelas no conjunto considerado (treino, validação ou teste) e $H$ será o horizonte de previsão (aqui, $H=64$).

A opção por esta métrica global se justificará pela necessidade de comparabilidade direta com outros modelos de previsão multi-passo, como a rede LSTM apresentada anteriormente. No caso da LSTM, a avaliação também considerará janelas deslizantes de entrada e agregará os erros em todos os horizontes previstos. Dessa forma, a definição de RMSE global na Equação (\ref{eq:rmse_multi}) garantirá que ambos os modelos sejam avaliados sob o mesmo critério de desempenho, assegurando uma comparação justa e consistente entre abordagens de natureza distinta.

\subsection{Treinamento, validação e empacotamento em MLflow}

O treinamento será realizado de forma sequencial por horizonte, com \textit{early stopping} e \textit{watchlist} de treino/validação interno ao XGBoost. Será considerado, para cada $i \in \{1,\ldots,H\}$, um \texttt{DMatrix} com as colunas alvo $y_{t+i}$ e as entradas $\mathbf{z}_t$. Será adotado \texttt{num\_boost\_round} elevado, aliado a \texttt{early\_stopping\_rounds}, de modo a permitir convergência estável sem sobreajuste. Como salvaguarda física, será aplicado o truncamento não negativo em todas as saídas conforme visto na Equação (\ref{eq:nonneg_clip}).

Com vistas à operacionalização, os $H$ modelos resultantes serão incorporados a uma única classe Python do tipo \texttt{mlflow.pyfunc.PythonModel}. Tal classe: (i) carregará os artefatos \texttt{model\_i.json} ($i=1,\ldots,H$) e metadados de \textit{features} utilizadas; (ii) ao receber um lote de entradas, organizará as \textit{features} conforme o treino, aplicará todos os $\widehat{f}_i$ e concatenará as saídas conforme Equação (\ref{eq:aggregate_preds}); e (iii) devolverá $\widehat{\mathbf{y}}$ com truncamento não negativo. O artefato final será registrado no MLflow como um único modelo. Assim, a inferência será chamada de forma direta e transparente, embora internamente sejam executados os $H$ preditores especializados. Esse empacotamento facilitará a reprodutibilidade, a comparação entre execuções e a integração com pipelines.

\subsection{Função principal de treinamento/validação}

Será implementada uma função principal de treino/validação por saída, responsável por treinar o modelo do horizonte $i$, avaliar RMSE e $R^2$ em treino, validação e teste, e registrar parâmetros, métricas e artefatos no MLflow. A função receberá, entre outros, as janelas \texttt{X\_train}, \texttt{X\_val}, \texttt{X\_test}, os respectivos alvos colunares (\texttt{y\_train[:, i]}, \texttt{y\_val[:, i]}, \texttt{y\_test[:, i]}), e o dicionário de hiperparâmetros do XGBoost, conforme a Tabela \ref{tab:param_xgb_train}. Em termos conceituais, sua ação corresponderá à minimização da Equação \ref{eq:xgb_obj} e à composição final da Equação (\ref{eq:aggregate_preds}), com salvaguarda de não negatividade conforme a Equação (\ref{eq:nonneg_clip}).

\begin{table}[!h]
    \centering
    \caption{Principais parâmetros da função \texttt{treinar\_modelo\_saida\_booster}.}
    \resizebox{0.95\textwidth}{!}{%
    \begin{tabular}{|l|p{10cm}|}
        \hline
        \textbf{Parâmetro} & \textbf{Descrição} \\
        \hline
        \texttt{input\_window} ($L$) & Comprimento da janela de entrada (nº de passos históricos). \\
        \hline
        \texttt{output\_window} ($H$) & Horizonte de previsão (nº de passos futuros; aqui $H=64$). \\
        \hline
        \texttt{features} & Lista e ordem das variáveis preditoras utilizadas na vetorização (\textit{flatten}). \\
        \hline
        \texttt{target} & Variável alvo (irradiância global $y_t$). \\
        \hline
        \texttt{stride} ($s$) & Passo de deslizamento entre janelas sucessivas. \\
        \hline
        \texttt{flatten\_X} & Indicador de vetorização das janelas ($\mathbf{X}_t \mapsto \mathbf{z}_t$); necessário para XGBoost. \\
        \hline
        \texttt{params} & Hiperparâmetros do XGBoost (p.\,ex., \texttt{learning\_rate}, \texttt{max\_depth}, \texttt{min\_child\_weight}, \texttt{gamma}, \texttt{lambda}, \texttt{alpha}, \texttt{tree\_method}, \texttt{device}, \texttt{eval\_metric}). \\
        \hline
        \texttt{num\_boost\_round} & Número máximo de iterações de boosting. \\
        \hline
        \texttt{early\_stopping\_rounds} & Critério de parada antecipada com base no conjunto de validação. \\
        \hline
        \texttt{mlflow\_experiment\_name} & Nome do experimento no MLflow para rastreamento. \\
        \hline
        \texttt{seed} & Semente para reprodutibilidade. \\
        \hline
        \texttt{clip\_non\_negative} & Aplicação de truncamento não negativo às predições. \\
        \hline
    \end{tabular}
    }
    \par\small{Fonte: Autor (2025)}
    \label{tab:param_xgb_train}
\end{table}

Optar-se-á por não normalizar as variáveis por três razões práticas: (i) modelos de árvores serão pouco sensíveis à escala, (ii) preservar-se-á a interpretabilidade física em unidades originais ao longo do pipeline e (iii) evitar-se-á potencial viés de \textit{leakage} na aplicação de transformações dependentes do conjunto.

A generalidade da função de treinamento por horizonte, parametrizada por $L$, $H$, \texttt{features}, \texttt{stride} e \texttt{params}, permitirá replicar facilmente: (a) otimizações de hiperparâmetros (substituindo-se \texttt{params} por amostras geradas por um otimizador externo); (b) estudos de sensibilidade (p.\,ex., variar \texttt{max\_depth}/\texttt{min\_child\_weight}); (c) testes de ablação de variáveis (incluir/excluir subconjuntos de \texttt{features}); e (d) mudanças de $L$ e $H$ sem alterações estruturais no código. O empacotamento em uma classe \texttt{pyfunc} no MLflow garantirá que, mesmo com $H$ modelos internos, a interface de predição permaneça unificada, facilitando reuso e implantação.

\section{Decomposição robusta RSTL como variável auxiliar}

Além da construção de variáveis derivadas baseadas em diferenças e médias móveis, será considerada a decomposição robusta da série temporal, conhecida como RSTL (\textit{Robust Seasonal-Trend decomposition using LOESS}). Tal abordagem não será empregada aqui como modelo de previsão autônomo, mas sim como ferramenta de extração de componentes estruturais da irradiância, a fim de fornecer variáveis auxiliares que possam enriquecer os modelos principais de previsão.

A decomposição será expressa pela Equação (\ref{eq:rstl_decomp}), em que a série de irradiância global $y_t$ será representada como a soma de três termos aditivos.
\begin{equation}
y_t \;=\; T_t + S_t + R_t
\label{eq:rstl_decomp}
\end{equation}

\noindent
Em que $T_t$ será a componente de tendência, capturando variações de longo prazo; $S_t$ será a componente sazonal, representando padrões repetitivos em períodos fixos; e $R_t$ será o resíduo, contendo a parcela não explicada pelas duas componentes anteriores.

Para a decomposição será utilizada a implementação \texttt{STL} da biblioteca \texttt{statsmodels}, ajustada em modo robusto (\texttt{robust=True}) para reduzir a influência de outliers e assegurar maior estabilidade estatística. O período da decomposição será definido como $P=64$, correspondente a 16 horas na base em frequência de 15 minutos. Os parâmetros de suavização serão configurados de forma a permitir uma separação clara entre tendência e sazonalidade: uma janela de 155 amostras para a componente sazonal e uma janela de 255 amostras para a componente de tendência.

As três componentes resultantes ($T_t$, $S_t$ e $R_t$) serão incorporadas ao conjunto de dados como novas variáveis auxiliares:
\begin{equation}
\mathbf{x}_t^{\,\text{RSTL}} = \big[T_t,\, S_t,\, R_t\big]
\label{eq:rstl_features}
\end{equation}

A reconstrução da série pela soma das componentes (\ref{eq:rstl_decomp}) será comparada com a série original, obtendo-se erro médio absoluto (MAE) e raiz do erro quadrático médio (RMSE) de baixa magnitude, confirmando a consistência da decomposição. Esses resultados atestarão que a série poderá ser decomposta de maneira fiel em suas estruturas fundamentais.

A justificativa para essa escolha reside no fato de que modelos de aprendizado de máquina, como o XGBoost e as redes LSTM, poderão se beneficiar da separação explícita entre tendência, sazonalidade e resíduo. O fornecimento dessas componentes como variáveis auxiliares tenderá a facilitar a identificação de padrões e a reduzir a complexidade da função de predição a ser aprendida, uma vez que parte da variabilidade já será explicitamente decomposta. Além disso, esse procedimento será facilmente replicável para outros horizontes de previsão e para diferentes configurações de modelos, preservando a generalidade do pipeline.

Em resumo, o RSTL será utilizado como um mecanismo de pré-processamento auxiliar, destinado a fornecer informações estruturais adicionais ao modelo principal, sem que ele próprio seja considerado um preditor final da irradiância.

\section{Otimização de hiperparâmetros}

A seleção adequada de hiperparâmetros é essencial para garantir a capacidade de generalização dos modelos de previsão. Esse processo poderá ser formulado como o problema de otimização apresentado na Equação (\ref{eq:hyperopt}), no qual se busca a configuração que minimiza a função de perda avaliada sobre o conjunto de validação.

\begin{equation}
\theta^{*} = \arg\min_{\theta \in \Theta} \; \mathcal{L}(M(\theta); \mathcal{D}_{\text{val}})
\label{eq:hyperopt}
\end{equation}

\noindent
Em que $\theta$ representará o vetor de hiperparâmetros; $\Theta$ será o espaço de busca; $M(\theta)$ será o modelo parametrizado; $\mathcal{L}(\cdot)$ corresponderá à função de perda utilizada; e $\theta^{*}$ será a configuração ótima encontrada.

Nesta dissertação será adotada exclusivamente a \textbf{otimização Bayesiana}, devido à sua eficiência na exploração de espaços de busca contínuos e potencialmente de alta dimensionalidade. Diferentemente de métodos exaustivos como \textit{Grid Search}, a otimização Bayesiana evita avaliações redundantes em regiões sabidamente inferiores e concentra recursos computacionais em áreas promissoras do espaço de hiperparâmetros. Essa característica é especialmente relevante para modelos como LSTM e XGBoost, cujos hiperparâmetros interagem de forma não linear e apresentam superfícies de perda complexas.

A abordagem bayesiana modelará a função de perda $\mathcal{L}(\theta)$ como uma função aleatória, atualizada iterativamente após cada avaliação. Seja $\mathcal{M}_t$ o modelo substituto após $t$ iterações (por exemplo, um processo Gaussiano). A escolha do próximo ponto será guiada por uma \textit{função de aquisição} $a(\theta; \mathcal{M}_t)$, conforme a Equação (\ref{eq:bayesopt}), que balanceará exploração de regiões pouco testadas e exploração de regiões potencialmente ótimas.

\begin{equation}
\theta_{t+1} = \arg\max_{\theta \in \Theta} \; a(\theta; \mathcal{M}_t)
\label{eq:bayesopt}
\end{equation}

\noindent
Funções de aquisição clássicas incluem \textit{Expected Improvement}, \textit{Probability of Improvement} e \textit{Upper Confidence Bound}, todas capazes de direcionar a busca para regiões com maior probabilidade de redução da perda. O processo repete-se até atingir convergência, limite de iterações ou outro critério de parada definido.

A escolha pela otimização Bayesiana assegurará maior eficiência computacional, melhor cobertura do espaço de busca e menor risco de convergência para regiões subótimas, permitindo ajustar hiperparâmetros de forma robusta e consistente em todos os modelos avaliados nesta dissertação.




\chapter{Estudo de caso}\label{chap:resultados}
\section{Estudo de Caso: Estação BSRN de São Martinho da Serra (SMS)}

\subsection{Fonte dos dados, localização e instrumentação}

Foi utilizada a base pública do repositório PANGAEA referente à estação BSRN de São Martinho da Serra (SMS, código 70), a qual disponibiliza arquivos mensais categorizados como \textit{Basic measurements of radiation at station São Martinho da Serra (YYYY–MM)}. Todos os meses \textbf{disponíveis} foram baixados e \textbf{concatenados} em um único conjunto antes das análises, uma vez que a distribuição original é segmentada por mês. A estação está situada nas coordenadas $-29{,}44278^{\circ}$ (latitude), $-53{,}82305^{\circ}$ (longitude) e elevação de $489$~m. As variáveis principais empregadas foram a irradiância global de onda curta à superfície (\textit{Shortwave Downwelling Radiation} — SWD), a temperatura do ar a $2$~m, a umidade relativa e a pressão atmosférica (\cite{Pereira_2018_SMS_Basic_2016_10}~).

Após a concatenação de todos os meses, foram contabilizados \textbf{5.416.497} registros, de minuto em minuto. Considerando que as séries originais são fornecidas em UTC, realizou-se a conversão para o fuso horário local (UTC$-3$), de modo a alinhar fisicamente os horários de nascer e pôr do sol com a dinâmica intradiária observada. Essa decisão viabiliza a interpretação energética por hora do dia e reduz artefatos temporais.

A análise de completude evidenciou diferenças relevantes na disponibilidade de dados entre os anos. Verificou-se que 2008 (42,29\%), 2013 (25,64\%) e 2017 (55,17\%) apresentaram perdas expressivas de registros, enquanto outros anos, como 2010 (0,00\%) e 2014 (0,14\%), apresentaram séries quase completas. Essa caracterização inicial é essencial, pois permite identificar previamente os períodos mais adequados para modelagem, bem como justificar a exclusão de anos com grandes lacunas. O resumo anual é apresentado na Tabela~\ref{tab:faltantes_ano}.

\begin{table}[!h]
    \centering
    \caption{Resumo anual de dados faltantes.}
    \begin{tabular}{|c|r|r|r|}
        \hline
        \textbf{Ano} & \textbf{Minutos esperados} & \textbf{Minutos faltantes} & \textbf{\% faltante} \\
        \hline
        2006 & 525.600 & 129.601 & 24,66\% \\
        2007 & 525.600 & 1       & 0,00\% \\
        2008 & 527.040 & 222.875 & 42,29\% \\
        2009 & 525.600 & 480     & 0,09\% \\
        2010 & 525.600 & 0       & 0,00\% \\
        2011 & 525.600 & 1.414   & 0,27\% \\
        2012 & 527.040 & 78.789  & 14,95\% \\
        2013 & 525.600 & 134.777 & 25,64\% \\
        2014 & 525.600 & 726     & 0,14\% \\
        2015 & 525.600 & 16.740  & 3,18\% \\
        2016 & 527.040 & 19.657  & 3,73\% \\
        2017 & 525.600 & 289.963 & 55,17\% \\
        \hline
    \end{tabular}
    \par\small{Fonte: Autor (2025)}
    \label{tab:faltantes_ano}
\end{table}

Além da base BSRN, foram avaliadas outras fontes de dados. Considerou-se o uso do \textit{Helioclim}, que fornece estimativas de irradiância obtidas por satélite (\cite{Lefevre_2014_Helioclim}~), mas tais valores são simulados e não resultam de medições diretas em superfície, o que pode comprometer análises mais sensíveis à variabilidade local. Também foram testados os dados do Instituto Nacional de Meteorologia (INMET), que disponibiliza séries medidas em estações meteorológicas de superfície (\cite{INMET_2020}~), porém apenas em resolução horária, insuficiente para capturar a dinâmica minuto a minuto requerida neste estudo. 

Dessa forma, a escolha pela base BSRN justifica-se por três aspectos principais: 
(i) os dados são medidos diretamente por instrumentação em solo; 
(ii) apresentam resolução temporal de um minuto; 
(iii) possuem padronização internacional de qualidade e consistência. 
Essas características tornam a BSRN a opção mais adequada para o objetivo de previsão intradiária de irradiância solar.



\FloatBarrier


\subsection{Pré-processamento e variáveis derivadas}

Inicialmente, aplicou-se reamostragem para janelas de $15$ minutos por média, conforme Equação~(\ref{eq:reamostragem}), harmonizando a resolução temporal com os horizontes preditivos e suavizando flutuações de alta frequência. Em seguida, foi estimada a irradiância de céu limpo via modelo físico (Ineichen, conforme metodologia), a partir da qual definiu-se o índice de claridade $k_t^{*}$ pela Equação~(\ref{eq:ktstar}). O índice foi truncado para $[0,1]$, mitigando eventuais distorções instrumentais e preservando o significado físico de transmitância atmosférica efetiva.

Por fim, adotaram-se codificações sazonais contínuas para hora e mês com $\cos^2(\cdot)$, conforme Equação~(\ref{eq:sazonal_unificada}), evitando descontinuidades entre categorias e fornecendo preditores suaves com suporte físico, ilustrados na Figura~\ref{fig:sazonal_hora_mes}. As novas variáveis apresentaram curvas suaves com pico alinhado ao meio-dia (hora) e ao verão austral (mês). Tais perfis são consistentes com o regime de radiação local e oferecem variáveis contínuas que evitam saltos artificiais entre categorias. Na modelagem, espera-se que essas codificações atuem como \textit{priors} físicos fracos, facilitando o aprendizado de padrões periódicos diários e anuais.

\begin{figure}[!h]
    \centering
    \caption{Codificação sazonal contínua: hora (esq.) e mês (dir.).}
    \includegraphics[width=0.9\textwidth]{Figuras/hora e mes sazonal.png}
    \par\small{Fonte: Autor (2025)}
    \label{fig:sazonal_hora_mes}
\end{figure}

\FloatBarrier


\subsection{Energia intradiária e justificativa da janela 5h–20h}

A energia diária integrada sobre toda a série totalizou \textbf{$18{,}02$~MWh/m$^2$}. Esse valor foi obtido pela conversão da irradiância (potência por unidade de área, em W/m$^2$) em energia (Wh/m$^2$), considerando a soma das contribuições horárias. Na prática, o procedimento equivale à integral temporal da irradiância medida, ou seja, à multiplicação da potência média em cada intervalo de tempo pelo respectivo passo temporal (1~h), conforme ilustrado na Equação~(\ref{eq:energia_integrada}).

\begin{equation}
    E = \sum_{t=1}^{N} \overline{SWD}(t) \cdot \Delta t
    \label{eq:energia_integrada}
\end{equation}

\noindent Em que:
\begin{itemize}
    \item $E$ representa a energia integrada [Wh/m$^2$];
    \item $\overline{SWD}(t)$ é a irradiância média no intervalo $t$ [W/m$^2$];
    \item $\Delta t$ corresponde ao intervalo temporal de integração [h];
    \item $N$ é o número total de intervalos considerados no dia.
\end{itemize}

Não houve contribuição antes das 5h e após as 20h, conforme ilustra a Figura~\ref{fig:energia_por_hora}, de modo que a janela 5h–20h acarreta \textbf{perda nula} de energia. A distribuição por hora confirma o padrão diurno esperado, com máxima concentração entre 11h e 13h (11h: 13,17\%; 12h: 13,55\%; 13h: 12,92\%). Ao agregar por faixas, a janela 10h–14h responde por \textbf{52,3\%} da energia diária, enquanto 5h–9h e 15h–20h respondem por \textbf{23,1\%} e \textbf{24,6\%}, respectivamente. Esses achados sustentam a adoção da janela temporal para as etapas de análise e modelagem, priorizando o período energeticamente relevante.

\begin{figure}[!h]
    \centering
    \caption{Distribuição da energia solar diária por hora.}
    \includegraphics[width=0.9\textwidth]{Figuras/distribuição energia.png}
    \par\small{Fonte: Autor (2025)}
    \label{fig:energia_por_hora}
\end{figure}

\FloatBarrier


\subsection{Consistência, lacunas e interpolação}

Para as análises e modelagem, foi adotada a janela operacional de \textbf{5h–20h}, período no qual se concentra praticamente toda a energia diária. Nessa janela, para toda a base, eram esperados \textbf{3.928.606} registros; identificaram-se \textbf{3.612.410} observações presentes e \textbf{316.196} ausentes ($8{,}05\%$).

Com o intuito de preservar a grade temporal uniforme de 15 minutos, foram inicialmente inseridas linhas nos instantes faltantes. Em seguida, aplicou-se \textit{interpolação linear}, conforme Equação~(\ref{eq:interp_linear}) apenas para lacunas curtas, em que existiam valores válidos imediatamente antes e depois do \textit{gap}. Esse procedimento foi realizado para as variáveis SWD (9.436 pontos), temperatura (17.654), umidade relativa (11.351), pressão (5.084), $\mathrm{SWD}_{\mathrm{clear\_sky}}$ (4.803) e $k_t^{*}$ (4.803). 

No entanto, quando as lacunas ultrapassavam 30 minutos, optou-se pela remoção do \textbf{dia inteiro}. Essa escolha metodológica garante consistência intradiária, evitando descontinuidades temporais que poderiam ser interpretadas de forma incorreta pelos modelos preditivos. Ao todo, \textbf{1.691 dias} foram eliminados. Assim, o conjunto final manteve \textbf{2.305.920 amostras} com grade temporal regular e fisicamente coerente.

\begin{table}[H]
    \centering
    \caption{Tratamento de lacunas e consistência temporal (janela 5h–20h).}
    \resizebox{\columnwidth}{!}{%
    \begin{tabular}{|l|r|}
        \hline
        Registros esperados & 3.928.606 \\
        \hline
        Registros presentes & 3.612.410 \\
        \hline
        Faltantes (\%) & 316.196 (8{,}05\%) \\
        \hline
        Interpolados (SWD; Temp; RH; Pressure; $\mathrm{SWD}_{\mathrm{clear\_sky}}$; $k_t^{*}$)
        & 9.436; 17.654; 11.351; 5.084; 4.803; 4.803 \\
        \hline
        Dias removidos ($>$30 min de falha) & 1.691 \\
        \hline
        Percentuais interpolados & SWD: 0{,}16\%; Temp: 0{,}31\%; RH: 0{,}20\%; Pressure: 0{,}09\%; $\mathrm{SWD}_{\mathrm{clear\_sky}}$: 0{,}08\%; $k_t^{*}$: 0{,}08\% \\
        \hline
        Exclusão total & 39{,}71\% \\
        \hline
        Base final (5h–20h) & 2.305.920 \\
        \hline
    \end{tabular}
    }
    \par\small{Fonte: Autor (2025)}
    \label{tab:gaps_sms}
\end{table}



\subsection{Envelope físico: SWD vs. céu limpo}

Para avaliar o afastamento da série observada em relação ao limite físico local, foram comparadas as médias horárias de SWD e $\mathrm{SWD}_{\mathrm{clear\_sky}}$ para dois meses de referência (01/2015 e 02/2016). Conforme ilustrado na Figura~\ref{fig:swd_clearsky_2015_2016}, o céu limpo delineia um \textit{envelope} superior coerente com o máximo teórico próximo ao meio-dia solar, enquanto as séries observadas permanecem abaixo desse limite. Em 01/2015 observou-se maior proximidade entre as curvas ao redor do pico, ao passo que 02/2016 apresentou maior afastamento, compatível com atenuação por nebulosidade. Essa comparação fundamenta o uso de $k_t^{*}$ como preditor de transparência atmosférica e reforça a relevância de variáveis sazonais para capturar assimetrias intra-anuais.

\begin{figure}[!h]
    \centering
    \caption{Média horária de SWD e $\mathrm{SWD}_{\mathrm{clear\_sky}}$: 01/2015 vs. 02/2016.}
    \includegraphics[width=0.9\textwidth]{Figuras/SWD VS CLEAR.png}
    \par\small{Fonte: Autor (2025)}
    \label{fig:swd_clearsky_2015_2016}
\end{figure}



\subsection{Estatística descritiva e indicadores}

A análise estatística descritiva da base final está sintetizada na Tabela~\ref{tab:estat_desc_sms}. Observa-se que a irradiância global $\mathrm{SWD}$ apresenta média $\overline{\mathrm{SWD}}=293{,}09$~W/m$^2$ e desvio-padrão de $318{,}97$~W/m$^2$, enquanto a mediana é de apenas $170{,}13$~W/m$^2$. O valor mínimo encontrado foi $0$~W/m$^2$ (períodos noturnos), e o máximo $1305{,}27$~W/m$^2$. Essa combinação de média superior à mediana, aliada à assimetria entre valores mínimos e máximos, reflete a distribuição típica de séries de irradiância, em que longos períodos apresentam valores muito baixos e relativamente poucos instantes concentram valores elevados próximos ao pico solar.

Para a irradiância sob céu limpo $\mathrm{SWD}_{\mathrm{clear\_sky}}$, a mediana de $346{,}94$~W/m$^2$ e o percentil 95 de $1025{,}21$~W/m$^2$ delineiam o \textit{envelope} físico esperado em condições atmosféricas ideais. O índice de transmitância $k_t^{*}$ apresentou média $0{,}798$ e mediana próxima a $1{,}0$, indicando elevada frequência de condições próximas ao céu limpo, embora haja dispersão (dp $0{,}299$) que traduz a variabilidade atmosférica associada à presença de nuvens.

As variáveis meteorológicas também apresentam coerência física: a temperatura variou de $-2{,}93^{\circ}$C a $43{,}13^{\circ}$C, cobrindo desde eventos de inverno rigoroso até máximas de verão; a umidade relativa variou de aproximadamente $1{,}09\%$ até $101{,}94\%$, revelando tanto condições secas quanto situações de saturação; e a pressão atmosférica oscilou entre $943$ e $978$~hPa, intervalo compatível com a altitude da estação.  

Essa caracterização é relevante porque estabelece a amplitude e a dispersão das variáveis de entrada, permitindo:  
\begin{itemize}
    \item validar a coerência física dos dados disponíveis;  
    \item identificar a presença de assimetrias importantes que impactam a modelagem (particularmente a concentração de valores baixos em SWD);  
    \item fornecer referência para a segmentação de cenários de céu limpo e nebulosidade por meio de $k_t^{*}$, a ser utilizada na avaliação do desempenho dos modelos.  
\end{itemize}

\begin{table}[H]
    \centering
    \caption{Estatísticas descritivas (resumo da base final, 5h–20h).}
    \resizebox{0.9\textwidth}{!}{%
    \begin{tabular}{|l|r|r|r|r|r|r|r|}
        \hline
        \textbf{Variável} & \textbf{count} & \textbf{média} & \textbf{dp} & \textbf{mín} & \textbf{mediana} & $P_{95}$ & \textbf{máx} \\
        \hline
        SWD [W/m$^2$] & 153.728 & 293{,}09 & 318{,}97 & 0{,}00 & 170{,}13 & 932{,}80 & 1.305{,}27 \\
        \hline
        $\mathrm{SWD}_{\mathrm{clear\_sky}}$ [W/m$^2$] & 153.728 & 388{,}06 & 359{,}21 & 0{,}00 & 346{,}94 & 1.025{,}21 & 1.117{,}04 \\
        \hline
        $k_t^{*}$ [--] & 153.728 & 0{,}798 & 0{,}299 & 0{,}00 & 0{,}999 & 1{,}000 & 1{,}000 \\
        \hline
        Temp [$^{\circ}$C] & 153.728 & 18{,}60 & 6{,}20 & -2{,}93 & 18{,}77 & 28{,}55 & 43{,}13 \\
        \hline
        RH [\%] & 153.728 & 76{,}97 & 18{,}64 & 1{,}09 & 79{,}12 & 100{,}55 & 101{,}94 \\
        \hline
        Pressure [hPa] & 153.728 & 960{,}31 & 4{,}62 & 943{,}00 & 960{,}00 & 968{,}00 & 978{,}00 \\
        \hline
    \end{tabular}
    }
    \par\small{Fonte: Autor (2025)}
    \label{tab:estat_desc_sms}
\end{table}


\subsection{Síntese interpretativa}

Em síntese, a consolidação mensal dos arquivos da estação SMS (\cite{Pereira_2018_SMS_Basic_2016_10}~) resultou em um conjunto multivariado robusto para análise e modelagem de irradiância. O ajuste para UTC$-3$ e a adoção da janela 5h–20h foram determinantes para interpretar corretamente a distribuição intradiária de energia (perda nula fora da janela; pico concentrado em 11h–13h). A comparação SWD vs. céu limpo evidenciou o papel do $k_t^{*}$ como medida de transparência atmosférica, com médias elevadas e mediana próxima da unidade. O tratamento de lacunas combinou interpolação curta e exclusão de dias com falhas extensas, preservando a coerência física e temporal das séries. Por fim, as codificações sazonais contínuas fornecem preditores estruturados para capturar periodicidades diária e anual, cuja relevância será avaliada na seção de modelagem.

\section{Análise Exploratória de Dados (DEA)}

A presente seção discute os resultados da análise exploratória aplicada à série temporal de irradiância solar e variáveis meteorológicas associadas, considerando o período de 2006 a 2017. Diferentemente das etapas metodológicas, aqui são apresentadas as distribuições, correlações e padrões observados, bem como as implicações físicas e estatísticas para a modelagem.

\subsection{Distribuições univariadas com ênfase no suporte amostral}

As Figuras~\ref{fig:swd_ano} a \ref{fig:swd_temp} apresentam a média de $\mathrm{SWD}$ segmentada por variáveis temporais e meteorológicas, com o número de observações ($n_b$) indicado no topo de cada barra. Essa informação é fundamental para avaliar a robustez estatística das médias, distinguindo situações em que há grande volume de dados e, portanto, estimativas mais estáveis, daquelas em que o suporte amostral é limitado e as conclusões devem ser tratadas como exploratórias.

Na escala anual (Figura~\ref{fig:swd_ano}), verifica-se variabilidade entre os anos, com máximos em 2010 e 2011 (ambos com mais de $23{,}000$ observações) e mínimos em 2006 e 2013. Contudo, a interpretação para 2013 deve ser cautelosa, pois há apenas $256$ registros, o que confere alta incerteza. De modo similar, 2017 conta com pouco mais de $4{,}000$ observações, enquanto 2009, 2010, 2011 e 2016 apresentam bases mais completas, permitindo médias anuais mais confiáveis. Esse padrão mostra que aparentes oscilações interanuais muitas vezes refletem diferenças de completude dos dados, e não apenas variabilidade climática.

\begin{figure}[!h]
    \centering
    \caption{Média de $\mathrm{SWD}$ por ano.}
    \includegraphics[width=0.9\textwidth]{Figuras/SWD por ano.png}
    \par\small{Fonte: Autor (2025)}
    \label{fig:swd_ano}
\end{figure}

O ciclo sazonal mensal é bem definido (Figura~\ref{fig:swd_mes}), com máximos em dezembro e janeiro (acima de $400$~W/m$^2$) e mínimos em junho e julho (cerca de $170$–$180$~W/m$^2$). O suporte amostral é elevado em todos os meses, variando de aproximadamente $10{,}400$ (fevereiro) a mais de $15{,}000$ (agosto), de forma que as médias mensais podem ser consideradas representativas. Já no recorte por dia do mês (Figura~\ref{fig:swd_dia}), observa-se estabilidade relativa entre os dias, sem tendência sistemática, mas com queda clara no número de registros a partir do dia 29, efeito esperado pela própria duração dos meses. O dia 31, com apenas $n_b \approx 3{,}072$, deve ser interpretado com cautela em comparação aos dias 1 a 28, que possuem em torno de $5{,}000$ registros cada.

\begin{figure}[!h]
    \centering
    \caption{Média de $\mathrm{SWD}$ por mês.}
    \includegraphics[width=0.9\textwidth]{Figuras/SWD por mes.png}
    \par\small{Fonte: Autor (2025)}
    \label{fig:swd_mes}
\end{figure}

\begin{figure}[!h]
    \centering
    \caption{Média de $\mathrm{SWD}$ por dia do mês.}
    \includegraphics[width=0.9\textwidth]{Figuras/SWD por dia.png}
    \par\small{Fonte: Autor (2025)}
    \label{fig:swd_dia}
\end{figure}

\FloatBarrier


O perfil intradiário (Figura~\ref{fig:swd_hora}) confirma o ciclo solar típico, com crescimento entre 5h e 12h (pico de aproximadamente $650$~W/m$^2$) e posterior decréscimo até o pôr do sol. Nesse caso, cada horário conta com $n_b=9{,}712$ registros, assegurando comparabilidade entre horas. Esse equilíbrio decorre da metodologia adotada no pré-processamento: sempre que um dia apresentava lacunas superiores a 30 minutos, todo o dia foi excluído da base, ao invés de manter séries incompletas. Essa decisão garante que o modelo aprenda de forma consistente o comportamento intradiário completo, sem distorções causadas por dias com cobertura parcial. De modo semelhante, os minutos intrahorários possuem suporte equilibrado ($n_b=38{,}848$ em cada classe), mostrando que não há efeito sistemático associado ao minuto da coleta.

\begin{figure}[!h]
    \centering
    \caption{Média de $\mathrm{SWD}$ por hora do dia.}
    \includegraphics[width=0.9\textwidth]{Figuras/SWD por hora.png}
    \par\small{Fonte: Autor (2025)}
    \label{fig:swd_hora}
\end{figure}
\FloatBarrier


Com o objetivo de reduzir a natureza cíclica dessas variáveis e facilitar a captura de padrões pelos modelos de aprendizado, foram aplicadas transformações sazonais de base trigonométrica, resultando nas variáveis hora sazonal e mês sazonal Essa transformação reposiciona os instantes em uma escala contínua normalizada ($[0,1]$), preservando a circularidade implícita. O resultado é uma relação mais próxima de linear entre a irradiância e as variáveis temporais transformadas, conforme observado nas Figuras~\ref{fig:swd_hora_saz} e \ref{fig:swd_mes_saz}.

Na Figura~\ref{fig:swd_hora_saz}, observa-se crescimento monotônico da irradiância média ao longo das classes de hora sazonal, com valores próximos de zero no início da escala e máximos superiores a $600$~W/m$^2$ nas classes finais. O suporte amostral é equilibrado, com cerca de $19{,}424$ observações por classe (exceto o primeiro bin, que reúne $38{,}848$ registros), conferindo robustez às médias. Esse padrão mais linear tende a favorecer algoritmos sensíveis a relações monotônicas, reduzindo a necessidade de o modelo "aprender" a não linearidade do ciclo solar.

De forma análoga, a variável mês sazonal (Figura~\ref{fig:swd_mes_saz}) traduz a sazonalidade anual em um gradiente contínuo: os menores valores da escala concentram médias de $170$–$200$~W/m$^2$, enquanto os maiores valores superam $400$~W/m$^2$. O suporte amostral é consistente, variando entre $25{,}344$ e $46{,}784$ registros por classe, o que assegura estabilidade nas médias. Assim, a transformação cossenoidal não apenas lineariza a relação entre mês e irradiância, mas também preserva a simetria do ciclo, melhorando o potencial de generalização dos modelos preditivos.

\begin{figure}[!h]
    \centering
    \caption{Média de $\mathrm{SWD}$ por hora sazonal.}
    \includegraphics[width=0.9\textwidth]{Figuras/SWD por hora sazonal.png}
    \par\small{Fonte: Autor (2025)}
    \label{fig:swd_hora_saz}
\end{figure}

\begin{figure}[!h]
    \centering
    \caption{Média de $\mathrm{SWD}$ por mês sazonal.}
    \includegraphics[width=0.9\textwidth]{Figuras/SWD por mes sazonal.png}
    \par\small{Fonte: Autor (2025)}
    \label{fig:swd_mes_saz}
\end{figure}
\FloatBarrier



Entre as variáveis meteorológicas, a pressão atmosférica (Figura~\ref{fig:swd_press}) apresenta associação pouco estruturada, mas seu suporte amostral é bastante desigual: classes centrais concentram dezenas de milhares de observações (p.ex., $n_b=39{,}434$ entre 957–960~hPa), enquanto as extremidades reúnem menos de $100$ registros. Portanto, médias extremas de pressão não devem ser tomadas como representativas. A umidade relativa (Figura~\ref{fig:swd_rh}) exibe relação inversa com a irradiância, mas novamente a robustez varia: classes secas ($<20\%$) possuem menos de $200$ registros, ao passo que classes úmidas ($>60\%$) reúnem dezenas de milhares de observações, conferindo maior confiabilidade à conclusão de que altos valores de $\mathrm{RH}$ estão associados a irradiância reduzida. No caso da temperatura (Figura~\ref{fig:swd_temp}), observa-se crescimento da média de $\mathrm{SWD}$ até cerca de $30^{\circ}$C; contudo, as classes acima desse valor apresentam forte queda no número de observações ($n_b=1{,}595$ em 31–35$^{\circ}$C e apenas $n_b=12$ em 35–39$^{\circ}$C), tornando inviável afirmar uma estabilização ou declínio consistente.

\begin{figure}[!h]
    \centering
    \caption{Média de $\mathrm{SWD}$ por intervalo de umidade relativa.}
    \includegraphics[width=0.9\textwidth]{Figuras/SWD por RH.png}
    \par\small{Fonte: Autor (2025)}
    \label{fig:swd_rh}
\end{figure}

\begin{figure}[!h]
    \centering
    \caption{Média de $\mathrm{SWD}$ por intervalo de pressão atmosférica.}
    \includegraphics[width=0.9\textwidth]{Figuras/SWD por pressure.png}
    \par\small{Fonte: Autor (2025)}
    \label{fig:swd_press}
\end{figure}

\begin{figure}[!h]
    \centering
    \caption{Média de $\mathrm{SWD}$ por intervalo de temperatura.}
    \includegraphics[width=0.9\textwidth]{Figuras/SWD por TEMP .png}
    \par\small{Fonte: Autor (2025)}
    \label{fig:swd_temp}
\end{figure}

De forma geral, a análise mostra que as distribuições temporais (ano, mês, hora e minuto) contam com elevado e, em muitos casos, homogêneo suporte amostral, legitimando conclusões robustas sobre ciclos sazonais e intradiários. Já as variáveis meteorológicas apresentam padrões relevantes, especialmente no caso da umidade relativa e da temperatura, mas a interpretação deve sempre considerar a concentração de dados em faixas intermediárias e a escassez nos extremos. Assim, a análise descritiva evidencia tanto os determinantes físicos da irradiância quanto os limites impostos pela disponibilidade amostral em determinadas condições.










% \begin{figure}[!h]
%     \centering
%     \caption{Média de $\mathrm{SWD}$ por classes de $k_t^{*}$.}
%     \includegraphics[width=0.85\textwidth]{figuras/swd_por_kt.png}
%     \par\small{Fonte: Autor (2025)}
%     \label{fig:swd_kt}
% \end{figure}

\FloatBarrier



% \subsection{Sazonalidade e ciclos diários}

% A Figura~\ref{fig:swdday} resume o perfil intradiário da irradiância. Verifica-se o crescimento exponencial a partir do nascer do sol, atingindo máximo próximo ao meio-dia solar ($\approx 644$ W/m$^2$ às 12h) e posterior declínio até o pôr do sol. Esse padrão em formato gaussiano é típico de séries de irradiância e confirma a necessidade de modelos capazes de capturar periodicidade horária.

% A Figura~\ref{fig:swdmonth} mostra a sazonalidade anual, com máximos médios em dezembro ($414$ W/m$^2$) e janeiro ($408$ W/m$^2$) e mínimos em junho ($169$ W/m$^2$) e julho ($179$ W/m$^2$), refletindo a geometria solar do hemisfério sul. Esse ciclo anual justifica a adoção de variáveis sazonais (como seno e cosseno do mês) na modelagem.

% \begin{figure}[!h]
%     \centering
%     \caption{Média de $\mathrm{SWD}$ por hora do dia (perfil intradiário).}
%     \includegraphics[width=0.85\textwidth]{figuras/swd_por_hora.png}
%     \par\small{Fonte: Autor (2025)}
%     \label{fig:swdday}
% \end{figure}

% \begin{figure}[!h]
%     \centering
%     \caption{Média de $\mathrm{SWD}$ por mês do ano (sazonalidade anual).}
%     \includegraphics[width=0.85\textwidth]{figuras/swd_por_mes.png}
%     \par\small{Fonte: Autor (2025)}
%     \label{fig:swdmonth}
% \end{figure}

\subsection{Correlação entre variáveis}

As matrizes de Pearson e Spearman, Figura~\ref{fig:corr} e Figura~\ref{fig:corr1}, respectivamente,  mostram que a variável temporal transformada hora sazonal é o preditor individual mais associado à irradiância: a correlação é forte e positiva tanto em Spearman ($\rho\approx0{,}81$) quanto em Pearson ($r\approx0{,}75$). Esse ganho, frente à correlação praticamente nula com a hora bruta ($\rho\approx-0{,}13$; $r\approx-0{,}09$), confirma que a transformação cossenoidal lineariza o ciclo intradiário e preserva a monotonicidade esperada com o ângulo zenital solar. Em seguida, destacam-se a temperatura do ar, com correlação positiva moderada e estável entre os métodos ($\rho\approx0{,}47$; $r\approx0{,}48$), e a umidade relativa, com correlação negativa moderada a forte e igualmente estável ($\rho\approx-0{,}57$; $r\approx-0{,}60$). A sazonalidade anual transformada \textit{mes\_sazonal} apresenta correlação positiva de baixa a moderada ($\rho\approx0{,}25$; $r\approx0{,}27$), superior à do \textit{mes} categorizado ($\rho\approx0{,}03$; $r\approx0{,}03$), reforçando que a codificação cíclica também favorece o aprendizado do padrão sazonal. As demais variáveis temporais (\textit{dia}, \textit{minuto} e \textit{ano}) exibem correlações próximas de zero, e a pressão atmosférica mantém associação muito fraca com $\mathrm{SWD}$ em ambas as métricas ($\rho\approx0{,}02$; $r\approx0{,}03$).

Comparando-se os dois coeficientes, observou-se que Spearman tende a realçar levemente relações monotônicas não estritamente lineares: isso aparece sobretudo em \textit{hora\_sazonal}, cuja força aumenta de $r\approx0{,}75$ para $\rho\approx0{,}81$, sugerindo pequeno desvio de linearidade no ciclo diurno que ainda assim é capturado como monotônico. Para temperatura e umidade relativa, os valores são praticamente coincidentes entre Pearson e Spearman, indicando que, no escopo amostral, as relações com $\mathrm{SWD}$ são aproximadamente lineares. Já a \textit{hora} bruta ganha (em módulo) na correlação de Spearman em relação a Pearson, efeito compatível com a forma não linear (em “arco”) do ciclo intradiário quando não transformado.

As correlações entre preditores ajudam a antecipar redundâncias: \textit{mes\_sazonal} apresenta correlação moderada com temperatura ($\rho\approx0{,}61$; $r\approx0{,}60$) e moderada negativa com pressão ($\rho\approx-0{,}49$; $r\approx-0{,}50$); temperatura é moderadamente anticorrelacionada a umidade relativa ($\rho\approx-0{,}50$; $r\approx-0{,}50$) e a pressão ($\rho\approx-0{,}52$; $r\approx-0{,}53$); \textit{hora\_sazonal} correlaciona-se de forma leve a moderada com temperatura ($\rho\approx0{,}33$; $r\approx0{,}33$) e negativamente com umidade relativa ($\rho\approx-0{,}42$; $r\approx-0{,}42$). Tais associações sugerem atenção à multicolinearidade em modelos lineares; para esses casos, recomenda-se priorizar as codificações sazonais (\textit{hora\_sazonal} e \textit{mes\_sazonal}) em lugar das versões brutas (\textit{hora} e \textit{mes}), e empregar regularização. Em modelos baseados em árvores, a redundância tende a ser menos crítica, mas a preferência por codificações cíclicas mantém-se vantajosa por favorecer separações mais simples.

Em síntese, as variáveis mais promissoras para previsão de $\mathrm{SWD}$, à luz da Figura~\ref{fig:corr}, são: \textit{hora\_sazonal} (maior correlação direta), \textit{RH} (maior correlação inversa), \textit{Temp} (correlação direta moderada) e \textit{mes\_sazonal} (efeito sazonal anual capturado de forma mais linear). As variáveis \textit{hora} e \textit{mes} podem ser descartadas em favor de suas versões sazonais; \textit{dia}, \textit{minuto}, \textit{ano} e \textit{Pressure} mostram baixo potencial explicativo isolado e devem ser incluídas, se for o caso, apenas como controles ou para capturar efeitos de interação. Considerando que Spearman foi ligeiramente mais sensível às relações monotônicas induzidas pelas transformações cíclicas, adotou-se sua leitura como referência para interpretar força ordinal das associações, mantendo-se Pearson como verificação de linearidade. Essa dupla leitura sustenta a decisão de engenharia de variáveis e orienta a parcimônia na seleção de preditores.


\begin{figure}[!h]
    \centering
    \caption{Matriz de correlação de Pearson.}
    \includegraphics[width=0.95\textwidth]{Figuras/Pearson.png}
    \par\small{Fonte: Autor (2025)}
    \label{fig:corr}
\end{figure}

\begin{figure}[!h]
    \centering
    \caption{Matriz de correlação de Spearman.}
    \includegraphics[width=0.95\textwidth]{Figuras/Spearman.png}
    \par\small{Fonte: Autor (2025)}
    \label{fig:corr1}
\end{figure}

\subsection{Síntese da Análise Exploratória (DEA)}

A DEA, ancorada nas distribuições univariadas (com $n_b$ reportado no topo de cada barra) e nas matrizes de correlação (Figura~\ref{fig:corr}), permitiu consolidar evidências sobre sazonalidade, condicionamento atmosférico e seleção de preditores. O pré-processamento adotado assegurou suporte homogêneo por hora e minuto — dias com lacunas superiores a 30 minutos foram removidos integralmente — e as codificações sazonais (\textit{hora\_sazonal} e \textit{mes\_sazonal}) foram construídas com bins de largura igual, evitando quebras artificiais.

\begin{itemize}
    \item \textbf{Sazonalidade e codificação cíclica.} O ciclo intradiário foi capturado de forma quase linear por \textit{hora\_sazonal}, que apresentou a correlação mais forte com $\mathrm{SWD}$ (Spearman $\approx 0{,}81$; Pearson $\approx 0{,}75$), superando amplamente a \textit{hora} bruta (correlações próximas de zero). No ciclo anual, \textit{mes\_sazonal} exibiu associação positiva baixa a moderada (Spearman $\approx 0{,}25$; Pearson $\approx 0{,}27$), superior à do \textit{mes} categorizado. Esse resultado confirma a vantagem de codificadores sazonais para modelos supervisionados.
    \item \textbf{Condicionamento atmosférico.} Temperatura do ar mostrou correlação positiva moderada e consistente entre métricas (Spearman/Pearson $\approx 0{,}47$–$0{,}48$); umidade relativa apresentou correlação negativa moderada a forte (Spearman $\approx -0{,}57$; Pearson $\approx -0{,}60$). Em contraste, pressão atmosférica manteve associação muito fraca com $\mathrm{SWD}$. A interpretação em caudas deve ser cautelosa: classes extremas de temperatura ($>31^{\circ}\mathrm{C}$), umidade muito baixa ($<18\%$) e pressões nos extremos possuem $n_b$ reduzido, elevando a incerteza das médias.
    \item \textbf{Completude e robustez amostral.} As segmentações por hora ($n_b=9{,}712$ por classe) e minuto ($n_b=38{,}848$) apresentaram suporte equilibrado por decisão metodológica; por dia do mês houve queda natural de $n_b$ nos dias 29–31; por ano observou-se heterogeneidade marcante (p.ex., 2013 com $n_b=256$ versus 2010–2011 com $n_b>23{,}000$), o que explica parte da variabilidade anual nas médias.
    \item \textbf{Implicações para seleção de preditores.} Recomenda-se priorizar \textit{hora\_sazonal}, \textit{Temp}, \textit{RH} e \textit{mes\_sazonal} como conjunto básico. As versões brutas \textit{hora} e \textit{mes} podem ser substituídas por suas codificações cíclicas. Variáveis com correlação próxima de zero (\textit{dia}, \textit{minuto}, \textit{ano} e \textit{Pressure}) tendem a ter baixo poder explicativo isolado e, se incluídas, devem atuar como controles ou em termos de interação. Quando disponível, a componente de céu claro ou o índice de transmitância ($k_t^{*}$) permanece como candidato relevante para distinguir cenários de nebulosidade.
    \item \textbf{Leitura de correlações e modelagem.} Spearman destacou de forma ligeiramente superior relações monotônicas não estritamente lineares — especialmente em \textit{hora\_sazonal} — enquanto Pearson confirmou a proximidade de linearidade para \textit{Temp} e \textit{RH}. Para modelos lineares, recomenda-se regularização para mitigar multicolinearidade entre preditores sazonais e meteorológicos (p.ex., correlação entre \textit{mes\_sazonal} e \textit{Temp}); em modelos baseados em árvores, as codificações cíclicas ainda favorecem separações mais parcimoniosas.
\end{itemize}

Em conjunto, os achados confirmam a adequação da base e a eficácia das decisões de engenharia de dados (remoção de dias incompletos e codificação cíclica), ao mesmo tempo em que delineiam um conjunto parcimonioso e informativo de variáveis. A evidência empírica respalda o avanço para a etapa de modelagem, na qual se espera que preditores sazonais e meteorológicos — com ênfase em \textit{hora\_sazonal}, \textit{Temp}, \textit{RH} e \textit{mes\_sazonal} — ofereçam ganhos de precisão e generalização, desde que a avaliação considere a heterogeneidade de suporte amostral observada nas diferentes classes.






% ---------------------------------------
% CAPÍTULO 5 — Resultados e Discussões
% ---------------------------------------
\chapter{Resultados e Discussões}\label{chap:resultados}

% Aqui nesta seção serão apresentados os resultados dos modelos (RMSE e $R^2$), assim como o gráfico de previsão e as otimizações feitas.

% \subsection{Visão geral dos experimentos}
% % Breve resumo da lógica dos cenários e organização dos resultados
\section{Resultados do ARIMA}

O processo de seleção automática de parâmetros (\texttt{auto\_arima}) indicou que a estrutura mais adequada para a série de irradiância foi o modelo ARIMA$(2,0,0)(1,0,0)_{64}$ com intercepto. Essa configuração combina dois termos autorregressivos de curta defasagem e um componente sazonal diário, coerente com a periodicidade de 24 horas da irradiância solar amostrada em intervalos de 15 minutos ($s=64$).

A previsão in-sample obtida pelo modelo é apresentada na Figura~\ref{fig:arima_resultados}, no período de 14 a 16 de janeiro de 2016. Observa-se que a curva prevista ($SWD_{ARIMA}(t)$) acompanha de forma satisfatória a tendência suave da série observada, reproduzindo adequadamente o padrão de subida e descida ao longo do ciclo diário. Pequenas discrepâncias surgem em horários próximos ao pico, quando a cobertura de nuvens provoca variações abruptas na irradiância.

\begin{figure}[!h]
    \centering
    \caption{Previsão in-sample do modelo ARIMA e resíduos correspondentes no período de 14 a 16 de janeiro de 2016.}
    \includegraphics[width=0.95\textwidth]{Figuras/ARIMA.png}
    \par\small{Fonte: Autor (2025)}
    \label{fig:arima_resultados}
\end{figure}
\FloatBarrier
A série de resíduos ($Res_{ARIMA}(t)$) apresenta comportamento característico de oscilações rápidas, concentradas em torno do meio-dia solar. Essas oscilações decorrem de flutuações de nebulosidade que o modelo linear não é capaz de reproduzir. 

Nesse sentido, apenas o resíduo $Res_{ARIMA}(t)$ foi incorporado ao conjunto de variáveis explicativas. Tal decisão fundamenta-se no fato de que o resíduo carrega informação adicional sobre irregularidades atmosféricas de curta duração, atuando como uma \textit{feature} auxiliar para os modelos de aprendizado de máquina (LSTM e XGBoost). Dessa forma, espera-se que a inclusão dessa variável contribua para capturar padrões não lineares e melhorar a acurácia das previsões.



\section{Resultados com LSTM}

\subsection{LSTM base (sem variáveis adicionais)}
Inicialmente foi avaliada uma configuração de rede LSTM com hiperparâmetros escolhidos com base na literatura, sem realização de processo de otimização. Foram adotados \textit{input window} de 64 passos (equivalentes a um dia de histórico, dado o intervalo de 15 minutos) e \textit{output window} de 64 passos (previsão para o dia subsequente). Adicionalmente, foi testada a utilização de uma janela de sete dias como histórico, de modo a avaliar o impacto do maior horizonte de entrada no desempenho.

A arquitetura considerou duas camadas, 40 neurônios por camada, função de ativação linear na saída, taxa de dropout de $0{,}2$ e otimizador Adam com taxa de aprendizado de $0{,}001$. O treinamento foi realizado com \textit{batch size} de 512, até 100 épocas, com \textit{early stopping} definido em 10 épocas de paciência.

As variáveis explicativas utilizadas encontram-se listadas a seguir:

\begin{itemize}
    \item Ano, dia e minuto (indicadores temporais);
    \item $\mathrm{SWD}$ (irradiância global medida);
    \item $RH$ (umidade relativa);
    \item Pressão atmosférica;
    \item Temperatura do ar;
    \item $k_t^{*}$ (índice de transmitância atmosférica);
    \item Hora sazonal (função trigonométrica);
    \item Mês sazonal (função trigonométrica).
\end{itemize}

Como requisito para consistência das janelas de entrada e saída, foram considerados apenas os anos com número suficiente de dias completos no banco de dados. A Tabela~\ref{tab:dias_completos} apresenta a contagem de dias completos por ano, sendo escolhidos os períodos de 2009, 2010, 2011, 2012, 2015 e 2016 para compor os experimentos.

\begin{table}[H]
    \centering
    \caption{Número de dias completos por ano na base de dados.}
    % \resizebox{\textwidth}{!}{%
    \begin{tabular}{|c|c|}
        \hline
        \textbf{Ano} & \textbf{Dias completos} \\
        \hline
        2009 & 321 \\
        2010 & 365 \\
        2011 & 363 \\
        2012 & 296 \\
        2015 & 249 \\
        2016 & 335 \\
        \hline
    \end{tabular}
    % }
    \par\small{Fonte: Autor (2025)}
    \label{tab:dias_completos}
\end{table}

A divisão entre treino, validação e teste foi realizada de forma cronológica, conforme descrito a seguir:
\begin{itemize}
    \item \textbf{Treino}: anos de 2009 a 2012;
    \item \textbf{Validação}: ano de 2015;
    \item \textbf{Teste}: ano de 2016.
\end{itemize}

Essa estratégia garante que o modelo seja treinado com uma série de anos consecutivos, avaliado em um ano intermediário e testado em um período posterior, permitindo observar sua capacidade de generalização temporal.

Com essa configuração inicial, o modelo apresentou RMSE de \textbf{159,24 W/m²} e $R^2$ de \textbf{0,7538} no conjunto de teste. Esses valores indicam que, mesmo sem variáveis auxiliares ou ajuste fino de hiperparâmetros, a LSTM foi capaz de capturar parte relevante da variabilidade da irradiância.

Testes adicionais mostraram que o uso de sete dias de histórico no \textit{input window} não trouxe ganhos de desempenho, resultando em RMSE de \textbf{161,00 W/m²} e $R^2$ de \textbf{0,7483}. Esse resultado sugere que a irradiância do dia seguinte depende majoritariamente do comportamento do dia imediatamente anterior, não se beneficiando da inclusão de dias mais distantes no passado. 

Da mesma forma, ao substituir as variáveis sazonais (\textit{hora sazonal} e \textit{mês sazonal}) por suas versões lineares (\textit{hora} e \textit{mês}), o modelo apresentou piora de desempenho, com RMSE de \textbf{161,83 W/m²} e $R^2$ de \textbf{0,7457}. Isso evidencia que a representação sazonal por funções trigonométricas é mais adequada para capturar a periodicidade da irradiância solar.

Dessa forma, para todos os experimentos subsequentes, decidiu-se adotar: (i) \textit{input window} de 64 passos (1 dia) e (ii) variáveis sazonais ao invés de representações lineares. Essa decisão visa reduzir o espaço de busca de configurações e concentrar a análise nas variáveis e ajustes com maior potencial de ganho preditivo.


A Tabela~\ref{tab:resultados_lstm} apresenta o resumo dos resultados, que será atualizado nas subseções seguintes à medida que diferentes variantes do modelo forem avaliadas.

\begin{table}[!h]
    \centering
    \caption{Resumo inicial de desempenho dos modelos LSTM.}
    % \resizebox{0.9\textwidth}{!}{%
    \begin{tabular}{|l|c|c|}
        \hline
        \textbf{Modelo} & \textbf{RMSE (W/m²)} & \textbf{$R^2$} \\
        \hline
        LSTM base (sem variáveis adicionais) & 159,24 & 0,7538 \\
        \hline
    \end{tabular}
    % }
    \par\small{Fonte: Autor (2025)}
    \label{tab:resultados_lstm}
\end{table}
\FloatBarrier

Para avaliação qualitativa, foram selecionados três dias distintos:
\begin{enumerate}
    \item Um dia típico, de céu claro, em que o modelo apresenta desempenho satisfatório;
    \item Um caso de transição \textbf{normal $\rightarrow$ chuvoso}, em que a previsão não acompanha as oscilações abruptas provocadas pela nebulosidade repentina;
    \item Um caso de transição \textbf{chuvoso $\rightarrow$ normal}, em que o histórico prejudica a previsão do dia subsequente de céu limpo.
\end{enumerate}

As Figuras~\ref{fig:lstm_base_normal}, \ref{fig:lstm_base_normal_chuvoso} e \ref{fig:lstm_base_chuvoso_normal} ilustram essas situações.

\begin{figure}[!h]
    \centering
    \caption{Previsão com LSTM base em dia típico de céu claro (20/01/2016).}
    \includegraphics[width=0.95\textwidth]{Figuras/LSTM BASICO DIA 20-01-2016.png}
    \par\small{Fonte: Autor (2025)}
    \label{fig:lstm_base_normal}
\end{figure}

No dia típico de céu claro (Figura~\ref{fig:lstm_base_normal}), a rede LSTM conseguiu reproduzir de forma satisfatória a curva de irradiância, acompanhando bem tanto a ascensão matinal quanto o decaimento ao final da tarde. Nesse caso, a regularidade do padrão diário favoreceu o modelo, que aprendeu adequadamente a dinâmica suave da série.

\begin{figure}[!h]
    \centering
    \caption{Previsão com LSTM base em caso de transição normal $\rightarrow$ chuvoso (15/05/2016).}
    \includegraphics[width=0.95\textwidth]{Figuras/LSTM BASICO DIA 15-05-2016.png}
    \par\small{Fonte: Autor (2025)}
    \label{fig:lstm_base_normal_chuvoso}
\end{figure}

Já na transição de um dia normal para um dia chuvoso (Figura~\ref{fig:lstm_base_normal_chuvoso}), a previsão mostrou maior dificuldade. A rede, ao receber como histórico um dia de céu limpo, projetou para o dia seguinte um padrão igualmente regular. Entretanto, a presença de nebulosidade intensa resultou em valores observados muito inferiores, gerando grandes discrepâncias. Este foi o cenário mais problemático para o modelo LSTM básico, pois evidencia sua limitação em antecipar mudanças abruptas nas condições atmosféricas.

\begin{figure}[!h]
    \centering
    \caption{Previsão com LSTM base em caso de transição chuvoso $\rightarrow$ normal (19/04/2016).}
    \includegraphics[width=0.95\textwidth]{Figuras/LSTM BASICO DIA 19-04-2016.png}
    \par\small{Fonte: Autor (2025)}
    \label{fig:lstm_base_chuvoso_normal}
\end{figure}

No caso inverso, de transição de um dia chuvoso para um dia normal (Figura~\ref{fig:lstm_base_chuvoso_normal}), o modelo também apresentou limitações, embora em menor intensidade. O histórico de irradiância baixa levou a previsões subestimadas no dia subsequente de céu limpo. Ainda assim, o desvio não foi tão acentuado quanto no caso anterior, visto que o padrão de crescimento diário foi parcialmente capturado, ainda que com valores deslocados para baixo.



\subsection{LSTM com resíduo ARIMA}

Nesta variante, foi incorporado exclusivamente o resíduo do ARIMA, $Res_{ARIMA}(t)$, conforme Equação (\ref{eq:arima_residuo}), como variável auxiliar ao conjunto de entradas. Mantiveram-se as demais configurações do modelo: janelas de entrada e saída de 64 passos (1 dia), duas camadas LSTM com 40 unidades, \textit{dropout} de $0{,}2$, \textit{optimizer} Adam ($\eta=0{,}001$) e \textit{early stopping} com paciência de 10 épocas. 

Com a inclusão de $Res_{ARIMA}(t)$, observou-se melhora nas métricas globais, com RMSE de \textbf{157,40 W/m²} e $R^2$ de \textbf{0,7595} no conjunto de teste, em comparação ao modelo base (RMSE 159,24 W/m²; $R^2$ 0,7538). Esse ganho é compatível com a expectativa de que o resíduo concentre variações rápidas não capturadas pela dinâmica suavizada.

A Figura~\ref{fig:lstm_res_arima_normal_chuvoso} ilustra o caso representativo de transição \textbf{normal $\rightarrow$ chuvoso}. Nota-se que a inclusão de $Res_{ARIMA}(t)$ reduz parcialmente o viés em períodos com nebulosidade, atenuando a superestimação típica do modelo base. Ainda assim, permanece dificuldade em antecipar quedas abruptas de irradiância, sobretudo quando a nebulosidade se estabelece de maneira súbita ao longo do dia. Esse comportamento é coerente com a natureza da variável: o resíduo oferece um \textit{sinal de desvio} aprendido a partir do histórico, mas não contém informação exógena prospectiva suficiente para antecipar mudanças de regime atmosférico.

\begin{figure}[!h]
    \centering
    \caption{Previsão com LSTM + $Res_{ARIMA}(t)$ em transição normal $\rightarrow$ chuvoso.}
    \includegraphics[width=0.95\textwidth]{Figuras/LSTM ARIMA DIA 15-05-2016.png}
    \par\small{Fonte: Autor (2025)}
    \label{fig:lstm_res_arima_normal_chuvoso}
\end{figure}

Em síntese, a variável $Res_{ARIMA}(t)$ contribuiu para capturar irregularidades de curta duração, produzindo melhora modesta porém consistente nas métricas globais. Contudo, a ausência de previsores meteorológicos impede o modelo de antecipar eventos subdiários abruptos, o que motiva a avaliação, nas subseções seguintes, do uso de variáveis de previsão meteorológica como entradas adicionais.

A Tabela~\ref{tab:resultados_lstm_atualizada} apresenta a atualização da comparação acumulada entre os modelos LSTM. Nota-se que a inclusão do resíduo ARIMA elevou o desempenho em relação ao modelo base, confirmando a relevância dessa variável derivada. Essa tabela será progressivamente expandida ao longo desta seção, permitindo observar de forma sistemática os ganhos ou perdas decorrentes de cada conjunto de variáveis incorporadas.


\begin{table}[H]
    \centering
    \caption{Resumo atualizado de desempenho dos modelos LSTM (até esta subseção).}
    % \resizebox{0.8\textwidth}{!}{%
    \begin{tabular}{|l|c|c|}
        \hline
        \textbf{Modelo} & \textbf{RMSE (W/m²)} & \textbf{$R^2$} \\
        \hline
        LSTM base (sem variáveis adicionais) & 159{,}24 & 0{,}7538 \\
        LSTM + $Res_{ARIMA}(t)$ & 157{,}40 & 0{,}7595 \\
        \hline
    \end{tabular}
    % }
    \par\small{Fonte: Autor (2025)}
    \label{tab:resultados_lstm_atualizada}
\end{table}

\subsection{LSTM com variáveis de previsão meteorológica}

Na terceira configuração, foram incorporadas variáveis provenientes de previsões meteorológicas (temperatura, umidade relativa e pressão atmosférica) ao conjunto de entradas. O restante da arquitetura foi mantido inalterado em relação aos experimentos anteriores, de forma a isolar o impacto do acréscimo dessas variáveis.

Com essa modificação, observou-se um ganho substancial no desempenho global: o RMSE \textbf{103,10 W/m²}, enquanto o $R^2$ alcançou \textbf{0,8968}. Esses valores representam uma melhora expressiva frente ao modelo base (RMSE 159,24 W/m²; $R^2$ 0,7538) e ao modelo com resíduo ARIMA (RMSE 157,40 W/m²; $R^2$ 0,7595), confirmando a relevância da incorporação de previsores meteorológicos.

A Figura~\ref{fig:lstm_prev_meteo_normal_chuvoso} apresenta o caso de transição \textbf{normal $\rightarrow$ chuvoso}. Nota-se que, embora o modelo ainda encontre dificuldades em antecipar quedas abruptas de irradiância, o padrão geral da curva prevista aproxima-se de maneira muito mais realista do observado, em comparação às variantes anteriores. O uso das variáveis meteorológicas permitiu ao modelo ajustar-se melhor ao regime atmosférico do dia subsequente, capturando de forma mais adequada oscilações decorrentes da nebulosidade.

\begin{figure}[!h]
    \centering
    \caption{Previsão com LSTM + variáveis de previsão meteorológica em transição normal $\rightarrow$ chuvoso.}
    \includegraphics[width=0.95\textwidth]{Figuras/LSTM PREVISAO DIA 15-05-2016.png}
    \par\small{Fonte: Autor (2025)}
    \label{fig:lstm_prev_meteo_normal_chuvoso}
\end{figure}

Esse resultado evidencia que a inclusão de informações exógenas, ainda que sujeitas a incertezas próprias de modelos meteorológicos, é fundamental para a previsão de irradiância. Enquanto os modelos anteriores se apoiavam apenas em padrões históricos e resíduos, o uso de previsores externos fornece ao modelo um contexto físico adicional, reduzindo de maneira significativa o erro médio.

A Tabela~\ref{tab:resultados_lstm_atualizada} apresenta a comparação acumulada entre os modelos até esta etapa. Nota-se que a LSTM com previsores meteorológicos é, até o momento, a configuração de melhor desempenho, superando claramente as abordagens baseadas apenas em informações históricas.

\begin{table}[H]
    \centering
    \caption{Resumo atualizado de desempenho dos modelos LSTM (Tabela 10).}
    % \resizebox{0.85\textwidth}{!}{%
    \begin{tabular}{|l|c|c|}
        \hline
        \textbf{Modelo} & \textbf{RMSE (W/m²)} & \textbf{$R^2$} \\
        \hline
        LSTM base (sem variáveis adicionais) & 159{,}24 & 0{,}7538 \\
        LSTM + $Res_{ARIMA}(t)$ & 157{,}40 & 0{,}7595 \\
        LSTM + previsores meteorológicos & 103{,}10 & 0{,}8968 \\
        \hline
    \end{tabular}
    % }
    \par\small{Fonte: Autor (2025)}
    \label{tab:resultados_lstm_atualizada}
\end{table}

\subsection{LSTM com variáveis de previsão meteorológica com ruído}


Por fim, avaliou-se o impacto da utilização de variáveis de previsão meteorológica sujeitas a incerteza. Para tal, foi adicionado um ruído aleatório de até $\pm 5\%$ sobre os valores de temperatura, umidade relativa e pressão atmosférica previstos, de modo a simular um cenário mais realista, no qual até mesmo as melhores previsões meteorológicas estão sujeitas a erros.

Com essa configuração, o modelo apresentou RMSE de \textbf{104,21 W/m²} e $R^2$ de \textbf{0,8945}. Embora esses resultados indiquem uma leve piora em relação ao uso das previsões “exatas” (RMSE 103,10 W/m²; $R^2$ 0,8968), o desempenho permanece substancialmente superior ao obtido sem a inclusão de previsores meteorológicos, ou mesmo ao uso apenas do resíduo ARIMA.

A Figura~\ref{fig:lstm_prev_meteo_ruido_normal_chuvoso} apresenta novamente o caso de transição \textbf{normal $\rightarrow$ chuvoso}. Observa-se que, apesar do ruído introduzido nas variáveis meteorológicas, o modelo manteve capacidade de adaptação ao regime do dia seguinte, ainda que com menor precisão. O padrão geral da curva prevista segue mais próximo da série real em comparação com as versões sem previsores, evidenciando a resiliência do modelo à presença de incertezas moderadas.

\begin{figure}[!h]
    \centering
    \caption{Previsão com LSTM + variáveis de previsão meteorológica com ruído em transição normal $\rightarrow$ chuvoso.}
    \includegraphics[width=0.95\textwidth]{Figuras/LSTM PREVISAO RESIDUO DIA 15-05-2016.png}
    \par\small{Fonte: Autor (2025)}
    \label{fig:lstm_prev_meteo_ruido_normal_chuvoso}
\end{figure}

Esse resultado demonstra que, embora a disponibilidade de previsões exatas represente um limite superior de desempenho, o modelo LSTM se beneficia de forma robusta mesmo em condições mais próximas da prática operacional. A introdução de ruído reduziu marginalmente as métricas, mas o ganho em relação às variantes sem previsores continua expressivo, reforçando a importância da utilização dessas variáveis.

A Tabela~\ref{tab:resultados_lstm_atualizada} apresenta o resumo comparativo atualizado, evidenciando a superioridade das configurações com previsores meteorológicos, mesmo sob a presença de incertezas.

\begin{table}[!h]
    \centering
    \caption{Resumo atualizado de desempenho dos modelos LSTM (Tabela 10).}
    % \resizebox{0.85\textwidth}{!}{%
    \begin{tabular}{|l|c|c|}
        \hline
        \textbf{Modelo} & \textbf{RMSE (W/m²)} & \textbf{$R^2$} \\
        \hline
        LSTM base (sem variáveis adicionais) & 159{,}24 & 0{,}7538 \\
        LSTM + $Res_{ARIMA}(t)$ & 157{,}40 & 0{,}7595 \\
        LSTM + previsores meteorológicos & 103{,}10 & 0{,}8968 \\
        LSTM + previsores meteorológicos (com ruído) & 104{,}21 & 0{,}8945 \\
        \hline
    \end{tabular}
    % }
    \par\small{Fonte: Autor (2025)}
    \label{tab:resultados_lstm_atualizada}
\end{table}

\FloatBarrier


\subsection{Otimização de hiperparâmetros da LSTM}

Com o objetivo de identificar a configuração de rede mais adequada para cada conjunto de atributos de entrada, foi conduzido um processo de otimização de hiperparâmetros utilizando o framework \textit{Optuna}, com o algoritmo \textit{Tree-structured Parzen Estimator} (TPE). Foram analisadas três variantes de entrada de dados: \textit{Feature Set 1} (LSTM + residúo ARIMA), \textit{Feature Set 2} (variáveis com previsões meteorológicos) e \textit{Feature Set 3} (variáveis com previsões ruidosos). Em todos os casos, manteve-se o mesmo critério de parada antecipada (\textit{early stopping} com paciência de 10 épocas) e a mesma métrica de avaliação (RMSE no conjunto de validação).  

Cada estudo foi composto por \textbf{200 trials independentes}, com armazenamento dos resultados em banco de dados SQLite, permitindo análise posterior via \textit{DataFrame} e ferramentas gráficas do \textit{Optuna}. O espaço de busca, comum às três variantes, encontra-se descrito na Tabela~\ref{tab:space_lstm}.

\begin{table}[!h]
    \centering
    \caption{Espaço de busca dos hiperparâmetros da LSTM (comum às três variantes).}
    \resizebox{0.9\textwidth}{!}{%
    \begin{tabular}{|l|l|}
        \hline
        \textbf{Hiperparâmetro} & \textbf{Intervalo / Opções} \\
        \hline
        Número de camadas ($num\_layers$) & [2, 4] (inteiro) \\
        Tamanho da camada oculta ($hidden\_size$) & [1, 80] (inteiro) \\
        Taxa de aprendizado ($learning\_rate$) & [0{,}001, 0{,}5] (escala logarítmica) \\
        Taxa de dropout ($dropout$) & [0{,}1, 0{,}5] (contínuo) \\
        Tamanho do \textit{batch} ($batch\_size$) & \{512, 1024, 2048\} \\
        Otimizador ($optimizer$) & \{Adam, RMSProp\} \\
        Função de ativação ($activation\_function$) & \{Sigmoid, Linear, ReLU, Tanh\} \\
        Mecanismo de atenção ($attention$) & \{None, Simple\} \\
        \hline
    \end{tabular}
    }
    \par\small{Fonte: Autor (2025)}
    \label{tab:space_lstm}
\end{table}

Para evitar redundâncias e facilitar a interpretação, apenas o \textit{Feature Set 2} será apresentado em detalhe, por ter obtido o melhor desempenho geral. As demais variantes (\textit{Feature Sets 1} e \textit{3}) seguiram o mesmo procedimento e apresentaram comportamentos semelhantes, sendo suas métricas resumidas ao final desta subseção.

A Figura~\ref{fig:opt_hist_feature2} apresenta o histórico de otimização, que mostra a evolução do RMSE ao longo dos 200 \textit{trials}. Observa-se rápida convergência nas primeiras 30 iterações, com redução significativa do erro e estabilização da curva do melhor valor após aproximadamente o 40º \textit{trial}. Essa tendência indica que o número de tentativas foi adequado para cobrir o espaço de busca, com baixa probabilidade de ganhos adicionais relevantes.

\begin{figure}[H]
    \centering
    \caption{Histórico de otimização dos \textit{trials} (Feature Set 2).}
    \includegraphics[width=1\textwidth]{Figuras/Evolução - Feature 2.png}
    \par\small{Fonte: Autor (2025)}
    \label{fig:opt_hist_feature2}
\end{figure}
\FloatBarrier

O melhor resultado foi obtido no \textit{trial} número 156, com RMSE de \textbf{98,73~W/m²}, conforme mostrado na Tabela~\ref{tab:best_trial_feature2}. Esse valor representa uma melhora expressiva em relação ao modelo base, cujo RMSE era de 103,10~W/m², correspondendo a um ganho relativo de aproximadamente \textbf{4,2\%} apenas com ajuste fino de hiperparâmetros.

\begin{table}[!h]
    \centering
    \caption{Melhor conjunto de hiperparâmetros obtido (Feature Set 2).}
    \begin{tabular}{|l|l|}
        \hline
        \textbf{Hiperparâmetro} & \textbf{Valor ótimo} \\
        \hline
        Número de camadas & 3 \\
        Tamanho da camada oculta & 26 \\
        Taxa de aprendizado & 0{,}00268 \\
        Dropout & 0{,}25 \\
        \textit{Batch size} & 512 \\
        Otimizador & Adam \\
        Função de ativação & Tanh \\
        Mecanismo de atenção & None \\
        \hline
    \end{tabular}
    \par\small{Fonte: Autor (2025)}
    \label{tab:best_trial_feature2}
\end{table}
\FloatBarrier

A Figura~\ref{fig:importance_feature2} mostra a importância relativa dos hiperparâmetros estimada pelo \textit{Optuna}. Verifica-se que a \textbf{taxa de aprendizado} foi o fator de maior influência no desempenho (55,1\%), seguida pela \textbf{função de ativação} (40,0\%). Demais variáveis apresentaram impacto marginal, com contribuições inferiores a 3\%. Esse padrão reforça que o comportamento dinâmico da otimização e a estabilidade de convergência da LSTM dependem fortemente da calibração da taxa de aprendizado, enquanto parâmetros estruturais (como tamanho da camada oculta e número de camadas) exercem efeito secundário.

\begin{figure}[H]
    \centering
    \caption{Importância dos hiperparâmetros (\textit{Feature Set 2}).}
    \includegraphics[width=0.95\textwidth]{Figuras/Hiperparametros - Feature 2.png}
    \par\small{Fonte: Autor (2025)}
    \label{fig:importance_feature2}
\end{figure}
\FloatBarrier

A análise de correlação entre os parâmetros numéricos e o RMSE confirmou essa predominância: a taxa de aprendizado apresentou correlação positiva de \textbf{+0,86}, indicando que valores muito altos tendem a aumentar o erro. O parâmetro \textit{dropout} mostrou correlação leve e positiva (+0,12), sugerindo que níveis excessivos de regularização prejudicam o ajuste. Por outro lado, o tamanho da camada oculta e o número de camadas exibiram correlações próximas de zero, demonstrando que a complexidade estrutural da rede teve efeito limitado.

A Figura~\ref{fig:slice_feature2} apresenta o \textit{slice plot}, que ilustra a relação direta entre cada hiperparâmetro e o RMSE. Observa-se que os menores erros concentram-se em \textbf{taxas de aprendizado entre 0,002 e 0,01}, \textbf{valores de \textit{dropout} entre 0,2 e 0,3} e \textbf{arquiteturas de até 3 camadas ocultas}. Esse comportamento indica que taxas muito pequenas dificultam a convergência, enquanto valores muito altos levam à instabilidade durante o treinamento.

\begin{figure}[!h]
    \centering
    \caption{Relação entre hiperparâmetros e RMSE (\textit{Slice Plot}, Feature Set 2).}
    \includegraphics[width=0.95\textwidth]{Figuras/Slice - Feature 2.png}
    \par\small{Fonte: Autor (2025)}
    \label{fig:slice_feature2}
\end{figure}
\FloatBarrier

A Figura~\ref{fig:parallel_feature2} apresenta o \textit{Parallel Coordinates Plot}, que tem o objetivo de mostrar as interações entre múltiplos hiperparâmetros e o valor da função objetivo — neste caso, o erro de validação (RMSE) obtido pelo modelo. Cada linha do gráfico representa um \textit{trial} individual (ou seja, uma combinação testada de hiperparâmetros), enquanto cada eixo vertical corresponde a um dos parâmetros ajustados durante a otimização.  

A cor de cada linha indica o valor do RMSE, de modo que tons mais escuros de azul representam combinações que resultaram em menores erros. Essa codificação visual permite identificar, de forma intuitiva, quais regiões do espaço de busca estão associadas a melhor desempenho do modelo.

Em síntese, o gráfico possibilita visualizar:
\begin{itemize}
    \item Quais faixas de valores de hiperparâmetros estão associadas aos menores erros de previsão;
    \item Como diferentes parâmetros interagem entre si, evidenciando combinações mais promissoras — por exemplo, \textit{learning rates} moderadas (em torno de 0,003) associadas a três camadas ocultas e função de ativação \textit{tanh}, que tenderam a produzir os menores valores de RMSE;
    \item Quais parâmetros apresentaram menor impacto no desempenho, como o otimizador (Adam ou RMSProp) e o mecanismo de atenção, cujas variações não alteraram significativamente a qualidade dos resultados.
\end{itemize}
\FloatBarrier

De maneira geral, as linhas mais escuras concentram-se em regiões intermediárias dos eixos de taxa de aprendizado, número de camadas e tamanho da camada oculta, indicando que configurações excessivamente complexas ou com \textit{learning rates} elevadas tendem a aumentar o erro. O padrão observado reforça a conclusão de que a LSTM alcança melhor desempenho com uma estrutura de complexidade moderada e parâmetros de aprendizado ajustados de forma precisa.


\begin{figure}[H]
    \centering
    \caption{Interações entre hiperparâmetros (\textit{Parallel Coordinates Plot}, Feature Set 2).}
    \includegraphics[width=0.95\textwidth]{Figuras/Paralelo - Feature 2.png}
    \par\small{Fonte: Autor (2025)}
    \label{fig:parallel_feature2}
\end{figure}

\FloatBarrier

A Tabela~\ref{tab:comparativo_variantes} resume os melhores resultados obtidos para cada conjunto de atributos. Em todos os casos, o processo de otimização contribuiu para redução do erro em relação aos modelos de referência, demonstrando a relevância do ajuste automatizado de hiperparâmetros.

\usepackage{makecell} % Adicione no preâmbulo se ainda não estiver



% Requires: \usepackage{array}
\begin{table}[h]
    \centering
    \caption{Resumo dos resultados de otimização das três variantes de entrada.}
    \begin{tabular}{|l|c|c|c|l|}
        \hline
        \textbf{Set} & 
        \shortstack{\textbf{Base}\\\textbf{(W/m²)}} & 
        \shortstack{\textbf{Otimizado}\\\textbf{(W/m²)}} & 
        \textbf{(\%)} & 
        \textbf{Configuração Ótima} \\
        \hline
        1 & 157,40 & \textbf{148,62} & 5,6 & 2 camadas, 16 neurônios, LR=0,0047, Sigmoid \\
        2 & 103,10 & \textbf{98,73} & 4,2 & 3 camadas, 26 neurônios, LR=0,0027, Tanh \\
        3 & 104,21 & \textbf{101,29} & 2,8 & 3 camadas, 25 neurônios, LR=0,0032, Tanh \\
        \hline
    \end{tabular}
    \par\small{Fonte: Autor (2025)}
    \label{tab:comparativo_variantes}
\end{table}

\FloatBarrier

Observa-se que, embora os ganhos absolutos sejam modestos, a consistência entre os três estudos reforça a estabilidade da metodologia. Em todos os casos, a combinação de \textbf{otimizador Adam}, função de ativação \textbf{Tanh ou Sigmoid} e taxas de aprendizado em torno de $3\times10^{-3}$ apresentou melhor desempenho. Além disso, redes mais rasas e compactas mantiveram desempenho superior, sugerindo que o problema em questão não demanda arquiteturas profundas para capturar sua dinâmica temporal.


Em síntese, os resultados da otimização indicam que:
\begin{itemize}
    \item A \textbf{taxa de aprendizado} é o hiperparâmetro mais sensível, exercendo impacto dominante sobre o RMSE em todas as variantes (importância média de 0,59).
    \item A \textbf{função de ativação} tem efeito secundário, mas relevante, com destaque para \textit{tanh} e \textit{sigmoid}, que favoreceram estabilidade de gradientes.
    \item O uso de \textbf{mecanismos de atenção} não trouxe melhoria significativa, sugerindo que a LSTM padrão já captura as dependências temporais relevantes.
    \item O número de camadas e o tamanho da camada oculta tiveram influência marginal, apontando que arquiteturas mais simples são mais eficientes para o problema em questão.
\end{itemize}



\input{5. Resultados/5.2 Modelo XGboost}
\section{Comparação entre LSTM e XGBoost}

\subsection{Desempenho por cenário}

\subsection{Análise por horizonte de previsão}

\subsection{Robustez às variáveis de previsão e ruído}

\subsection{Discussão crítica dos modelos}


\section{Síntese dos resultados}
% Quadro comparativo final + principais achados




% \input{cap5-resultados/01-previsao-ghi-metricas-graficos}
% \input{cap5-resultados/02-comparacao-baselines-ablation}
% \input{cap5-resultados/03-analise-clima-cenarios}
% \input{cap5-resultados/04-hems-resultados-economia-conforto}
% \input{cap5-resultados/05-limitacoes-e-analise-critica}

% ---------------------------------------
% CAPÍTULO 6 — Conclusão
% ---------------------------------------
\chapter{Conclusão}\label{chap:conclusao}
% \input{cap6-conclusao/01-conclusoes}
% \input{cap6-conclusao/02-trabalhos-futuros}

% ---------------------------------------
% Apêndices (opcional)
% ---------------------------------------
\appendix
\chapter{Códigos principais}\label{ap:codigos}
% \input{apendices/A-codigos-principais}

\chapter{Tabelas suplementares}\label{ap:tabelas}
% \input{apendices/B-tabelas-suplementares}



\bibliographystyle{delaeabnt}
\bibliography{bibliografia} % Nome do arquivo .bib

 \end{document}

