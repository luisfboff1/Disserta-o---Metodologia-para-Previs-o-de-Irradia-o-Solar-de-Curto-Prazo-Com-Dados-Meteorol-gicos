\section{Pré-processamento e Engenharia de Dados}

Após a seleção da base de dados, será realizado o pré-processamento das séries temporais de irradiância e variáveis meteorológicas. Esta etapa é fundamental para garantir a consistência e a confiabilidade do conjunto de dados a ser utilizado na modelagem, além de possibilitar a extração de informações adicionais relevantes por meio da engenharia de atributos. O processo será conduzido em múltiplas fases, descritas a seguir.

\subsection{Análise de completude e qualidade dos dados}

Inicialmente, será realizada a verificação da \textit{completude temporal} da base, avaliando-se o número de minutos observados em cada ano em comparação ao esperado para uma série contínua. Essa análise permitirá identificar anos com lacunas significativas e quantificar a proporção de registros efetivamente disponíveis. A inspeção da completude temporal é relevante, pois séries com baixa cobertura histórica tendem a prejudicar a capacidade de generalização dos modelos preditivos.

\subsection{Energia diária e definição do intervalo de análise}

Para definir a faixa horária mais adequada ao estudo, será calculada a energia diária total de irradiância incidente. A energia em um dia $d$ será obtida pela integração da irradiância ao longo do tempo, conforme a Equação (\ref{eq:energia_diaria}).

\begin{equation}
E_d = \int_{t \in d} SWD(t)\, dt \approx \sum_{i=1}^{N_d} SWD(t_i) \cdot \Delta t
\label{eq:energia_diaria}
\end{equation}
\noindent
Em que $E_d$ representa a energia diária total incidente em um dia $d$ (Wh/m$^2$); $SWD(t_i)$ é a irradiância global no instante $t_i$ (W/m$^2$); $\Delta t$ corresponderá ao passo temporal de integração (1 minuto); e $N_d$ será o número de amostras disponíveis no dia.

Antes do cálculo da energia, será verificado o fuso horário dos dados. Em diversas bases meteorológicas e radiométricas, os registros são fornecidos em tempo universal coordenado (UTC). Para que os valores façam sentido físico em relação ao nascer e ao pôr do sol locais, será necessário realizar a conversão para o fuso horário da estação de medição, no caso UTC$-3$. Esse ajuste assegurará que a distribuição horária da irradiância corresponda à realidade, permitindo que o modelo represente adequadamente os padrões locais.

A análise dos perfis de energia ao longo das 24 horas indicará o período em que existe energia efetivamente relevante para alimentar os modelos. Assim, será adotado esse intervalo como janela de observação, reduzindo-se a quantidade de valores nulos (noturnos) e concentrando a modelagem nos períodos mais significativos para aplicações fotovoltaicas.

\subsection{Tratamento de valores faltantes e lacunas temporais}

Após o recorte temporal, serão verificadas lacunas de diferentes magnitudes. O tratamento será realizado de forma seletiva:

\begin{itemize}
    \item \textbf{Lacunas curtas} (até 30 minutos): serão preenchidas por interpolação utilizando a função \texttt{interpolate} da biblioteca \texttt{pandas}. No caso do método linear com índice temporal, os valores intermediários serão calculados pela Equação (\ref{eq:interp_linear}), assumindo variação linear entre os pontos vizinhos.

    \begin{equation}
        y(t) = y(t_a) + \frac{y(t_b) - y(t_a)}{t_b - t_a} \cdot (t - t_a)
        \label{eq:interp_linear}
    \end{equation}

    Em que $y(t)$ será o valor interpolado para um instante intermediário $t$; $t_a$ representará o instante anterior válido à lacuna; e $t_b$ representará o instante posterior válido à lacuna.

    \item \textbf{Lacunas longas} (superiores a 30 minutos): dias contendo tais falhas serão descartados, de forma a não comprometer a consistência da série.
\end{itemize}

\subsection{Correção de valores espúrios}

Serão identificados valores fisicamente inválidos ou inconsistentes nas medições, tais como: irradiância solar negativa ($SWD < 0$), removida por não ter significado físico; umidade relativa fora do intervalo $[0, 100]\,\%$, corrigida por truncamento nos limites; e pressão atmosférica ou temperatura fora de faixas climatológicas plausíveis para a região (850–1100 hPa e $[-10, 50]^{\circ}C$, respectivamente), que serão descartadas ou substituídas por interpolação. Adicionalmente, valores iguais a zero durante o período diurno (07:00–20:00) serão tratados como falhas instrumentais e corrigidos por interpolação quando isolados.

\subsection{Reamostragem temporal}

Visando reduzir a variabilidade de alta frequência e alinhar a resolução da previsão com aplicações energéticas, as séries serão reamostradas para intervalos de 15 minutos. O valor médio da irradiância em cada janela será descrito pela Equação (\ref{eq:reamostragem}).

\begin{equation}
SWD^{15}(t) = \frac{1}{N}\sum_{i=1}^{N} SWD(t_i)
\label{eq:reamostragem}
\end{equation}

Em que $SWD^{15}(t)$ será a irradiância média reamostrada a cada 15 minutos; $SWD(t_i)$ corresponderão aos valores originais de irradiância na janela; e $N$ será o número de observações originais dentro da janela de 15 minutos.

\subsection{Variáveis derivadas a partir de modelos físicos}

Será utilizado o modelo de céu claro de \textit{Ineichen}, implementado na biblioteca \texttt{pvlib}, para estimar a irradiância teórica sem atenuação atmosférica ($SWD_{cs}$). A partir disso, será construído o índice de claridade $k_t^{*}$, conforme a Equação (\ref{eq:ktstar}).

\begin{equation}
k_t^{*}(t) = \frac{SWD(t)}{SWD_{cs}(t)}
\label{eq:ktstar}
\end{equation}

Em que $k_t^{*}(t)$ será o índice de claridade adimensional; $SWD(t)$ corresponderá à irradiância global medida no instante $t$ (W/m$^2$); e $SWD_{cs}(t)$ será a irradiância estimada para céu claro (W/m$^2$).

Os valores serão truncados no intervalo $[0,1]$, de forma a evitar distorções por inconsistências de medição. Este índice fornecerá uma medida da transparência atmosférica, capturando a influência de nuvens e aerossóis.


\subsection{Codificação de variáveis sazonais}

A variabilidade intradiária e anual da irradiância será incorporada por meio de uma mesma família de funções sazonais contínuas, evitando descontinuidades típicas de variáveis categóricas e garantindo padronização entre diferentes escalas temporais. A estratégia consistirá em normalizar a variável cíclica para o intervalo $[0,1]$ e, em seguida, aplicar uma função periódica suave com pico controlável por um parâmetro de fase. A forma geral será apresentada na Equação (\ref{eq:sazonal_unificada}).

\begin{equation}
\text{saz}(t) \;=\; \cos^{2}\!\left(\pi \cdot \Big(\tilde{x}(t) - \varphi\Big)\right),
\qquad
\tilde{x}(t) \;=\; \frac{x(t)-x_{\min}}{x_{\max}-x_{\min}}
\label{eq:sazonal_unificada}
\end{equation}

\noindent
Em que $\text{saz}(t)$ representará a codificação sazonal contínua e adimensional no instante $t$; $x(t)$ será a variável cíclica de interesse (hora do dia ou mês do ano); $\tilde{x}(t)$ será a versão normalizada de $x(t)$ para o intervalo $[0,1]$; $x_{\min}$ e $x_{\max}$ delimitarão o ciclo considerado; e $\varphi \in [0,1]$ será o parâmetro de \textit{fase}, que posicionará o máximo da função (pico) ao longo do ciclo.

A escolha de $\cos^{2}(\cdot)$ apresenta três vantagens práticas: (i) domínio naturalmente limitado em $[0,1]$, evitando escalas negativas; (ii) suavidade e derivabilidade, úteis a modelos baseados em gradiente; e (iii) controle explícito da posição do pico por meio de $\varphi$. Note-se que $\sin^{2}(\cdot)$ é equivalente a um deslocamento de fase de $\cos^{2}(\cdot)$; portanto, adotar-se-á uma única forma para padronizar a construção.

Do ponto de vista metodológico, a construção acima evitará descontinuidades entre valores adjacentes (por exemplo, entre mês=12 e mês=1), padronizará a amplitude em $[0,1]$ e permitirá alinhar os picos aos comportamentos físicos de interesse por meio de $\varphi$. Essa padronização também facilitará a interpretação e a comparação entre estudos, uma vez que a mesma forma funcional será aplicada às diferentes escalas temporais consideradas.


\subsection{Incorporação de resíduo ARIMA}

Adicionalmente, será ajustado um modelo autorregressivo integrado de médias móveis (ARIMA) simples sobre a série de irradiância. O ARIMA é um modelo estatístico clássico para séries temporais, capaz de capturar dependências lineares por meio de três componentes principais: (i) a parte autorregressiva (AR), que modela a série em função de valores passados; (ii) a parte de diferenciação (I), que garantirá a estacionariedade por meio de diferenças sucessivas; e (iii) a parte de médias móveis (MA), que considerará o efeito de choques ou erros de previsão passados. A previsão fornecida pelo ARIMA, denotada por $SWD_{ARIMA}(t)$, corresponderá a uma estimativa linear da irradiância baseada na dinâmica histórica da série.

A partir dessa previsão, será calculado o resíduo, definido na Equação (\ref{eq:arima_residuo}), que representará a parcela da variabilidade não explicada pelo modelo linear.

\begin{equation}
Res_{ARIMA}(t) = SWD(t) - SWD_{ARIMA}(t)
\label{eq:arima_residuo}
\end{equation}

\noindent
Em que $SWD_{ARIMA}(t)$ será a previsão de irradiância obtida pelo modelo ARIMA; e $Res_{ARIMA}(t)$ será o resíduo, correspondente à diferença entre a série observada e a previsão linear.

A motivação para incorporar tanto a previsão $SWD_{ARIMA}(t)$ quanto o resíduo $Res_{ARIMA}(t)$ como variáveis explicativas está no fato de que o ARIMA tende a capturar bem o comportamento médio e suave da série, mas não reproduz de forma satisfatória variações repentinas, como aquelas provocadas por nuvens de passagem rápida. Nesse contexto, espera-se que o resíduo contenha informação adicional sobre essas flutuações abruptas, fornecendo ao modelo de aprendizado de máquina um sinal explícito de discrepâncias entre o comportamento esperado (linear) e o observado (não linear).

\subsection{Integração de previsores do dia seguinte}

Para simular um cenário operacional realista, serão incorporadas ao modelo variáveis meteorológicas previstas para o dia seguinte. Como as bases radiométricas históricas não fornecem previsões NWP associadas, adotar-se-á a aproximação padrão de deslocar em um dia as séries observadas de temperatura, umidade relativa e pressão, conforme Equação (\ref{eq:prev_meteo}). Essas variáveis $X_{prev}(t)$ funcionarão como preditores exógenos que antecipam mudanças meteorológicas relevantes — como queda de temperatura ou aumento de umidade — que influenciam diretamente a irradiância. Essa estratégia busca refletir a prática comum em sistemas operacionais de previsão solar, nos quais modelos estatísticos e de deep learning utilizam previsões meteorológicas como insumo adicional.

Para representar a incerteza inerente a previsões reais, também serão geradas versões perturbadas por ruído gaussiano das mesmas variáveis ($X_{prev\_ruido}$). Essa construção permitirá avaliar a robustez do modelo diante de previsões imperfeitas, aproximando-se de cenários em que há erros de prognóstico típicos de NWP. O conjunto final incluirá, portanto, variáveis originais e derivadas que capturam tanto o estado atual quanto tendências antecipadas, ampliando o potencial explicativo do modelo. A utilidade prática dessas variáveis será determinada empiricamente na etapa de experimentação, por meio de métricas de desempenho e análises de importância.


