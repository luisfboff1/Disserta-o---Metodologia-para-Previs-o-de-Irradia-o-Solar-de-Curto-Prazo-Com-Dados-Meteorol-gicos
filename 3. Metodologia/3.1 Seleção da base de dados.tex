\section{Seleção da Base de Dados}

A escolha da base de dados constitui etapa fundamental em estudos de previsão de irradiância solar, uma vez que a qualidade e a completude das observações impactam diretamente a robustez dos modelos preditivos. Em linhas gerais, será buscada uma base que contemple as seguintes características desejáveis:

\begin{itemize}
    \item \textbf{Alta resolução temporal:} dados com amostragem em escala de minutos, possibilitando posterior agregação em janelas de 15 minutos e a captura de variações rápidas na irradiância;
    \item \textbf{Confiabilidade das medições:} séries provenientes de estações meteorológicas de referência, com instrumentação calibrada e documentação do processo de aquisição;
    \item \textbf{Amplitude de variáveis meteorológicas:} além da irradiância solar, a inclusão de variáveis exógenas como temperatura, umidade relativa e pressão atmosférica, que auxiliarão na modelagem das condições atmosféricas;
    \item \textbf{Extensão temporal suficiente:} histórico de múltiplos anos, permitindo contemplar diferentes estações, ciclos sazonais e padrões climáticos;
    \item \textbf{Disponibilidade pública ou institucional:} acesso transparente e reprodutível, garantindo a possibilidade de replicação do estudo por outros pesquisadores.
\end{itemize}

Embora a descrição detalhada da base utilizada seja apresentada no estudo de caso, nesta seção metodológica estabelecem-se as diretrizes que nortearão a sua seleção. Dessa forma, assegurar-se-á que o processo de escolha não se limite a uma decisão circunstancial, mas seja fundamentado em critérios técnicos alinhados às boas práticas da literatura.
